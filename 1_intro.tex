%\chapter{Introduction générale} \label{Introduction générale}

% COVER PAGE
\centerline{\bfseries\textcolor{bleusection}{ \Huge Introduction générale}}  

\bigskip

% Figure cover
\begin{tikzpicture}
  \def\ig{%
   \includegraphics[width=\linewidth,keepaspectratio]{./1_intro/cover.jpg}}
 \node [inner sep=0pt](mypicture) at (0,0) {\phantom{\ig}};
 \clip[rounded corners=5mm] ($(mypicture.south west)+(\bord,\bord)$) rectangle ($(mypicture.north east)-(\bord,\bord)$);
 \node[inner sep=0pt](mypicture) at (0,0) {\ig};
\end{tikzpicture}


% Table des matières intro
{\LARGE
\begin{enumerate}[label=\textcolor{bleusection}{\arabic*}{.}, leftmargin=2cm]
  \item \nameref{intro.1}
  \item \nameref{intro.2}
  \item \nameref{intro.3}
\end{enumerate}
}

% DEBUT INTRO
\clearpage
\pagestyle{intro}


\section{Une biodiversité marine en danger}\label{intro.1}

\subsection{Distribution de la biodiversité marine mondiale}\label{intro.1.1}

Les océans représentent plus de 70 \% de la surface de la Terre, abritent 50 à 80 \% des espèces vivantes de notre planète \citep{mora_how_2011, costello_global_2013}, et génèrent plus de 60 \% des services écosystémiques mondiaux \citep{millenium_ecosystem_assessment_ecosystem_2005, bindoff_changing_2019, ipbes_global_2019}, dont notamment la moitié de la production du dioxygène atmosphérique par photosynthèse. L’océan global est un organe clé de la régulation du climat, grâce à l’importance des flux thermiques et bio-géochimiques qu’il entretient avec l’atmosphère ; il est notamment responsable de l’absorption de plus de 25 \% des émissions annuelles de CO\textsubscript{2} d’origine anthropique \citep{heinze_ocean_2015}.

L’étendue géographie de l’océan global, du pôle Sud au pôle Nord, ainsi que sa très forte anisotropie verticale (notamment température et lumière) créent de nombreuses niches écologiques, peu à peu exploitées au cours de l’évolution par une incroyable diversité de formes vivantes. Si la répartition de la biodiversité océanique mondiale dépend du taxon considéré, elle est nettement plus importante en milieu côtier \citep{tittensor_global_2010}, et particulièrement forte en milieu tropical, notamment dans le « triangle de corail » \citep{sanciangco_habitat_2013} (Malaisie, Indonésie, Philippines et îles Salomon ; voir \autoref{figure_intro1}).


%%%%%%%%%%%%%%%%%%%%%%%%%%%%%%%%%%%%%%%%%%%%%%%%%%%%%%%%%%%%%
%%% Figure intro1: Cartographie de la richesse spécifique %%%
%%%%%%%%%%%%%%%%%%%%%%%%%%%%%%%%%%%%%%%%%%%%%%%%%%%%%%%%%%%%%
\begin{figure}[H]
	\begin{center}
	\includegraphics[width=\linewidth,keepaspectratio]{./1_intro/global_diversity_Tittensor2010}
		\caption[Cartographie de la richesse spécifique mondiale tous taxa confondus]{Cartographie de la richesse spécifique mondiale tous taxa confondus (d'après \citet{tittensor_global_2010}). Données compilées pour 11567 espèces appartenant à 13 taxa différents.}
	\label{figure_intro1}
\end{center}
\end{figure}

\subsection{Une biodiversité sous pression}\label{intro.1.2}

\subsubsection{Contexte de changement global}\label{intro.1.2.1}

La compréhension de la distribution des espèces ainsi que des forces structurant les assemblages ont toujours fasciné les biologistes depuis les premiers travaux de Darwin \citep{darwin_origin_1859}. La prise en compte des impacts anthropiques sur la biodiversité mondiale et la nécessité de mettre en place des plans de conservation efficaces pour sa préservation \citep{margules_systematic_2000} ont motivé la poursuite des études sur les patrons de biodiversité à différentes échelles \citep{rosa_multiscale_2017}. En effet, la très forte intensification des émissions de gaz à effet de serre depuis le début de l’ère industrielle a déjà contribué à d’importantes modifications climatiques à l’échelle globale, et tout semble indiquer que ce n’est que le début \citep{ipcc_climate_2014, ipcc_climate_2019}. Ces modifications climatiques, conjointement aux autres pressions anthropiques (déforestation, surpêche, pollutions chimiques…), ont déjà largement affecté la biosphère dans son ensemble et contribué à initier ce que l’on reconnaît aujourd’hui comme la « sixième crise d’extinction d’espèces » \citep{barnosky_has_2011, dirzo_defaunation_2014, ceballos_accelerated_2015}. Malheureusement, tous les scénarii semblent indiquer que cette érosion de la biodiversité va continuer tout du long du 21ème siècle \citep{pereira_scenarios_2010, tittensor_mid-term_2014, visconti_projecting_2016}.

Les océans ne sont pas épargnés par les impacts de « l’Anthropocène » \citep{mcgill_fifteen_2015}, et l’érosion de la biodiversité marine continue de s’accélérer \citep{mccauley_marine_2015, bindoff_changing_2019, ohara_mapping_2019}. En effet, il a été montré que la température est le principal facteur environnemental régissant la distribution des espèces marines \citep{tittensor_global_2010}, et les modifications de la température des océans risque de réarranger la distribution et les assemblages d’espèces à l’échelle globale \citep{pereira_scenarios_2010, tittensor_global_2010, poloczanska_responses_2016}. Les habitats marins côtiers sont particulièrement vulnérables dans la mesure où ils abritent une importante biodiversité \citep{halpern_global_2008} et sont directement soumis à une population humaine beaucoup plus dense que la moyenne (27 \% de la population mondiale vit à moins de 100 km de la côte \citep{kummu_over_2016}). Parmi eux, les récifs coralliens sont largement étudiés car ils contiennent à eux seuls environ 35 \% de la biodiversité marine mondiale \citep{reaka-kudla_biodiversity_2005} et sont largement impactés par le changement climatique \citep{hoegh-guldberg_coral_2007, death_27-year_2012, graham_predicting_2015, hughes_coral_2017}.

Bien que certains assemblages tempérés semblent plus robustes que les récifs coralliens aux changements globaux \citep{stuart-smith_stability_2010}, la majorité des biomes marins sont susceptibles d’être affectés \citep{waycott_accelerating_2009, marba_mediterranean_2014, telesca_seagrass_2015, wernberg_climate-driven_2016, halpern_recent_2019, ohara_mapping_2019}. Les résultats moins alarmistes en milieu tempéré pourrait être en partie expliqué par un manque de données pour ces habitats \citep{wernberg_impacts_2011}. En effet, la situation en 2008 indiquait que plus de 40 \% des océans subissaient déjà un fort impact anthropique cumulé \citep{halpern_global_2008}, et cet impact a significativement augmenté au cours de la dernière décennie pour près de 60 \% des océans \citep{halpern_recent_2019} (voir \autoref{figure_intro2}). Aujourd’hui, près de la totalité de l’océan mondial (97,7 \%) est affecté par plusieurs pressions anthropiques \citep{halpern_spatial_2015} et 83 \% de l’océan global abrite plus de 25 \% d’espèces menacées \citep{ohara_mapping_2019}.


%%%%%%%%%%%%%%%%%%%%%%%%%%%%%%%%%%%%%%%%%%%%%%%%%%%
%%% Figure intro2: Cartographie impacts cumulés %%%
%%%%%%%%%%%%%%%%%%%%%%%%%%%%%%%%%%%%%%%%%%%%%%%%%%%
\begin{figure}[H]
	\begin{center}
	\includegraphics[width=\linewidth,keepaspectratio]{./1_intro/cumulative_impacts_Halpern2019}
		\caption[Cartographie des impacts anthropiques cumulés à l’échelle mondiale]{Cartographie des impacts anthropiques cumulés à l’échelle mondiale (d'après \citet{halpern_recent_2019}).}
	\label{figure_intro2}
\end{center}
\end{figure}

\subsubsection{Pressions anthropiques locales}\label{intro.1.2.2}

Les changements globaux induits par les émissions massives de gaz à effet de serre depuis le début de l’ère industrielle ne sont malheureusement pas les seuls facteurs affectant la biodiversité, et les écosystèmes sont menacés à l’échelle globale par bon nombre de pressions anthropiques \citep{hoekstra_confronting_2004, halpern_global_2008}. Il a notamment été démontré que la fragmentation et la perte d’habitats ont un impact considérable sur leur biodiversité \citep{brooks_habitat_2002, haddad_habitat_2015}. Les habitats marins ne font une fois de plus pas exception à la règle, et bien qu’en apparence protégés de l’action de l’Homme, ils sont soumis à d’importantes pressions anthropiques locales \citep{micheli_cumulative_2013, halpern_spatial_2015, holon_predictive_2018} (voir \autoref{figure_intro3}). Parmi les principales pressions anthropiques, on dénombre notamment : la pêche (dégradation des habitats par chalutage \citep{hiddink_global_2017}, effondrement des stocks halieutiques à cause de la surpêche \citep{christensen_century_2014, essington_fishing_2015, link_global_2019} et impacts des engins de pêche laissés à l’abandon \citep{wilcox_understanding_2015, moschino_is_2019}, les aménagements côtiers (destruction des habitats et perturbation des régimes hydro-sédimentaires favorisant la sédimentation sur des habitats sensibles \citep{airoldi_effects_2003, holon_predictive_2018}), les pratiques agricoles (enrichissement en nitrates et phosphates modifiant la structure des communautés \citep{berger_effects_2003, savage_effects_2010}), les rejets de station d’épuration (pathogènes, matière organique et nutriments altèrent la qualité de l’eau et affectent les habitats limitrophes \citep{orth_global_2006, waycott_accelerating_2009}), le trafic maritime (collisions avec les cétacés \citep{peltier_monitoring_2019}, perturbation des vertébrés marins \citep{bruintjes_rapid_2016, simpson_anthropogenic_2016, bas_marine_2017,slabbekoorn_effects_2018}), le mouillage (dégradation des herbiers sous-marins \citep{short_natural_1996} et des récifs biogéniques \citep{ballesteros_mediterranean_2006} par l’action mécanique de l’ancre et de la chaîne \citep{milazzo_boat_2004}), les espèces exotiques envahissantes (une des principales causes d’extinction d’espèces \citep{bellard_alien_2016}, considérée comme l’un des plus grand défi en matière de conservation \citep{pysek_invasive_2010}), les rejets de macro-déchets plastiques (chaque année, plus de 300 millions de tonnes de déchets plastiques atterrissent dans l’océan global \citep{law_plastics_2017} et contaminent bon nombre d’espèces marines qui les ingèrent \citep{schuyler_global_2013, wilcox_threat_2015}).

%%%%%%%%%%%%%%%%%%%%%%%%%%%%%%%%%%%%%%%%%%%%%%%%%%%%%%%%%%
%%% Figure intro3: Illustration pressions anthropiques %%%
%%%%%%%%%%%%%%%%%%%%%%%%%%%%%%%%%%%%%%%%%%%%%%%%%%%%%%%%%%
\begin{figure}[H]
	\begin{center}
	\includegraphics[width=\linewidth,keepaspectratio]{./1_intro/pressions}
		\caption[Illustrations de quelques pressions anthropiques]{Illustrations de quelques pressions anthropiques. De gauche à droite et de haut en bas: rejet de station d’épuration, macro-déchets plastiques, algues filamenteuses recouvrant un récif coralligène, ancien engin de pêche coincé sur un récif coralligène, relève d’une ancre d’un bateau au mouillage dans l’herbier, présence d’une espèce exotique envahissante (\textit{Caulerpa taxifolia}) dans un herbier de posidonie (\textit{©Andromède Ccéanologie}).}
	\label{figure_intro3}
\end{center}
\end{figure}

\subsection{La Méditerranée: une exceptionelle biodiversité soumise à \\ d’importantes pressions}\label{intro.1.3}

Située entre l’Europe et l’Afrique, la mer Méditerranée est ce qu’il reste aujourd’hui de l’océan Téthys. Son histoire géologique intense, sa géomorphologie et ses conditions environnementales ont largement contribué à son histoire évolutive et à l’exceptionnelle biodiversité qui la caractérise depuis la fin de l’Eocène (-42Ma à -39Ma) \citep{boudouresque_marine_2004, renema_hopping_2008}. Aujourd’hui, alors qu’elle ne représente que 0,32 \% du volume de l’océan global, la mer Méditerranée abrite près de 7 \% de la biodiversité marine mondiale (17 000 espèces) \citep{coll_biodiversity_2010} dont environ un quart est endémique au bassin \citep{bianchi_rmesualtrsine_2000}. L’essentiel de cette biodiversité se concentre sur le plateau continental, notamment les poissons et les invertébrés benthiques \citep{coll_mediterranean_2012, katsanevakis_invading_2014} (voir \autoref{figure_intro4}). L’exploitation des ressources halieutiques a été un facteur primordial dans l’établissement et le développement des civilisations autour du bassin Méditerranéen au cours de l’histoire \citep{coll_biodiversity_2010}, mais elle représente aujourd’hui une menace pour cette biodiversité côtière. Par ailleurs, la Méditerranée n’est pas épargnée par les changements climatiques et pourrait même faire partie des régions les plus touchées \citep{giorgi_climate_2006, adloff_mediterranean_2015}. Les herbiers de posidonie (Posidonia oceanica) et les récifs coralligènes, qui figurent parmi les habitats marins les plus riches de Méditerranée \citep{boudouresque_marine_2004}, sont particulièrement sujets aux fortes pressions anthropiques globales et locales.

%%%%%%%%%%%%%%%%%%%%%%%%%%%%%%%%%%%%%%%%%%%%%%%%%%%%%%%
%%% Figure intro4: Richesse spécifique Méditerranée %%%
%%%%%%%%%%%%%%%%%%%%%%%%%%%%%%%%%%%%%%%%%%%%%%%%%%%%%%%
\begin{figure}[H]
	\begin{center}
	\includegraphics[width=\linewidth,keepaspectratio]{./1_intro/species_richness_Katsanevakis2014}
		\caption[Richesse spécifique de poissons et invertébrés en Méditerranée]{Richesse spécifique de poissons et invertébrés en Méditerranée \citep{katsanevakis_invading_2014}.}
	\label{figure_intro4}
\end{center}
\end{figure}

\subsubsection{Les herbiers de Posidonie : un habitat fragile aux multiples services écosystémiques}\label{intro.1.3.1}

La notion de services écosystémiques vise à quantifier l’ensemble des bénéfices que les humains tirent du fonctionnement et de l’intégrité des écosystèmes \citep{de_groot_global_2012}. Si les services attribués aux écosystèmes marins sont encore peu documentés \citep{townsend_challenge_2018}, plus de la moitié de la valeur du capital naturel et des services écosystémiques mondiaux sont attribués aux seuls herbiers sous-marins \citep{millenium_ecosystem_assessment_ecosystem_2005, ipbes_global_2019}. En effet, leurs rôles écologiques et économiques sont énormes : séquestration carbone dans les rhizomes et la matte, production primaire locale pour les herbivores, export de matière organique vers les habitats à faible production, oxygénation de la colonne d’eau, atténuation de la houle, fixation du sédiment et des matières en suspension, nurserie à poissons… (voir \autoref{figure_intro5}).

%%%%%%%%%%%%%%%%%%%%%%%%%%%%%%%%%%%%%%%%%%%%%%%%%%%%%%%
%%% Figure intro5: Richesse spécifique Méditerranée %%%
%%%%%%%%%%%%%%%%%%%%%%%%%%%%%%%%%%%%%%%%%%%%%%%%%%%%%%%
\begin{figure}[H]
	\begin{center}
	\includegraphics[width=\linewidth,keepaspectratio]{./1_intro/services_herbiers_posidonies_Boudouresque}
		\caption[Services écosystémiques rendus par les herbiers sous-marins]{Services écosystémiques rendus par les herbiers sous-marins (source : C.F. Boudouresque).}
	\label{figure_intro5}
\end{center}
\end{figure}

La posidonie (Posidonia oceanica), une espèce protégée endémique de Méditerranée, forme de grandes prairies sous-marines entre la surface et 40 m de fond (voir \autoref{figure_intro6}). De par sa proximité à la côte et à la surface, cette espèce souffre tout particulièrement des effets des changements globaux \citep{marba_mediterranean_2014} et des pressions locales telles que l’augmentation de la population et de l’urbanisation côtières, responsables de la destruction de cet habitat fragile \citep{montefalcone_human_2010, marba_mediterranean_2014, holon_impact_2015, telesca_seagrass_2015}. Les herbiers de posidonie sont reconnus comme « habitat d’intérêt spécifique » en termes de biodiversité par la Directive Européenne sur les Habitats (Directive Habitats 92/43/CEE).

%%%%%%%%%%%%%%%%%%%%%%%%%%%%%%%%%%%%%%%%%%%%%%%%%%%%%%%%%%
%%% Figure intro6: Illustrations herbiers de posidonie %%%
%%%%%%%%%%%%%%%%%%%%%%%%%%%%%%%%%%%%%%%%%%%%%%%%%%%%%%%%%%
\begin{figure}[H]
	\begin{center}
	\includegraphics[width=\linewidth,keepaspectratio]{./1_intro/herbiers_posidonie}
		\caption[Illustrations d’un herbier de Posidonie]{Illustrations d’un herbier de Posidonie (de haut en bas et de gauche à droite : herbier avec quelques castagnoles (\textit{Chromis chromis}) ; graine germée de posidonie ; herbier avec une grande nacre (\textit{Pinna nobilis}) ; plant de posidonie avec ses graines) (\textit{©Andromède Océanologie}).}
	\label{figure_intro6}
\end{center}
\end{figure}

\subsubsection{Les récifs coralligènes : une biodiversité semblable aux récifs coralliens}\label{intro.1.3.2}

Les récifs coralligènes sont produits par l’accumulation de plus de 1500 espèces d’algues calcaires encroûtantes et d’animaux bio-constructeurs (polychètes, bryozoaires et gorgonaires)  \citep{ballesteros_mediterranean_2006}; ce sont les seules formations calcaires d’origine biogénique en Méditerranée \citep{ballesteros_mediterranean_2006}, et leur diversité et productivité sont similaires à celles des récifs coralliens \citep{bianchi_biocostruzione_2001}. Ces récifs sont des niches écologiques importantes pour un grand nombre d’espèces mobiles : poissons, crustacés, échinodermes, mollusques, tuniqués \citep{ballesteros_mediterranean_2006} (voir \autoref{figure_intro7}). Comme les herbiers de posidonie, l’habitat « récifs coralligènes » est reconnu comme « habitat d’intérêt spécifique » en termes de biodiversité par la Directive Européenne sur les Habitats (Directive Habitats 92/43/CEE).

%%%%%%%%%%%%%%%%%%%%%%%%%%%%%%%%%%%%%%%%%%%%%%%%%%%%%%%%
%%% Figure intro7: Illustrations récifs coralligènes %%%
%%%%%%%%%%%%%%%%%%%%%%%%%%%%%%%%%%%%%%%%%%%%%%%%%%%%%%%%
\begin{figure}[H]
	\begin{center}
	\includegraphics[width=\linewidth,keepaspectratio]{./1_intro/recifs_coralligenes}
		\caption[Illustrations des assemblages des récifs coralligènes]{Illustrations des assemblages des récifs coralligènes (\textit{©Andromède Océanologie}).}
	\label{figure_intro7}
\end{center}
\end{figure}

Bien que situés à des profondeurs allant de 20 à 120 m, les récifs coralligènes ne sont pas exempts des effets des multiples pressions anthropiques qui affectent la biodiversité marine. Notamment, cet habitat est particulièrement affecté par l’accroissement de sédimentation dus aux activités anthropiques côtières et aux modifications de régimes hydro-sédimentaires \citep{airoldi_effects_2003}, ainsi que par la pêche et le mouillage \citep{ballesteros_mediterranean_2006}. 

\medskip

\setlength{\fboxsep}{3pt}
\setlength{\fboxrule}{0.6pt}
\noindent\framebox{%
  \begin{minipage}{\linewidth}
    La\textbf{ biodiversité marine} est d’une \textbf{exceptionnelle richesse}, et l’Humanité entière dépend des nombreux \textbf{services écosystémiques} qu’elle rend. Pourtant, de nombreuses pressions anthropiques attaquent et érodent cette biodiversité, notamment en milieu côtier, où les densités de populations humaines et la biodiversité sont les plus élevées. \textbf{La Méditerranée est le parfait exemple} de cette co-occurrence de hauts niveaux de biodiversité et de pression, dans un espace géographique semi-fermé et restreint. Il est urgent de réagir et de mettre en place des \textbf{moyens de conservation efficaces} pour \textbf{préserver les écosystèmes marins}, mais la méconnaissance de ce milieu contribue à notre passivité face à son altération. Il est donc indispensable de \textbf{mettre en place des réseaux de surveillance} afin d’étudier et suivre l’état de santé des habitats marins, pour anticiper les changements et \textbf{assister les décisionnaires} à gérer les espaces et les espèces.
  \end{minipage}}

\newpage

\section{Surveillance de la biodiversité marine}\label{intro.2}

Compte tenu de la crise climatique actuelle et de la sensibilité des assemblages marins aux diverses pressions anthropiques, il est plus que jamais nécessaire d’étudier et de suivre les dynamiques spatio-temporelles de la biodiversité marine afin de mettre en place des mesures de conservation efficaces \citep{magris_integrating_2014, klein_shortfalls_2015}. La surveillance globale de la biodiversité et de son évolution, notamment face aux changements climatiques, nécessite de s’accorder sur les \textbf{variables essentielles pour la biodiversité} (« Essential Biodiversity Variables », EBVs) à mesurer pour quantifier ces changements \citep{pereira_essential_2013, navarro_monitoring_2017, schmeller_operational_2017, haase_next_2018, kissling_building_2018}. L’intérêt de la conceptualisation des EBVs est né d’un besoin de structurer et d’harmoniser les données relatives à la surveillance de la biodiversité à différentes échelles \citep{kissling_building_2018} afin de répondre aux objectifs de la Convention sur la Diversité Biologique (CDB 2010). Aujourd’hui, ces EBVs et les indicateurs qui en découlent doivent permettre de répondre aux besoins essentiels de la biodiversité dans le cadre de l’Agenda 2030 et ses objectifs de développement durable (Biodiversity and the 2030 Agenda for Sustainable Development). Elles constituent un premier niveau d’abstraction entre les \textbf{observations brutes} et les \textbf{indicateurs de biodiversité}, on en distingue six classes : composition génétique, dynamique et distribution de populations, traits spécifiques, composition des communautés, fonctionnement et structure de l’écosystème \citep{pereira_essential_2013}. Dans le cadre de ce travail doctoral, nous nous concentrerons essentiellement sur la distribution des espèces et la structure des habitats.

\subsection{Indicateurs de biodiversité}\label{intro.2.1}

Si fondamentalement, elle cherche à caractériser la richesse biologique de notre planète, la notion de « biodiversité » semble bien complexe et protéiforme. En effet, elle inclut différentes échelles et différentes mesures quantitatives et qualitatives, c’est pourquoi il est extrêmement difficile de l’exprimer avec une seule métrique, et que de très nombreux indices ont été développés \citep{teixeira_catalogue_2016}. Dans ce manuscrit, nous nous intéresserons à des indicateurs classiques d’évaluation des assemblages écologiques ainsi qu’à des indicateurs moins consensuels visant à décrire la structure de l’habitat. Plus particulièrement, nous utiliserons les indicateurs suivants pour caractériser les habitats marins :

\begin{itemize}
    \item \textbf{Diversité Taxonomique} (ou « richesse spécifique ») : mesure la plus simple de la biodiversité, elle correspond au nombre d’espèces différentes observées dans un espace donné et à un moment donné. Ce nombre n’a jamais vocation à être exhaustif, et se limite souvent à ce qui est facilement observable (ex : la diversité bactérienne est plus difficile à mesurer que la diversité d’oiseaux) ;
    
    \item \textbf{Indice de Shannon} \citep{magurran_measuring_2004} : semblable à la diversité taxonomique, il dépend non seulement du nombre d’espèces différentes observées, mais également de leur abondance relative. En effet, il convient de pouvoir distinguer deux communautés composées des mêmes espèces mais dont l’une aurait des abondances équitablement réparties entre espèces, et l’autre dominée par une ou plusieurs espèces. Cet indicateur reprend la forme de l’entropie, il est calculé comme suit :
    
    \begin{equation}
	    S_j=-\sum_{i}p_{ij} log(p_{ij})
	    \label{eqintro.1}
    \end{equation}
    
    Avec \textit{p\textsubscript{ij}} la prévalence de l’espèce $i$ au sein du site $j$.
    
    \item \textbf{Diversité fonctionnelle} : toutes les espèces présentes des caractéristiques morphologiques (taille, forme, biomasse…) et comportementales (relations trophiques, mode de reproduction, mobilité, migrations…) très différentes et bien souvent uniques, appelées « traits fonctionnels ». De fait, les espèces ne sont pas interchangeables et il convient de pouvoir distinguer des assemblages en prenant en compte cette diversité fonctionnelle, essentielle au bon fonctionnement des écosystèmes et à la provision de services écosystémiques dont dépendent les humains \citep{hooper_effects_2005, cadotte_using_2009, clemente_identifying_2010, faith_evosystem_2010}. Une multitude d’indices ont été développés afin de quantifier cette diversité fonctionnelle \citep{petchey_functional_2002, magurran_measuring_2004, mouchet_towards_2008, villeger_new_2008} ;
    
    \item \textbf{Structure de l’habitat} : si ses effets sur la biodiversité ne sont pas systématiquement négatifs (Fahrig, 2017), la \textbf{fragmentation} des habitats est connue pour jouer un rôle important dans la structuration et la dynamique des assemblages écologiques (Wilson et al., 2016; Crooks et al., 2017). La fragmentation peut se mesurer avec différents indicateurs paysagers (De Montis et al., 2017). Par ailleurs, de nombreuses études ont montré que la \textbf{complexité structurale} de l’habitat avait un effet important sur la structuration des communautés, notamment l’abondance et la diversité des espèces marines \citep{luckhurst_analysis_1978, gratwicke_relationship_2005, harborne_biotic_2011, meager_topographic_2011, kovalenko_habitat_2012, graham_importance_2013, rees_abiotic_2014, darling_relationships_2017}. La complexité structurale se mesure généralement grâce à la \textbf{rugosité} de l’habitat \citep{friedman_multi-scale_2012, dustan_digital_2013, leon_measuring_2015} ou sa \textbf{dimension fractale} \citep{yanovski_structural_2017, young_cost_2017, fukunaga_integrating_2019}. 
    
\end{itemize}

\setlength{\fboxsep}{3pt}
\setlength{\fboxrule}{0.6pt}
\noindent\framebox{%
  \begin{minipage}{\linewidth}
    La \textbf{biodiversité} est une notion complexe qu’on ne peut pas synthétiser en un seul indicateur. Dans le cas de \textbf{suivis écologiques} et l’étude des assemblages, il convient de mesurer différents compartiments et \textbf{calculer différents indicateurs} pour fournir une évaluation satisfaisante de \textbf{l’état d’un écosystème} et de son \textbf{fonctionnement}.
  \end{minipage}
}

\subsection{Méthodes d’acquisition d’images et de données cartographiques marines}\label{intro.2.2}

Afin d’évaluer la biodiversité marine et de quantifier les effets des pressions anthropiques, il est indispensable de \textbf{cartographier la présence} des espèces marines et de \textbf{quantifier leur abondance} à différentes échelles (principe des EBVs de distribution de populations) :

\begin{itemize}
    \item \textbf{Inventaires à méso/macro-échelle} avec une \underline{longue période de retour} pour étudier les facteurs environnementaux régissant la \underline{distribution des espèces}, et quantifier à plus large échelle les effets des pressions anthropiques ;
    
    \item \textbf{Inventaires à micro-échelle} avec une \underline{courte période de retour} pour comprendre la \underline{complexité} \underline{des assemblages} de certains habitats et détecter localement des changements précoces dans les équilibres écologiques.
\end{itemize}

Ces inventaires sont réalisés à l’aide de différentes méthodes d’acquisition, notamment des méthodes d’acquisition d’images qui permettent de déterminer la nature des assemblages et de cartographier l’étendue géographique des espèces. Cependant, l’étude des milieux marins est fortement contrainte par la présence de l’eau, qui limite considérablement l’accès aux habitats marins mais contribue au développement de méthodes cartographiques innovantes. Parmi elles, il convient de distinguer les méthodes cartographiques à méso / macro-échelle par télédétection, et à micro-échelle par proxy-détection et mesures \textit{in situ}.

\subsubsection{Cartographie à méso/macro-échelle par télédétection}\label{intro.2.2.1}

Plusieurs méthodes de télédétection sont utilisées pour cartographier les fonds marins à méso/macro-échelle : les images satellite et aériennes, les images sonar et sondeur, et les caméras tractées.

\paragraph{Imagerie satellite et aérienne}

L’accessibilité croissante d’images satellite gratuites avec des périodes de retour de plus en plus courtes, une haute voir très haute résolution et des capteurs multi-spectraux de plus en plus résolus a permis de démocratiser l’utilisation de la télédétection dans de nombreux domaines. Son utilisation s’est largement démocratisée en écologie terrestre, notamment à des fins de conservation avec la cartographie des variables essentielles pour la biodiversité \citep{pettorelli_framing_2016, luque_improving_2018, jetz_essential_2019}. Plus particulièrement, la télédétection permet aujourd’hui de cartographier la richesse spécifique des forêts \citep{feret_mapping_2014, vaglio_laurin_biodiversity_2014, baldeck_operational_2015} et la structure spatiale des communautés végétales \citep{rocchini_measuring_2018}. Elle est également utilisée pour d’autres applications environnementales, notamment en surveillance des océans \citep{devi_applications_2015} et des habitats marins \citep{hedley_remote_2016, mccarthy_satellite_2017, appolloni_use_2020, purkis_remote_2018}. Cependant, la télédétection satellitaire appliquée à la cartographie marine est contrainte par les propriétés absorbantes de l’eau et se limite aux habitats peu profonds et aux eaux claires \citep{purkis_remote_2018}. Par ailleurs, les conditions de mer et l’inclinaison du soleil au moment de l’acquisition doivent permettre de limiter la réflexion à la surface de l’eau pour pouvoir exploiter les images (mer calme et soleil au zénith). En eaux peu profondes et avec de bonnes conditions de mer, il est possible de cartographier finement des récifs coralliens ou des herbiers (voir \autoref{figure_intro8}).

%%%%%%%%%%%%%%%%%%%%%%%%%%%%%%%%%%%%%%%%%%%%%
%%% Figure intro8: Télédétection herbiers %%%
%%%%%%%%%%%%%%%%%%%%%%%%%%%%%%%%%%%%%%%%%%%%%
\begin{figure}[H]
	\begin{center}
	\includegraphics[width=0.7\linewidth,keepaspectratio]{./1_intro/remote_sensing_Koedsin2016}
		\caption[Exemple de cartographie d’herbiers par télédétection]{Exemple de cartographie d’herbiers dérivée d’images Worldview-2 et de données de référence \textit{in situ} \citep{koedsin_integrated_2016}.}
	\label{figure_intro8}
\end{center}
\end{figure}

De façon similaire à la télédétection satellitaire, l’acquisition de données aériennes s’est largement développée en écologie grâce à la démocratisation des drones \citep{ivosevic_use_2015}, qui permettent d’acquérir à moindre coût des images aériennes à très haute résolution. Ce type d’image est notamment utilisé en milieu corallien pour cartographier les récifs \citep{casella_mapping_2017, collin_very_2018}. 

\paragraph{Imageries sondeur et sonar : une échographie des fonds marins}

Bien que l’eau soit translucide, elle absorbe une grande proportion des rayons lumineux qui la traversent \citep{wozniak_light_2007}. Cette propriété limite considérablement les applications de la télédétection satellitaire et aérienne, qui n’est plus applicable dès lors que la profondeur ou la turbidité devient trop importante. Dans ce cas, la télédétection acoustique active (émission d’un signal de fréquence et d’intensité connues, et mesure de la réponse) à l’aide d’un capteur immergé représente une alternative à la télédétection satellite ou aérienne. Le sonar latéral et le sondeur multifaisceaux sont tous les deux des méthodes de télédétection acoustique active couramment utilisées pour cartographier les fonds marins \citep{saxena_review_1999, brown_benthic_2011} (voir \autoref{figure_intro9}). 

%%%%%%%%%%%%%%%%%%%%%%%%%%%%%%%%%%%%
%%% Figure intro9: Sonar sondeur %%%
%%%%%%%%%%%%%%%%%%%%%%%%%%%%%%%%%%%%
\begin{figure}[H]
	\begin{center}
	\includegraphics[width=\linewidth,keepaspectratio]{./1_intro/sonar_sondeur}
		\caption[Imageries sonar et sondeur multi-faisceaux]{Imagerie par sonar (gauche) et sondeur multi-faisceaux (droite) (\textit{©Andromède Océanologie}).}
	\label{figure_intro9}
\end{center}
\end{figure}

Si les deux techniques utilisent toutes deux l’acoustique, elles fonctionnent différemment et fournissent des résultats de nature différentes :

\begin{itemize}
    \item \textbf{Sonar latéral} : il émet un cône d’impulsions sonores d’environ 100 – 500 kHz en direction du fond et analyse l’intensité des réflexions avec une série d’hydrophones \citep{brown_benthic_2011}. Il en ressort une cartographie de l’état de surface et la nature du fond, avec un signal d’autant plus fort que la surface du fond est dense et lisse (voir \autoref{figure_intro10} gauche);
    
    \item \textbf{Sondeur multifaisceaux} : il émet un cône d’impulsions sonores comme le sonar latéral, mais celui-ci mesure le temps mis par chaque impulsion pour traverser la colonne d’eau, se réfléchir sur le fond et revenir, et en déduit la profondeur en chaque point d’impact \citep{brown_benthic_2011}. Il en ressort une cartographie bathymétrique (voir \autoref{figure_intro10} droite).
\end{itemize}


%%%%%%%%%%%%%%%%%%%%%%%%%%%%%%%%%%%%%%%%%%%%%%%%%%
%%% Figure intro10: Acquisitions sonar sondeur %%%
%%%%%%%%%%%%%%%%%%%%%%%%%%%%%%%%%%%%%%%%%%%%%%%%%%
\begin{figure}[H]
	\begin{center}
	\includegraphics[width=\linewidth,keepaspectratio]{./1_intro/acquisitions_acoustiques}
		\caption[Exemples d’acquisitions acoustiques par sonar latéral et sondeur multi-faisceaux]{Exemples d’acquisitions acoustiques par sonar latéral et sondeur multi-faisceaux (à gauche : traces de mouillage dans l’herbier visibles au sonar ; à droite : reproduction de la bathymétrie de Saint-Jean-Cap-Ferrat au sondeur multi-faisceaux) (\textit{©Andromède Océanologie}).}
	\label{figure_intro10}
\end{center}
\end{figure}

Ces deux techniques sont complémentaires, elles permettent de définir des zones géographiques homogènes, identifiables par l’analyse de l’image sonar et par des points de vérité terrain collectés ponctuellement sur la zone d’étude (Brown et al., 2011). 

\paragraph{Caméra tractée}

De la même manière qu’un sonar est tracté derrière un bateau de sorte à ce qu’il navigue à une dizaine de mètres au-dessus du fond, il est possible de tracter une caméra fixée sur un dispositif lui permettant de naviguer entre deux eaux et rester à distance réduite du fond \citep{rende_advances_2015}. Ce type d’acquisition photo ou vidéo permet de réaliser des assemblages photos ou des reconstructions trois dimensions (3D) des fonds par photogrammétrie et de cartographier les habitats de profondeur intermédiaire (10---40 m de profondeur). Comme pour le sonar latéral, il est possible d’estimer précisément la position géographique de la caméra à partir d’un positionnement GPS du bateau, du cap, de la longueur du câble et de la profondeur du dispositif (voir \autoref{figure_intro11}).

%%%%%%%%%%%%%%%%%%%%%%%%%%%%%%%%%%%%%%
%%% Figure intro11: Caméra tractée %%%
%%%%%%%%%%%%%%%%%%%%%%%%%%%%%%%%%%%%%%
\begin{figure}[H]
	\begin{center}
	\includegraphics[width=0.7\linewidth,keepaspectratio]{./1_intro/towed_camera_Rende2015}
		\caption[Prototype de caméra tractée pour la cartographie d’habitats de profondeur intermédiaire]{Prototype de caméra tractée pour la cartographie d’habitats de profondeur intermédiaire \citep{rende_advances_2015}.}
	\label{figure_intro11}
\end{center}
\end{figure}

\setlength{\fboxsep}{3pt}
\setlength{\fboxrule}{0.6pt}
\noindent\framebox{%
  \begin{minipage}{\linewidth}
    Si les données obtenues par télédétection, quelle que soit leur nature, permet de réaliser des cartographies à \textbf{méso/macro-échelle} pour un \textbf{coût et un temps d’acquisition relativement faibles}, elles ne permettent pas d’étudier les phénomènes écologiques qui se produisent à \textbf{micro-échelle} ni de \textbf{suivre la composition} des assemblages dans le temps.
  \end{minipage}
}

\subsubsection{Cartographie micro-échelle par proxy-détection et mesures in situ}\label{intro.2.2.2}

Si la télédétection permet d’étudier la répartition des espèces et des habitats dans l’espace, certaines études nécessitent de collecter de la donnée cartographique à plus fine échelle, comme par exemple la reconnaissance et la mesure d’espèces. Pour ces études, il est possible de réaliser des mesures in situ ou à proximité (proxy-détection) en rapprochant le capteur du sujet.

\paragraph{Relevés plongeur}

A la manière d’un botaniste ou d’un géologue qui effectue ses relevés terrain, l’Homme peut bénéficier d’outils de plongée pour s’immerger et réaliser lui-même l’acquisition de données. Si la plongée en apnée ou sous cloche existe depuis l’Antiquité, il faudra attendre la fin du 18ème siècle pour voir arriver les premiers scaphandres permettant de se déplacer sous l’eau. Plus récemment, la plongée militaire et la plongée technique ont contribué à mettre au point des scaphandres recycleurs autonomes permettant de rester plusieurs heures sous l’eau et d’atteindre des profondeurs pouvant dépasser les 100 m de fond \citep{sieber_review_2010}. Parmi la multitude de protocoles possibles, la plongée permet notamment de réaliser des relevés photographiques et de cartographier les habitats par télémétrie acoustique.

\noindent\textbf{Relevés photographiques}

Les relevés photographiques in situ permettent de rendre compte de l’état d’un habitat à un instant donné, et de produire une évaluation qualitative (rendu visuel) ou quantitative (analyse ou interprétation d’images). Ils permettent notamment de réaliser des assemblages photo 2D ou des reconstructions 3D (par photogrammétrie) afin de cartographier les habitats, quantifier le fractionnement, positionner les différentes espèces dans l’espace, et assurer un suivi dans le temps. Par ailleurs, la standardisation des conditions d’acquisition (distance, éclairement) permettent de réaliser des protocoles d’échantillonnage par quadrats photographiques pour évaluer et suivre le biodiversité des habitats benthiques \citep{deter_rapid_2012} (voir \autoref{figure_intro12}).

%%%%%%%%%%%%%%%%%%%%%%%%%%%%%%%%%%%%%%%%%%%%%%
%%% Figure intro12: Quadrat photographique %%%
%%%%%%%%%%%%%%%%%%%%%%%%%%%%%%%%%%%%%%%%%%%%%%
\begin{figure}[H]
	\begin{center}
	\includegraphics[width=\linewidth,keepaspectratio]{./1_intro/quadrat}
		\caption[Quadrat photographique pour l’analyse de la biodiversité benthique par un taxonomiste]{Quadrat photographique pour l’analyse de la biodiversité benthique par un taxonomiste (\textit{©Andromède Océanologie}).}
	\label{figure_intro12}
\end{center}
\end{figure}

\noindent\textbf{Cartographie par télémétrie acoustique}

Le signal GPS n’étant évidemment pas accessible sous l’eau, tout positionnement ne peut alors être que relatif à une position connue (par cartographie sondeur préalable, position en surface ou bien référentiel arbitraire) et déterminé par interférométrie acoustique Ultra-Short Base Line (USBL). Ce type de technologie permet de déterminer un cap et une distance entre un émetteur et un récepteur, et donc de connaître la position d’un objet relativement à un point fixe. A l’aide de cette technologie, il est ainsi possible de cartographier des habitats marins sur une zone réduite (quelques dizaines à quelques centaines de m²) en disposant une antenne de réception fixe et en pointant la limite des objets avec un émetteur (voir \autoref{figure_intro13}). Les cartographies produites à différents pas de temps peuvent ensuite être géoréférencées à l’aide de points de position géographique connue (roches, aménagements,…) ou bien simplement alignées et comparées, comme c’est fait dans le cas de suivis temporels d’herbiers de posidonie \citep{descamp_underwater_2005, descamp_fast_2011}.

%%%%%%%%%%%%%%%%%%%%%%%%%%%%%%%%%%%%%%%%%%%%%
%%% Figure intro13: Télémétrie acoustique %%%
%%%%%%%%%%%%%%%%%%%%%%%%%%%%%%%%%%%%%%%%%%%%%
\begin{figure}[H]
	\begin{center}
	\includegraphics[width=0.8\linewidth,keepaspectratio]{./1_intro/telemetrie}
		\caption[Plongeur entrain de cartographier un herbier de posidonie à l’aide d’un télémètre acoustique]{Plongeur entrain de cartographier un herbier de posidonie à l’aide d’un télémètre acoustique (\textit{©Andromède Océanologie}).}
	\label{figure_intro13}
\end{center}
\end{figure}

\setlength{\fboxsep}{3pt}
\setlength{\fboxrule}{0.6pt}
\noindent\framebox{%
  \begin{minipage}{\linewidth}
    Si la plongée scientifique permet au naturaliste de faire \textbf{lui-même ses observations} et d’acquérir de la donnée \textbf{très fine}, elle est malheureusement contrainte par la \textbf{physiologique humaine} et les temps de décompression, qui augmentent exponentiellement avec la profondeur et peuvent pénaliser un plongeur durant \textbf{plusieurs heures} pour quelques \textbf{dizaines de minutes de travail au fond}. C’est pourquoi dans les cas plus extrêmes ou lorsque la complexité de la tâche le permet, les robots sont préférés au plongeur.
  \end{minipage}
}

\paragraph{Les robots: ROV et AUV}

Les robots sous-marins sont des plateformes mobiles au service de l’utilisateur, souvent adaptables pour la tâche souhaitée en les équipant du capteur ou de l’outil adéquat pour mener à bien sa mission. Si le coup de mise en œuvre est généralement élevé par rapport à un plongeur (coût du robot, déploiement par un navire océanographique adéquat), ils peuvent travailler plus profond (jusqu’à plusieurs centaines de mètres) et sans limite de temps due à la physiologie. Il en existe deux sortes : les Remotely Operated Vehicles (ROVs) et les Autonomous Underwater Vehicles (AUVs) \citep{bogue_underwater_2015}. Les ROVs sont télé-opérés depuis la surface via un câble, appelé communément « ombilical », de diamètre variable selon qu’il véhicule uniquement les ordres de navigation ou bien également l’alimentation électrique. Si l’ombilical qui les relie au navire leur confère un certain handicap, il est possible de les piloter en temps réel et d’adapter la navigation et l’acquisition en fonction de ce que voit le pilote en surface (voir \autoref{figure_intro14}).

%%%%%%%%%%%%%%%%%%%%%%%%%%%%%
%%% Figure intro14: ROV3D %%%
%%%%%%%%%%%%%%%%%%%%%%%%%%%%%
\begin{figure}[H]
	\begin{center}
	\includegraphics[width=0.7\linewidth,keepaspectratio]{./1_intro/ROV3D_Drap2015}
		\caption[ROV3D : un robot d’acquisition photogrammétrique pour l’archéologie sous-marine]{ROV3D : un robot d’acquisition photogrammétrique pour l’archéologie sous-marine \citep{drap_rov_2015}.}
	\label{figure_intro14}
\end{center}
\end{figure}

Comme leur nom l’indique, les AUVs sont autonomes et doivent donc être programmés pour réaliser un parcours d’acquisition précis et maintenir une distance au fond tout en évitant les éventuels obstacles, en utilisant des technologies de positionnement \citep{johnson-roberson_generation_2010, bonin-font_towards_2016} (voir \autoref{figure_intro15}). Ces robots permettent de réaliser des acquisitions de manière entièrement autonome une fois déployés, mais ils demandent une technologie et un temps de développement généralement beaucoup plus onéreux que les ROVs.

%%%%%%%%%%%%%%%%%%%%%%%%%%%
%%% Figure intro15: AUV %%%
%%%%%%%%%%%%%%%%%%%%%%%%%%%
\begin{figure}[H]
	\begin{center}
	\includegraphics[width=0.7\linewidth,keepaspectratio]{./1_intro/AUV_Johnson-Roberson2010}
		\caption[Exemple d’un robot autonome (AUV) pour de l’acquisition d’images à grande échelle]{Exemple d’un robot autonome (AUV) pour de l’acquisition d’images à grande échelle \citep{johnson-roberson_generation_2010}.}
	\label{figure_intro15}
\end{center}
\end{figure}

\setlength{\fboxsep}{3pt}
\setlength{\fboxrule}{0.6pt}
\noindent\framebox{%
  \begin{minipage}{\linewidth}
    Les robots sous-marins sont d’excellents outils pour acquérir des données photo sur des \textbf{grandes surfaces} (AUV) ou à des \textbf{grandes profondeurs} (ROV) ; mais leur \textbf{maniabilité réduite}, leur \textbf{coût} de développement et leur mise en œuvre rendent bien souvent le \textbf{plongeur autonome plus pertinent} et compétitif (jusqu’à une certaine profondeur), notamment depuis la démocratisation des scaphandres recycleurs. 
  \end{minipage}
}

\subsection{Les réseaux de surveillance en Méditerranée Française}\label{intro.2.3}

Les réseaux de surveillance ont pour objectif de réaliser des suivis écologiques de certaines espèces ou certains habitats à l’échelle régionale ou globale, en standardisant les méthodes de collecte et de traitement de données. L’analyse des données de ces réseaux permet de comparer la qualité écologique des différentes stations de mesures et de mesurer leur évolution dans le temps. De nombreux réseaux de surveillance existent en Méditerranée Française, dont une partie sont financés par l’Agence de l’Eau Rhône-Méditerranée-Corse et rassemblés sur la plateforme cartographique « Medtrix » (\href{https://medtrix.fr/}{https://medtrix.fr/}). Deux de ces réseaux sont gérés par l’entreprise Andromède Océanologie et concernent les habitats les plus riches et les plus sensibles de Méditerranée : les herbiers de posidonie (TEMPO) et les récifs coralligènes (RECOR).

\subsubsection{La société Andromède Océanologie}\label{intro.2.3.1}

Andromède Océanologie (www.andromede-ocean.com) est une PME créée en 2008, spécialisée dans les relevés écologiques in situ en plongée sous-marine. Son objet est de conduire des projets innovants liés à l'étude et à la valorisation de l'environnement marin. Les activités d’Andromède Océanologie et de son équipe de 13 personnes sont organisées en 3 pôles : 

\begin{itemize}
    \item \textbf{Un pôle bureau d’études} dont les capacités d’expertise ont notamment trait à la bathymétrie, la cartographie des biocénoses marines, l’analyse écologique et la gestion des écosystèmes marins;
    
    \item \textbf{Un pôle valorisation} qui gère notamment la diapothèque (plus de 25 000 clichés) de Laurent Ballesta, plongeur extrême et photographe sous-marin internationalement reconnu, auteur de nombreux livres, documentaires et expéditions;
    
    \item \textbf{Un pôle Recherche et Développement} qui met au point des technologies innovantes pour l’évaluation, le suivi et l’amélioration de l’état de santé des écosystèmes marins côtiers. Andromède porte ainsi plusieurs réseaux de surveillance de l’état écologique des eaux côtières, en partenariat avec l’Agence de l’Eau Rhône Méditerranée Corse (notamment TEMPO pour les herbiers de posidonie et RECOR pour les récifs coralligènes). Par ailleurs, un financement de « laboratoire commun » mutualise les moyens et efforts de recherche entre l’entreprise et l’Université de Montpellier (UMR MARBEC) pour le développement de méthodes d’observations innovantes sur les habitats Méditerranéens (\href{https://labcomintosea.edu.umontpellier.fr}{https://labcomintosea.edu.umontpellier.fr}).
\end{itemize}

La société est localisée en bord de mer à Carnon, à 15 minutes de Montpellier, et dispose d’un parc matériel et technologique très spécialisé consacré aux domaines de l'océanologie et de la plongée Hi-tech : 3 bateaux, sonar latéral, sondeur multifaisceaux, système de positionnement GPS-rtk, nombreux scaphandres de plongée à circuit ouvert et fermé (recycleur), équipement photo et vidéo sous-marines pour films professionnels, etc. Andromède a réalisé la plupart des cartographies des biocénoses marines côtières (0 à -80 m de fond) de Méditerranée française et quelques zones à l’étranger (ilots de Galite et Zembra en Tunisie, réserves de Tavolara et Carbonara en Sardaigne, Italie). Elle est à l’origine de la première cartographie continue des biocénoses (1 : 10 000) en Méditerranée Française (Andromède Océanologie, 2014). Ces cartes sont mises gratuitement à la disposition des professionnels de la mer sur la plateforme cartographique Medtrix (\href{www.medtrix.fr}{www.medtrix.fr}) dans le projet DONIA Expert. Une version simplifiée de ces cartes est disponible via l’application mobile de plaisance Donia : ancrage en toute sécurité et hors des habitats sensibles, cartes 3D des fonds pour repérer les sites de plongée ou de pêche, outil de navigation communautaire… Cette application a reçu le prix Bateau Bleu et le prix Entreprises et Biodiversité de 2013, et a fait l’objet d’une mise à jour importante en 2020.

\subsubsection{TEMPO : un réseau de surveillance des herbiers de posidonie en Méditerranée Française}\label{intro.2.3.2}

Les herbiers sous-marins sont souvent considérés comme des sentinelles de par leur sensibilité aux changements de conditions environnementales, et la moindre modification de leur distribution révèle des changements environnementaux \citep{orth_global_2006}. Plus particulièrement, la posidonie, qui pousse à des profondeurs allant de la surface jusqu’à plus de 40 m de fond en fonction de la clarté de l’eau, est couramment utilisée comme un bio-indicateur de la qualité de l’eau, notamment à travers l’évolution de sa limite inférieure (i.e limite profonde au-delà de laquelle les conditions favorables à la croissance de la posidonie ne sont plus réunies) \citep{boudouresque_regression_2009, ruiz_mediterranean_2009}. En effet, la profondeur de la limite inférieure est principalement déterminée par la clarté de l’eau, et représente donc un indicateur robuste de l’état global de l’ensemble de l’écosystème \citep{borum_european_2004}. Par ailleurs, il est important de prendre en compte des mesures quantitatives de la vitalité de l’herbier à une profondeur identique (-15 m ; profondeur représentative de l’herbier en Méditerranée) car les herbiers peu profonds montrent une grande variabilité naturelle \citep{marba_interannual_1997, balestri_spatial_2003}. Des indicateurs intègrent les différentes mesures de vitalité de l’herbier (densité, longueurs de feuilles, épiphytes…) et des espèces associées (herbivores, échinodermes, filtreurs, prédateurs…), afin d’évaluer le fonctionnement global de l’herbier, à la fois à profondeur intermédiaire et en limite inférieure. 

TEMPO est un réseau de surveillance de l’état écologique des herbiers de posidonie en Méditerranée française qui intègre ces deux composantes importantes de l’herbier : la limite inférieure et la profondeur intermédiaire. Ce réseau est opéré par Andromède océanologie depuis 2011 avec le soutien de l’Agence de l’Eau Rhône-Méditerranée-Corse. La caractérisation de l’état écologique de l’herbier est réalisée par une campagne régionale annuelle sur la période mi-mai / fin juin. Chaque année, une des trois régions concernées par cette agence de l’eau (Corse, Provence-Alpes Côte d’Azur et Occitanie) est échantillonnée, avec un roulement sur trois ans. Toutes régions confondues, le réseau TEMPO permet l’échantillonnage de 96 sites d’herbier lors des trois années de suivi dont 47 sites sont localisés à la profondeur intermédiaire et 53 en en limite inférieure, le plus souvent dans l’alignement des sites à profondeur intermédiaire (voir \autoref{figure_intro16}). 

%%%%%%%%%%%%%%%%%%%%%%%%%%%%%%%%%%%%%%%
%%% Figure intro16: le réseau TEMPO %%%
%%%%%%%%%%%%%%%%%%%%%%%%%%%%%%%%%%%%%%%
\begin{figure}[H]
	\begin{center}
	\includegraphics[width=\linewidth,keepaspectratio]{./1_intro/reseau_TEMPO}
		\caption[Localisation des sites du réseau de surveillance TEMPO]{Localisation des sites du réseau de surveillance TEMPO. En vert clair les 47 sites à -15m ; en vert foncé les 53 sites en limite inférieure.}
	\label{figure_intro16}
\end{center}
\end{figure}

La méthode choisie pour la surveillance de l’herbier de posidonie en limite inférieure prend en compte deux types de mesures : une cartographie de la limite inférieure de l’herbier par télémétrie acoustique, et des mesures de vitalité de l’herbier. La méthode de télémétrie acoustique permet à l’opérateur d’effectuer un point tous les 30 à 50 cm pour cartographier précisément la limite inférieure grâce à l’Aquamètre D100-NG dernière génération (\href{http://www.plsm.eu}{http://www.plsm.eu}) relié à une tablette tactile étanche. La caractérisation de l’état de conservation des herbiers de Posidonia oceanica est réalisée selon les protocoles standardisés du PREI \citep{gobert_assessment_2009}, de l’EBQI \citep{personnic_ecosystem-based_2014} et du BiPo \citep{lopez_y_royo_biotic_2010} basés sur des mesures biologiques in situ et en laboratoire. Par ailleurs, des mesures bioacoustiques sont réalisées sur certains sites afin de quantifier l’activité des espèces mobiles (en partenariat avec l’équipe Chorus : \href{https://chorusacoustics.com/}{https://chorusacoustics.com/}) (voir \autoref{figure_intro17}).

\setlength{\fboxsep}{3pt}
\setlength{\fboxrule}{0.6pt}
\noindent\framebox{%
  \begin{minipage}{\linewidth}
    Le protocole TEMPO permet d’acquérir une \textbf{diversité de mesures de vitalité de l’herbier} à \textbf{profondeur intermédiaire} (-15 m) et en \textbf{limite inférieure}, ainsi que des \textbf{espèces associées} (filtreurs, herbivores…). Si la cartographie de la limite inférieure par \textbf{télémétrie acoustique} permet de suivre finement son \textbf{évolution dans le temps}, la résolution dépend de la volonté du plongeur, et la manipulation peut être longue et devenir \textbf{physiologiquement contraignante} pour le plongeur dans le cas des herbiers les plus \textbf{profonds} (notamment en Corse) et les plus \textbf{fragmentés}.
  \end{minipage}
}

%%%%%%%%%%%%%%%%%%%%%%%%%%%%%%%%%%%%%%%%
%%% Figure intro17: la méthode TEMPO %%%
%%%%%%%%%%%%%%%%%%%%%%%%%%%%%%%%%%%%%%%%
\begin{landscape}
%\newgeometry{left=-3cm,bottom=-10cm}
%\enlargethispage{.5cm}
\begin{figure}[H]
    %\vspace{-\marginparsep}
    %\vspace{-\marginparwidth}
	\begin{center}
	\includegraphics[width=0.84\linewidth,keepaspectratio]{./1_intro/encart_TEMPO}
		\caption[TEMPO : réseau de surveillance des herbiers de posidonie en Méditerranée française]{TEMPO : réseau de surveillance des herbiers de posidonie en Méditerranée française.}
	\label{figure_intro17}
\end{center}
\end{figure}
\end{landscape}
%\restoregeometry

\newpage

\subsubsection{RECOR : un réseau de surveillance des récifs coralligènes}\label{intro.2.3.3}

« Il est urgent de développer des nouvelles méthodes pour comprendre la structuration de ces assemblages [coralligènes], et évaluer les impacts auxquels ils sont soumis, afin de fournir des un état de référence et explorer les possibles trajectoires d’évolution de ces assemblages d’une grande diversité » \citep{kipson_rapid_2011}. RECOR est un réseau de surveillance de l’état écologique des récifs coralligènes en Méditerranée, opéré par Andromède océanologie depuis 2010 avec le soutien de l’Agence de l’eau Rhône-Méditerranée-Corse. Comme pour TEMPO, la caractérisation de l’état écologique des récifs est réalisée par campagne régionale annuelle sur la période mi-mai / fin juin. Chaque année, une des trois régions concernées par cette agence de l’eau (Corse, Provence-Alpes Côte d’Azur et Occitanie) est échantillonnée, avec un roulement sur trois ans. Toutes régions confondues (Corse, Provence-Alpes Côte d’Azur et Occitanie) le réseau RECOR permet l’échantillonnage de 177 stations situées sur 97 sites (i.e. récifs) lors des trois années de suivi (voir \autoref{figure_intro18}). Les stations sont situées à des profondeurs comprises entre 17 et 90 m, et un site peut inclure plusieurs stations à des profondeurs différentes.


%%%%%%%%%%%%%%%%%%%%%%%%%%%%%%%%%%%%%%%
%%% Figure intro18: le réseau RECOR %%%
%%%%%%%%%%%%%%%%%%%%%%%%%%%%%%%%%%%%%%%
\begin{figure}[H]
	\begin{center}
	\includegraphics[width=\linewidth,keepaspectratio]{./1_intro/reseau_RECOR}
		\caption[Localisation des 97 sites du réseau de surveillance RECOR]{Localisation des 97 sites du réseau de surveillance RECOR.}
	\label{figure_intro18}
\end{center}
\end{figure}

Encore plus que pour les habitats peu profonds, le suivi des récifs coralligènes est limité par les contraintes physiologiques du plongeur. En effet, à ces profondeurs (couramment 50 à 80 m de fond), le temps passé sous l’eau est compté, et chaque minute supplémentaire se paye cher en temps de décompression. C’est pourquoi la méthode de suivi la plus utilisée est le quadrat photographique : le plongeur réalise 30 images standardisées (distance au récif et éclairage contrôlé), échantillonnées à différents endroits du récif, qui sont ensuite identifiés par un taxonomiste une fois de retour au bureau (Deter et al., 2012b). Les images sont analysées à l’aide du logiciel Coral Point Count v4.1 « coralligenous assemblages version » \citep{cpce_coral_2011} en échantillonnant aléatoirement 64 points par image (soit 30 x 64 = 1920 points par station), et chaque point est identifié par le même expert taxonomiste. Celui-ci calcule ensuite, à l’échelle de la station, des indicateurs d’état de conservation et de diversité, notamment :

\begin{itemize}
    \item \textbf{Coralligenous Assemblage Index (CAI)} \citep{deter_preliminary_2012} : indicateur représentatif de l’état écologique d’un récif. Il prend en compte trois composantes : la proportion de bio-constructeurs, la proportion de vase et la proportion de bryozoaires. Chacune des trois valeurs est standardisée par la valeur minimale (pour la vase) ou maximale (pour les bioconstructeurs et les bryozoaires) mesurée par région. Il est calculé comme suit :
    
    \begin{equation}
        \text{CAI\textsubscript{i}}=\frac{1}{3}\times(\frac{1-sludge_i}{1-\min_{i}sludge_i}+\frac{majbuilders_i}{\max_{i}majbuilders_i}+\frac{bryozoans_i}{\max_{i}bryozoans_i})
        \label{eqintro.2}
    \end{equation}
    
    \item \textbf{Indice de Shannon} \citep{magurran_measuring_2004} : voir section \ref{intro.2.1} ;
    
    \item \textbf{Nécroses} : pourcentage de mortalité d’algues bio-constructrices (potentiellement lié à la température, pathogènes, pollution, compétition…) ;
    
    \item \textbf{Indice algues filamenteuses} : pourcentage d’algues filamenteuses qui prolifèrent et recouvrent les récifs (potentiellement lié à la température, nutriments et salinité).

\end{itemize}

Ces indicateurs permettent de définir un état de diversité et de conservation à un instant t, et le suivi de leur évolution dans le temps permet de quantifier la potentielle dégradation ou récupération des récifs. Conjointement à ces analyses, les plongeurs réalisent des mesures de tailles, densités et nécroses de gorgones à l’aide de 30 quadrats de 0,5 x 0,5 m (voir \autoref{figure_intro19}). Les gorgones sont des espèces érigées suspensivores qui sont sensibles aux perturbations mécaniques et aux variations de qualité de l’eau et de la température, et sont donc un bon témoin de la qualité écologique d’un récif. Enfin, comme pour le réseau TEMPO, des mesures bioacoustiques sont réalisées sur certains sites afin de quantifier l’activité des espèces mobiles (en partenariat avec l’équipe Chorus : \href{https://chorusacoustics.com/}{https://chorusacoustics.com/}).

\setlength{\fboxsep}{3pt}
\setlength{\fboxrule}{0.6pt}
\noindent\framebox{%
  \begin{minipage}{\linewidth}
    Le protocole RECOR permet d’acquérir en peu de temps une \textbf{grande quantité de données} indispensables à \textbf{l’évaluation de la diversité des assemblages coralligènes} et de leur \textbf{état de santé}. Cependant, l’analyse a posteriori des images pour \textbf{l’identification des espèces du coralligène} requiert des \textbf{compétences taxonomistes} et sont extrêmement \textbf{chronophages} (1920 identifications par station sont nécessaires à l’évaluation de ces assemblages complexes).
  \end{minipage}
}

%%%%%%%%%%%%%%%%%%%%%%%%%%%%%%%%%%%%%%%%
%%% Figure intro19: la méthode RECOR %%%
%%%%%%%%%%%%%%%%%%%%%%%%%%%%%%%%%%%%%%%%
\begin{landscape}
%\enlargethispage{.5cm}
\begin{figure}[H]
	\begin{center}
	\includegraphics[width=\linewidth,keepaspectratio]{./1_intro/encart_RECOR}
		\caption[RECOR : réseau de surveillance des récifs coralligènes en méditerranéen française]{RECOR : réseau de surveillance des récifs coralligènes en méditerranéen française.}
	\label{figure_intro19}
\end{center}
\end{figure}
\end{landscape}

\newpage

%%% PROBLEMATIQUE %%%
\section{Problématique et objectifs de la thèse}\label{intro.3}

La \textbf{biodiversité mondiale} fait actuellement face à d’importantes \textbf{pressions d’origine anthropique}, et nous connaissons un rythme d’extinction d’espèces et de dégradation des habitats sans précédents depuis la dernière grande extinction de masse. Les \textbf{écosystèmes marins}, en particulier, concentrent une grande partie de la biodiversité \textbf{dont nous dépendons} directement ou indirectement, et souffrent du cumul de ces pressions à l’échelle globale. Des \textbf{mesures de gestion efficaces} pour limiter les impacts anthropiques sur le milieu marin sont plus indispensables que jamais, \textbf{notamment en Méditerranée}, une mer semi-fermée qui concentre de hauts niveaux de biodiversité et de pressions cumulées. Malheureusement, les \textbf{difficultés d’accès} au monde sous-marin limitent considérablement l’acquisition de données et donc nos connaissances sur la distribution et l’évolution de sa biodiversité. C’est pourquoi la \textbf{surveillance écologique} de ces habitats sensibles est primordiale : elle doit permettre aux décideurs et gestionnaires d’établir des \textbf{mesures de conservation} et de quantifier leur efficacité afin de limiter au maximum les impacts anthropiques sur ces habitats. C’est l’objet des deux réseaux de surveillance \textbf{TEMPO} et \textbf{RECOR}, centrés sur les \textbf{herbiers de posidonie} et les \textbf{récifs coralligènes}, les deux habitats les plus riches de Méditerranée. S’ils bénéficient d’une dizaine d’années d’expérience, ces deux réseaux sont en constante amélioration, et certaines étapes aujourd’hui fastidieuses ou imprécises pourraient notamment bénéficier de développements récents en matière d’analyses et de traitement d’images.

La \textbf{reconnaissance d’images} a connu une grande révolution avec le développement des premiers \textbf{réseaux de neurones convolutifs} au début des années 2010. Ces algorithmes, dont les performances dépassent parfois même celles d’un opérateur humain, sont de plus en plus utilisés en sciences, notamment en écologie avec la \textbf{reconnaissance d’espèces}. Dans le même temps, un autre type de traitement d’images s’est largement développé grâce à l’amélioration des algorithmes et à l’explosion de la puissance de calcul : la \textbf{photogrammétrie}. Cette technique permet de \textbf{reconstruire en 3D} un objet à partir d’images en deux dimensions réalisées sous différents angles de vue. Comme les réseaux de neurones convolutifs, elle connaît un succès croissant et des applications de plus en plus variées, y compris en écologie car elle permet de capturer la \textbf{structure tridimensionnelle de l’habitat} et de reconstituer des images aériennes de très haute définition.

\textbf{L’objectif de cette thèse} est de répondre aux \textbf{besoins de la surveillance} des habitats marins par le développement de méthodes et d’indicateurs innovants. Plus particulièrement, cette thèse CIFRE (Convention Industrielle de Formation par la REcherche) vise à répondre aux besoins des réseaux de surveillance TEMPO et RECOR, portés par la société Andromède Océanologie, par le \textbf{développement de méthodes opérationnelles} d’évaluation de la santé des habitats basées sur les \textbf{réseaux de neurones convolutifs} et la \textbf{photogrammétrie}. 

En effet, le réseau RECOR se base en partie sur de très nombreuses identifications d’espèces du coralligène, nécessitant le travail fastidieux d’un taxonomiste expert, qui représentent la principale limite à la capacité d’échantillonnage. En bénéficiant de la \textbf{grande base d’images annotées} constituée au cours des années, \textbf{l’entraînement d’un réseau de neurones convolutifs} permettrait d’automatiser l’interprétation des images collectées et d’augmenter de fait le volume de données potentiellement analysables. Par ailleurs, les récifs coralligènes possèdent une \textbf{structure tridimensionnelle} complexe, encore très peu étudiée et qui n’est aujourd’hui pas prise en compte par le réseau RECOR, or elle est le reflet de la longue évolution de ces récifs biogéniques et pourrait entretenir des liens étroits avec la \textbf{composition des assemblages}. La \textbf{photogrammétrie} apparaît comme une technique de choix pour étudier ces liens, car elle permet de reconstruire en trois dimensions les récifs dans toute leur complexité et \textbf{d’analyser leur structure} dans le détail avec des indicateurs architecturaux. Enfin, la cartographie de la limite inférieure des herbiers de posidonie souffre d’un manque de précision et d’un temps d’acquisition physiologiquement contraignant pour le plongeur avec la télémétrie acoustique. La \textbf{photogrammétrie} pourrait offrir des solutions de \textbf{cartographie rapide et automatisée} de cette limite, permettant d’améliorer l’efficacité et la précision des suivis réalisés dans le cadre du réseau TEMPO. 

La partie suivante du manuscrit détaille les aspects méthodologiques concernant les deux méthodes d’analyses d’images employées dans le cadre de ces travaux de recherche (i.e. réseaux de neurones convolutifs et photogrammétrie), afin de fournir au lecteur les bases théoriques à la bonne compréhension du reste du manuscrit. Le travail de recherche à proprement parler se divise en quatre chapitres détaillant successivement le développement et l’application des méthodes opérationnelles basées sur ces deux techniques d’analyses d’images, avec les objectifs scientifiques suivants :

\begin{enumerate}
    \item Développer et entraîner un réseau de neurones convolutifs à reconnaître les espèces du coralligène avec un taux d’erreur semblable ou inférieur à celui d’un expert taxonomiste ; \textbf{Voir chapitre 1}
    
    \item Définir un protocole d’acquisition plongeur pour réaliser des reconstructions 3D d’habitats marins par photogrammétrie, et quantifier la précision et la résolution de ces reconstructions ; \textbf{Voir chapitre 2}
    
    \item Développer une méthode de micro-cartographie automatique des herbiers de posidonie basée sur la photogrammétrie pour permettre un suivi à fine échelle et rapide de la limite inférieure des herbiers ; \textbf{Voir chapitre 3}
    
    \item Caractériser la structure des récifs coralligènes par photogrammétrie et explorer les liens entre structure, biodiversité des assemblages et conditions environnementales ; \textbf{Voir chapitre 4}
    
\end{enumerate}

Enfin, la dernière partie de ce manuscrit est consacrée à synthèse et à la discussion de l’ensemble des résultats de ce travail de thèse. Cette partie détaillera également comment les principales avancées peuvent être appliquées dans le cadre des réseaux de surveillance TEMPO et RECOR, et ouvrira sur les perspectives de travail à l’issue de ces recherches.