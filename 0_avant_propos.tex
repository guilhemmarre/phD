\phantomsection
\addcontentsline{toc}{section}{AVANT-PROPOS}
{\centerline { {\sffamily \Large AVANT-PROPOS}}}

\vspace*{1cm}
\vskip 0.5cm
\noindent

\normalsize
\noindent Cette thèse à été réalisée de septembre 2017 à juillet 2020 entre la société Andromède Océanologie (Carnon), l'UMR TETIS (INRAE, CIRAD, CNRS, AgroParisTech) et l'UMR MARBEC (Université de Montpellier, CNRS, IRD, Ifremer) à Montpellier. Ce travail a été cofinancé par Andromède Océanologie et l'Association Nationale de la Recherche et de la Technologie), et les missions de terrain ont principalement eu lieu dans le cadre des suivis environnementaux TEMPO (herbiers de posidonie) et RECOR (récifs coralligènes) financés par l'Agence de l'Eau Rhône-Méditerranée-Corse.

\medskip

\noindent Cette thèse a été dirigée par Julie Deter (UMR MARBEC) et Sandra Luque (UMRS TETIS), et a bénéficié des conseils avisés de Dino Ienco, Jérôme Pasquet et Rémi Cresson concernant les réseaux de neurones convolutifs, ainsi que des membres du comité de thèse Nicolas Mouquet, Gérard Subsol, François Guilhaumon, Maria Dornelas et Christiane Weber. 

\bigskip
\bigskip

%%% ARTICLES
\centerline{\textbf{\Large Publications}}

\medskip

% ARTICLE DEEP LEARNING
\noindent\href{https://doi.org/10.1016/j.ecoinf.2020.101110}{\textbf{Marre G.}, De Almeida Braga, C., Ienco, D., Luque, S., Holon, F., Deter, J. (2020). Deep convolutional neural networks to monitor coralligenous reefs: operationalizing biodiversity and ecological assessment. Ecological Informatics 59:101110}

\medskip

% ARTICLE METHODO
\noindent{\href{https://doi.org/10.3389/fmars.2019.00276}{\textbf{Marre, G.}, Holon, F., Luque, S., Boissey, P., Deter, J. (2019). Monitoring marine habitats with photogrammetry: a cost-effective, accurate, precise and high-resolution reconstruction method. Frontiers in Marine Science 6:276, 158–170}.}

\medskip

% ARTICLE HERBIERS
\noindent{\href{https://doi.org/10.3354/meps13338}{\textbf{Marre G.}, Deter, J., Holon, F., Boissery, P., Luque, S. (2020). Fine-scale automatic mapping of living \textit{Posidonia oceanica} seagrass beds with underwater photogrammetry. Marine Ecology Progress Series 643:63-74.}}

\bigskip
\bigskip

%%% AUTRES ARTICLES
\centerline{\textbf{\Large Autres publications}}

\noindent{\href{https://doi.org/10.3389/fmars.2019.00276}{Holon, F., \textbf{Marre, G.}, Parravicini, V., Mouquet, N., Bockel, T., Descamp, P., Tribot, A-S., Boissery, P., Deter, J. (2018). A predictive model based on multiple coastal anthropogenic pressures explains the degradation status of a marine ecosystem: Implications for management and conservation. Biological Conservation 222, 125-135.} Voir \autoref{annexe-holon}}

\clearpage
%%% RAPPORTS 
\centerline{\textbf{\Large Rapports d'activité}}

\noindent{\textbf{Marre, G.}, Holon, F., Delaruelle, G., Holon, F., Guilbert, A., Deter, J., 2017. Application de la photogrammétrie à la surveillance biologique: mise au point de la méthode. Rapport d'activité pour l'Agence de l'Eau Rhône-Méditerranée-Corse, 57 pages.}

\noindent{\textbf{Marre, G.}, Holon, F., Delaruelle, Fery, C., G., Holon, F., Guilbert, A., Deter, J., 2019. Acquisitions photogrammétriques 2018 - 2019 et développements méthodologiques. Rapport d'activité pour l'Agence de l'Eau Rhône-Méditerranée-Corse, 145 pages.}

\bigskip
\bigskip

%%% VULGARISATION
\centerline{\textbf{\Large Articles de vulgarisation scientifique}}

\noindent{\href{https://medtrix.fr/cahier-de-surveillance-3/}{\textbf{Marre, G.}, Luque, S., Holon, F., Boissery, P., Deter, J., 2018. Application de la photogrammétrie à la surveillance biologique des habitats sous-marins. Cahiers de la surveillance Medtrix n°3, Mars 2018.} Voir \autoref{annexe-cahiers}}

\medskip

\noindent{\href{https://www.agropolis.fr/publications/sciences-marines-et-littorales-en-occitanie-dossier-thematique-agropolis-international.php}{\textbf{Marre, G.}, Luque, S., Holon, F., Boissery, P., Deter, J., 2019. La photogrammétrie : une méthode d’observation innovante pour l’étude et la conservation du milieu marin. Les dossiers d'Agropolis International N°24: Sciences marines et littorales en Occitanie, Février 2019.}}

\bigskip
\bigskip

%%% PRESENTATIONS ORALES
\centerline{\textbf{\Large Communications dans des congrès}}

\noindent{\textbf{Marre, G.}, Ropars, B., 2017. At the crossroads between robotics, photogrammetry and ecology for the study of marine habitats. Communication orale au congrès Ecolotech. Salon de l'Ecologie, Montpellier, 9 novembre 2017.}

\medskip

\noindent{\textbf{Marre, G.}, Luque, S., Holon, F., Boissery, P., Deter, J., 2018. Rencontre entre robotique, photogrammétrie et écologie pour l'étude et le suivi des fonds marins. Communication orale au congrès Medtrix. Montpellier, 14 mars 2018. Voir \autoref{annexe-medtrix}}

\medskip

\noindent{\href{https://oreme.org/app/uploads/Oreme_AT16_G.Marre_.pdf}{\textbf{Marre, G.}, Luque, S., Holon, F., Boissery, P., Deter, J., 2018. Développement de la photogrammétrie pour 
l’étude et le suivi d’habitats marins. Communication orale à l'Apéro Technique de l'Observatoire des Sciences de l'Univers OREME. Montpellier, 28 mai 2018.}}

\medskip

\noindent{\textbf{Marre, G.}, Luque, S., Holon, F., Boissery, P., Deter, J., 2019. Méthodes photographiques pour la caractérisation de la structure et de la biodiversité des récifs coralligènes. Communication orale aux 10 ans de l'Observatoire des Sciences de l'Univers OREME. Montpellier, 11 octobre 2019.}

\medskip

\noindent{\textbf{Marre, G.}, Luque, S., Holon, F., Boissery, P., Deter, J., 2019. Suivi de communautés coralligènes par des
réseaux de neurones convolutifs. Communication orale au congrès Ecolotech. Salon de l'Ecologie, Montpellier, 7 novembre 2019.}

\medskip

\noindent{\textbf{Marre, G.}, Luque, S., Holon, F., Boissery, P., Deter, J., 2019. Photogrammétrie sous-marine et analyses d’images pour le suivi d’habitats benthiques Méditerranéens. Communication orale acceptée au congrès Merigeo à Bordeaux en mars 2020. Reporté à cause du Covid-19 à Nantes, novembre 2020. Voir \autoref{annexe-merigeo}}

\bigskip
\bigskip

%%% ENCADREMENT
\centerline{\textbf{\Large Activités d'encadrement}}

\noindent{Co-encadrement du stage de fin d'études (M2 - cursus ingénieur) de \textbf{Gaïlé Lejay} en 2018, ayant donné lieu à la rédaction d'un mémoire de fin d'études : "Utilisation de modèles 3D en écologie sous-marine, détermination d’indicateurs dérivés des modèles et analyse de la variation des paramètres selon l’état de conservation des habitats".}

\medskip

\noindent{Co-encadrement du stage de fin d'études (M2 - Computer Science) de \textbf{Cédric De Almeida Braga} en 2019, ayant donné lieu à la rédaction d'un mémoire de fin d'études : "Characterization of coralligenous assemblages : from automatic image classification to 3D species mapping".}


\bigskip
\bigskip

%%% FORMATIONS
\centerline{\textbf{\Large Formations suivies}}

\noindent{\textbf{Plongée recycleur Inspiration} - Carnon - Novembre 2017 (5 jours)}

\medskip

\noindent{\textbf{Python 3 : des fondamentaux aux concepts avancés du langage} - MOOC - Janvier 2018 (25 heures)}

\medskip

\noindent{\textbf{Linux pour les sciences} - Montpellier - Septembre 2018 (6 heures)}

\medskip

\noindent{\textbf{Data Sciences for Geosciences} - Brest - Janvier 2019 (5 jours)}

\medskip

\noindent{\textbf{Ethique de la recherche} - MOOC - Novembre 2019 (15 heures)}

\medskip

\noindent{\textbf{Plongée recycleur trimix hypoxique} - La Ciotat - Décembre 2019 (5 jours)}

\medskip




\bigskip
\bigskip

\newpage
%%% CAMPAGNES DE TERRAIN
\centerline{\textbf{\Large Campagnes de terrain}}

\noindent{\textbf{Réseaux de surveillance TEMPO / RECOR} - Corse - Juin 2017 (14 jours)}

\medskip

\noindent{\textbf{Suivi de transplantation de l'herbier du Larvotto} - Monaco - Juin - Juillet 2017 (10 jours)}

\medskip

\noindent{\textbf{Suivi du récif coralligène des Spélugues} - Monaco - Aout 2017 (4 jours)}

\medskip

\noindent{\textbf{Expériences premier article} - Roquebrune - Novembre 2017 (2 jours)}

\medskip

\noindent{\textbf{Suivi du récif coralligène des Spélugues} - Monaco - Janvier 2018 (5 jours)}

\medskip

\noindent{\textbf{Suivi de transplantation de l'herbier du Larvotto} - Monaco - Février 2018 (4 jours)}

\medskip

\noindent{\textbf{Suivi du récif coralligène des Spélugues} - Monaco - Avril 2018 (4 jours)}

\medskip

\noindent{\textbf{Réseaux de surveillance TEMPO / RECOR} - Occitanie - Juin 2018 (8 jours)}

\medskip

\noindent{\textbf{Restauration d'un récif coralligène (RESCOR)} - Saint-Jean-Cap-Ferrat - Octobre 2018 (3 jours)}

\medskip

\noindent{\textbf{Suivis des récifs artificiels de Cortiou (REXCOR)} - Marseille - Octobre 2018 (2 jours)}

\medskip

\noindent{\textbf{Cartographie en plongée tractée (CartoTract)} - Golfe Juan / Cannes - Septembre 2018 (2 jours)}

\medskip

\noindent{\textbf{Mise au point de l'acquisition acoustique et photogrammétrie (GOMBESSA)} - Villefranche-sur-Mer - Mars 2019 (2 jours)}

\medskip

\noindent{\textbf{Restauration d'un récif coralligène (RESCOR)} - Saint-Jean-Cap-Ferrat - Avril 2019 (5 jours)}

\medskip

\noindent{\textbf{Réalisation d'images complémentaires pour le tournage de GOMBESSA V} - Capraia, Italie - Avril 2019 (5 jours)}

\medskip

\noindent{\textbf{Suivis des rejets de station d'épuration} - Agglomération de Marseille - juin 2019 (3 jours)}

\medskip

\noindent{\textbf{Réseaux de surveillance TEMPO / RECOR} - PACA - Juin 2019 (4 jours)}

\medskip

\noindent{\textbf{Suivis des récifs artificiels de Cortiou (REXCOR)} - Marseille - Juin 2019 (2 jours)}

\medskip

\noindent{\textbf{Suivis des récifs artificiels de Cortiou (REXCOR)} - Marseille - Septembre 2019 (2 jours)}

\medskip

\noindent{\textbf{Réseaux de surveillance TEMPO / RECOR} - Occitanie - Mai 2020 (3 jours)}

\medskip

\noindent{\textbf{Réseaux de surveillance TEMPO / RECOR} - Corse - Mai 2020 (24 jours)}

\medskip
