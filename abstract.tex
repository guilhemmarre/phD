%\begin{center}
%\large\textbf{ABSTRACT}
%\end{center}
%\vspace*{1cm}
%\chapter*{\centering \underline{\large\textbf{ABSTRACT}}}
%\addcontentsline{toc}{chapter}{\bf ~~~~~~~~~~ABSTRACT}
\phantomsection
%\addcontentsline{toc}{section}{RÉSUMÉ}
\small
\newgeometry{left=2cm,right=2cm, top=1.7cm, bottom=2cm}
{\centerline {{\sffamily \Large RÉSUMÉ}}}

\noindent\textbf{Caractérisation de l’hétérogénéité spatiale de milieux naturels à partir d’imagerie optique très haute résolution spatiale : cas d’application aux milieux méditerranéens de garrigue.}

\noindent La préservation de la biodiversité est un enjeu prioritaire, identifié aussi bien au niveau national qu’au niveau européen et international. L’hétérogénéité spatiale des milieux naturels est l’une des composantes clefs pour l’étude de la biodiversité et permet de comprendre le fonctionnement des écosystèmes. Le bassin Méditerranéen est un \emph{hotspot} de biodiversité pour lequel le lien entre biodiversité et hétérogénéité spatiale des paysages s’illustre particulièrement bien. Les milieux méditerranéens s'organisent en mosaïques hétérogènes de quatre strates verticales~: le sol nu, l’herbe, les ligneux bas et les ligneux hauts. La biodiversité de ces milieux est aujourd'hui menacée par une fermeture de milieux naturels qui entraîne la disparition de certains habitats et l’homogénéisation des paysages, homogénéisation qui entraîne elle-même une augmentation des risques d’incendies.

\noindent Cette thèse se propose de développer des indices caractérisant l’hétérogénéité spatiale des milieux naturels dans un contexte méditerranéen à partir d’images de télédétection à très haute résolution. Parmi les différentes méthodes permettant de caractériser l'hétérogénéit

\noindent\textbf{Mots clefs~:}{ \it Hétérogénéité spatiale, Télédétection, Texture, Très haute résolution spatiale, Biodiversité, Paysages méditerranéens, Conservation de l’avifaune}.
\vskip 0.5cm
\noindent

{\centerline {{\sffamily \Large ABSTRACT}}}

\noindent\textbf{Characterization of the spatial heterogeneity of natural environments from very high spatial resolution optical imagery : an application case to garrigue Mediterranean habitats.}

\noindent The preservation of biodiversity is a priority issue, both at national, European and international levels. In order to provide a better understanding of ecosystem functioning, spatial heterogeneity of natural environments is becoming one of the key components for the study of biodiversity. The Mediterranean basin is a \emph{hotspot} of biodiversity for which the synergies between biodiversity and spatial heterogeneity of landscapes are particularly important. Mediterranean environments are organized into heterogeneous mosaics of four vertical strata~: bare soil, herbs, low ligneous and high ligneous. The biodiversity of these unique hotspots is now threatened by a closure of the landscape that leads to the habitat loss and landscape homogenization. The loss of heterogeneity is also leading to an increase in fire risks.

\noindent This thesis aims to develop indices characterizing the spatial heterogeneity of natural landscapes in a Mediterranean context using very high spatial resolution remote sensing images. Among the various methods dedicated to the characterization of heterogeneity, the FOTO (FOurier Based Textural Ordination) method is particularly relevant because it produces uncorrelated texture gradients in an unsupervised manner, allowing continuous variations in spatial heterogeneity to be characterized at different spatial scales. Thus, the first objective of this thesis is to test the potential of texture indices derived from the FOTO method for the characterization of spatial heterogeneity relative to four vertical strata. The second objective is to test the sensitivity of our approach to technical and environmental 

\noindent\textbf{Key words~:}{ \it Spatial heterogeneity, Remote sensing, Texture, Very high spatial resolution, biodiversity, Mediterranean landscape, Birds conservation}.
%\normalsize
%\restoregeometry
%\vskip 0.5cm
%\noindent
