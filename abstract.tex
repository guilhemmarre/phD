%\begin{center}
%\large\textbf{ABSTRACT}
%\end{center}
%\vspace*{1cm}
%\chapter*{\centering \underline{\large\textbf{ABSTRACT}}}
%\addcontentsline{toc}{chapter}{\bf ~~~~~~~~~~ABSTRACT}
\phantomsection
\addcontentsline{toc}{section}{RÉSUMÉ}
\small
\newgeometry{left=2cm,right=2cm, top=1.7cm, bottom=2cm}
{\centerline {{\sffamily \Large RÉSUMÉ}}}

\textbf{Caractérisation de l’hétérogénéité spatiale de milieux naturels à partir d’imagerie optique très haute résolution spatiale : cas d’application aux milieux méditerranéens de garrigue.}

La préservation de la biodiversité est un enjeu prioritaire, identifié aussi bien au niveau national qu’au niveau européen et international. L’hétérogénéité spatiale des milieux naturels est l’une des composantes clefs pour l’étude de la biodiversité et permet de comprendre le fonctionnement des écosystèmes. Le bassin Méditerranéen est un \emph{hotspot} de biodiversité pour lequel le lien entre biodiversité et hétérogénéité spatiale des paysages s’illustre particulièrement bien. Les milieux méditerranéens s'organisent en mosaïques hétérogènes de quatre strates verticales~: le sol nu, l’herbe, les ligneux bas et les ligneux hauts. La biodiversité de ces milieux est aujourd'hui menacée par une fermeture de milieux naturels qui entraîne la disparition de certains habitats et l’homogénéisation des paysages, homogénéisation qui entraîne elle-même une augmentation des risques d’incendies.

Cette thèse se propose de développer des indices caractérisant l’hétérogénéité spatiale des milieux naturels dans un contexte méditerranéen à partir d’images de télédétection à très haute résolution. Parmi les différentes méthodes permettant de caractériser l'hétérogénéité, la méthode FOTO (FOurier Based Textural Ordination) est particulièrement intéressante car elle produit de façon non supervisée un nombre limité de gradients de texture non corrélés, à partir desquels il est possible de décrire les variations continues de l’hétérogénéité spatiale, et ce, à plusieurs échelles spatiales. Ainsi le premier objectif de cette thèse est de tester le potentiel des gradients de texture issus de la méthode FOTO pour la caractérisation de l’hétérogénéité spatiale relative aux quatre strates verticales caractéristiques des milieux méditerranéens. Le deuxième objectif est de tester la sensibilité de l’approche développée à des facteurs techniques et environnementaux, afin de s’assurer de sa réplicabilité pour favoriser son utilisation dans un contexte opérationnel de suivi des milieux méditerranéens. Enfin, le dernier objectif est de valider la pertinence écologique des indices d’hétérogénéité développés à travers un cas d’application~: la caractérisation de la répartition spatiale d’espèces d’oiseaux sensibles à l’hétérogénéité de la végétation.

Combinés avec un indice de végétation, le NDVI, les indices de texture issus de la méthode FOTO ont pu être interprétés en termes d'hétérogénéité spatiale et ont permis de caractériser la composition et l'organisation des quatre strates verticales étudiées. Ces indices sont influencés par la présence de surfaces anthropisées comme les cultures ainsi que par la nature de l'information radiométrique des images de télédétection utilisées, qui impacte le contraste apparent des strates de végétation. Ainsi, l'application de la méthode sur une bande panchromatique est plus sensible aux motifs liés à l'alternance de sol nu et d'herbe tandis que l'application de la méthode sur le NDVI est plus sensible aux motifs lié à l'alternance des ligneux avec la strate herbacée. Enfin, nous avons montré l’intérêt de l’approche développée pour la prédiction de plusieurs espèces d’oiseaux à fort enjeux de conservation. Les indices d’hétérogénéité ont permis de mettre en évidence des structures de la végétation particulièrement favorables à certaines espèces d’oiseaux.

L'approche développée dans cette thèse est particulièrement intéressante car elle permet la production non supervisée de trois indices complémentaires caractérisant plusieurs composantes de l'hétérogénéité spatiale relatives à quatre strates. Des efforts sont encore nécessaires pour améliorer i) notre compréhension de la contribution de facteurs environnementaux et instrumentaux sur la stabilité de l'approche et ii) son automatisation en vue d'une application dans un contexte opérationnel pour la cartographie et le suivi de l'état de conservation des habitats naturels et de l'avifaune.

\textbf{Mots clefs~:}{ \it Hétérogénéité spatiale, Télédétection, Texture, Très haute résolution spatiale, Biodiversité, Paysages méditerranéens, Conservation de l’avifaune}.
\vskip 0.5cm
\noindent

\textbf{Characterization of the spatial heterogeneity of natural environments from very high spatial resolution optical imagery : an application case to garrigue Mediterranean habitats.}

The preservation of biodiversity is a priority issue, both at national, European and international levels. In order to provide a better understanding of ecosystem functioning, spatial heterogeneity of natural environments is becoming one of the key components for the study of biodiversity. The Mediterranean basin is a \emph{hotspot} of biodiversity for which the synergies between biodiversity and spatial heterogeneity of landscapes are particularly important. Mediterranean environments are organized into heterogeneous mosaics of four vertical strata~: bare soil, herbs, low ligneous and high ligneous. The biodiversity of these unique hotspots is now threatened by a closure of the landscape that leads to the habitat loss and landscape homogenization. The loss of heterogeneity is also leading to an increase in fire risks.

This thesis aims to develop indices characterizing the spatial heterogeneity of natural landscapes in a Mediterranean context using very high spatial resolution remote sensing images. Among the various methods dedicated to the characterization of heterogeneity, the FOTO (FOurier Based Textural Ordination) method is particularly relevant because it produces uncorrelated texture gradients in an unsupervised manner, allowing continuous variations in spatial heterogeneity to be characterized at different spatial scales. Thus, the first objective of this thesis is to test the potential of texture indices derived from the FOTO method for the characterization of spatial heterogeneity relative to four vertical strata. The second objective is to test the sensitivity of our approach to technical and environmental factors, in order to ensure its replicability, and promote its use in an operational context of monitoring Mediterranean environments. Finally, based on a case study centered on the spatial distribution of bird species sensitive to vegetation heterogeneity, the last objective is oriented towards the validation of the ecological relevance of the heterogeneity indices.

Combined with a vegetation index, NDVI, the texture indices derived from the FOTO method could be interpreted in terms of spatial heterogeneity and make it possible to characterize the composition and organization of the four vertical strata studied. These indices are influenced bymultiple factors, including the anthropization of landscapes showing patterns translated into surfaces such as crops, and the nature of the radiometric information of the remote sensing images processed. This pattern information impacts the apparent contrast of vegetation strata. Thus, the application of the method on a panchromatic band is more sensitive to patterns related to the alternation of bare soil and herbs while the application of the method on NDVI is more sensitive to patterns related to the alternation of ligneous strata with the herbaceous stratum. Finally, we have shown the interest of the approach developed for the prediction of several bird species with high conservation stakes. Heterogeneity indices have made it possible to highlight vegetation structures that are particularly favourable to certain bird species.

The approach developed in this thesis is particularly stimulating because it allows the unsupervised production of three complementary indices characterizing several components of spatial heterogeneity related to four strata. Further efforts are needed to improve i) our understanding of the contribution of environmental and instrumental factors to the stability of the approach and ii) its automation for application in an operational context for mapping and monitoring the conservation status of natural habitats and birdlife.

\textbf{Key words~:}{ \it Spatial heterogeneity, Remote sensing, Texture, Very high spatial resolution, biodiversity, Mediterranean landscape, Birds conservation}.
\normalsize
\restoregeometry
\vskip 0.5cm
\noindent
