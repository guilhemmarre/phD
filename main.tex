% Définition du type de document
\documentclass[10pt,twoside,openright,a4paper]{book} % This style for A4 format. 

% Chargement des packages depuis le fichier packages.tex
\usepackage{amsmath, amssymb, amsthm}
\usepackage{graphicx,color}
\usepackage{multirow}
%\usepackage[top=3.3cm, bottom=3.5cm, left=3.3cm, right=3.3cm]{geometry}
\usepackage{epsfig}
\usepackage{color, colortbl}
\usepackage{setspace}
\usepackage{latexsym}
\usepackage{mathrsfs}
\usepackage{graphicx}
\usepackage{tikz}
\usepackage[titletoc]{appendix}
\usepackage{alltt}
\graphicspath{ {images/} }
\usepackage{longtable}
\usepackage{tikz}
%\usepackage{acronym}
\usepackage{pgfplots}
\usepackage[hidelinks]{hyperref}
%\usepackage{tocloft}
%\setcounter{tocdepth}{1}
%\renewcommand{\cftdot}{}
\usepackage{titletoc,tocloft}
%\setlength{\cftsecindent}{2cm}
\setlength{\cftsubsecindent}{1.2cm}
\setlength{\cftsubsubsecindent}{1cm}
%\dottedcontents{section}[1.5em]{}{1.3em}{.6em}

\usepackage[sc]{mathpazo}
\usepackage{sectsty}
\usepackage{titlesec}
\usepackage{textcase}



\usepackage{natbib}
\usepackage[english,frenchb]{babel}
\usepackage[latin1,utf8]{inputenc}
\usepackage[T1]{fontenc}
\usepackage{csquotes}

\usepackage{caption}
\captionsetup[table]{labelsep=space,labelfont=bf}
\captionsetup[figure]{labelsep=space,labelfont=bf}
\setcounter{page}{1}
\newtheorem{definition}{Definition}
\newcommand{\brdefinition}{\begin{definition}}
	\newcommand{\erdefinition}{\end{definition}}

\newtheorem{corollary}{Corollary}
\newcommand{\bcorollary}{\begin{corollary}}
	\newcommand{\ecorollary}{\end{corollary}}
\newtheorem{example}{Example}
\newcommand{\bexample}{\begin{example}}
	\newcommand{\eexample}{\end{example}}
\newtheorem{remark}{Remark}
\newcommand{\bremark}{\begin{remark}}
	\newcommand{\eremark}{\end{remark}}
\newcommand{\bproof}{{\bf {Proof:}}}
\newcommand{\eproof}{}
\newcommand{\bsolution}{{\bf {Solution:}}}
\newcommand{\esolution}{}
\newtheorem{theorem}{Theorem}
\newcommand{\btheorem}{\begin{theorem}}
	\newcommand{\etheorem}{\end{theorem}}
\newtheorem{lemma}{Lemma}
\newcommand{\blemma}{\begin{lemma}}
	\newcommand{\elemma}{\end{lemma}}
\newcommand{\UD}{\ensuremath{\bigtriangleup}}
\newcommand{\IC}{\ensuremath{\mathcal{C}} }
%\renewcommand{\rm}{\normalshape}
\newcommand{\ol}{\overline}
\def\cen{\centerline}
\def\pnq{\par\noindent\quad}
\def\pn{\par\noindent}
\newcommand{\non}{\nonumber}
\newcommand{\mC}{{\mathbb C}}
\newcommand{\mN}{{\mathbb N}}
\newcommand{\mU}{{\mathbb U}}
\newcommand{\cA}{{\mathcal A}}
\newcommand{\cR}{{\mathcal R}}
\newcommand{\cS}{{\mathcal S}}
\newcommand{\cT}{{\mathcal T}}
\newcommand{\cUCV}{{\mathcal UCV}}
\newcommand{\cST}{{\mathcal ST}}
\newcommand{\cK}{{\mathcal K}}
\newcommand{\ds}{\displaystyle}
\newcommand{\brdef}{\begin{defi}}
	\newcommand{\erdef}{\end{defi}}
\newcommand{\bcor}{\begin{cor}}
	\newcommand{\ecor}{\end{cor}}
\newcommand{\bth}{\begin{thm}}
	\newcommand{\ble}{\begin{lem}}
		\newcommand{\ele}{\end{lem}}
	\newcommand{\bcha}{\end{cha}}\pagestyle{plain}
\renewcommand{\theequation}{\thechapter.\arabic{equation}}
\renewcommand{\thetheorem}{\thesection.\arabic{theorem}}
\renewcommand{\thecorollary}{\thesection.\arabic{corollary}}
\renewcommand{\thelemma}{\thesection.\arabic{lemma}}
\renewcommand{\thedefinition}{\thesection.\arabic{definition}}
\renewcommand{\theexample}{\thesection.\arabic{example}}
\renewcommand{\theremark}{\thesection.\arabic{remark}}
\renewcommand{\thechapter}{\arabic{chapter}}

\def\cen{\centerline}
\def\pnq{\par\noindent\quad}
\def\pn{\par\noindent}
\def\sevenpoint{%
	\def\rm{\sevenrm}%
	\def\it{\sevenit}%
	\def\bf{\sevenbf}%
	\rm}
%\pretoler
\renewcommand{\theequation}{\thechapter.\arabic{equation}}
\theoremstyle{definition}
\renewcommand*{\proofname}{{\rm Proof}}
%%%%%%%%%%%%%%%%%%%%%%%%%%%%%%%%%%%%%%%%%%%%%%%%%%%%%%%%%%%%%%%%%%%Chapter title format
\usepackage{sectsty}
\usepackage{titlesec}
\chapterfont{\centering }
\titleformat{\chapter}[display]
{\bf\centering}
{\chaptertitlename\ \thechapter}{16pt}{\large}
\sectionfont{\normalfont}
%\renewcommand*\contentsname{\large \centerline{TABLE OF CONTENTS}}
\renewcommand\cftchappresnum{} % prefix "Chapter " to chapter number in ToC
\cftsetindents{chapter}{0em}{3em}      % set amount of indenting
\cftsetindents{section}{0em}{3em}

\titleformat{\subsubsection}
{\normalfont\normalsize\bfseries}{\thesubsubsection}{1em}{}

% For Terms and Abbreviations 
\usepackage[acronym,section]{glossaries}
\makenoidxglossaries
%\renewcommand*\glspostdescription{\cftdotfill{\cftsecdotsep}}
%\renewcommand{\glossarysection}[2][]{{\centering\bfseries\MakeTextUppercase{#2}\par}}

%\renewcommand*{\bibpagespunct}{\addcolon\space}

\definecolor{butter1}{rgb}{0.988,0.914,0.310}
\definecolor{chocolate1}{rgb}{0.914,0.725,0.431}
\definecolor{chameleon1}{rgb}{0.541,0.886,0.204}
\definecolor{skyblue1}{rgb}{0.447,0.624,0.812}
\definecolor{plum1}{rgb}{0.678,0.498,0.659}
\definecolor{scarletred1}{rgb}{0.937,0.161,0.161}


%For Table Column Width
\usepackage{array}
\newcolumntype{L}[1]{>{\raggedright\let\newline\\\arraybackslash\hspace{0pt}}m{#1}}
\newcolumntype{C}[1]{>{\centering\let\newline\\\arraybackslash\hspace{0pt}}m{#1}}
\newcolumntype{R}[1]{>{\raggedleft\let\newline\\\arraybackslash\hspace{0pt}}m{#1}}

% sur le net
\renewcommand{\arraystretch}{1.2} 
\usepackage{makecell}%To keep spacing of text in tables
\setcellgapes{4pt}%parameter for the spacing

\usepackage{etoolbox}

\makeatletter
\patchcmd{\ttlh@hang}{\parindent\z@}{\parindent\z@\leavevmode}{}{}
\patchcmd{\ttlh@hang}{\noindent}{}{}{}
\makeatother

\usepackage{a4wide}
\usepackage{tabularx}
\usepackage{chngpage}
\usepackage{siunitx}
\usepackage{subscript}
\usepackage{rotfloat}
\usepackage{rotating}
\newcommand*{\innerTabular}[1]{{\begin{tabular}[c]{@{}c@{}}#1\end{tabular}}}
\usepackage{pdfpages}
\usepackage{booktabs}
\usepackage{calc}
\usepackage{fancybox}
\usepackage{enumitem}
\usepackage{eurosym}
\usepackage[skins]{tcolorbox}
\usepackage{lscape}
%\usepackage{lineno}
%\usepackage[nottoc,numbib]{tocbibind} %mettre la biblio dans la toc
%\newcommand{\nocontentsline}[3]{}
%\newcommand{\tocless}[2]{\bgroup\let\addcontentsline=\nocontentsline#1{#2}\egroup}

\newcommand{\myparagraph}[1]{\paragraph{#1}\mbox{}\\} %titre de paragraphe

\newcolumntype{Y}{>{\centering\arraybackslash}X} %modif de tabularx pour centrer

\definecolor{Gray}{gray}{0.9} %couleur dans les tableaux

\renewcommand{\thesection}{\arabic{section}} %numeros romains

\usepackage[Sonny]{fncychap} % joli chapitre

\usepackage{fancyhdr}
\fancyhead{}

\renewcommand{\labelitemi}{$\bullet$}

\setcounter{secnumdepth}{3}

%from nico with love
\newenvironment{changemargin}[3]{\begin{list}{}{%
\setlength{\topsep}{0pt}%
\setlength{\leftmargin}{0pt}%
\setlength{\rightmargin}{0pt}%
\setlength{\topmargin}{0pt}%
\setlength{\listparindent}{\parindent}%
\setlength{\itemindent}{\parindent}%
\setlength{\parsep}{0pt plus 1pt}%
\addtolength{\leftmargin}{#1}%
\addtolength{\rightmargin}{#2}%
\addtolength{\topmargin}{#3}%
}\item }{\end{list}}

%\include{./prem/newcommands}
\pgfplotsset{compat=1.14}
\DeclareCaptionType{Encadre}
% On défini le style: long et court
\setacronymstyle{long-sc-short}


%%% Mathématiques
\newacronym{pca}{PCA}{principal components analysis}

\newacronym{hac}{HAC}{hierarchical ascendant classification}

\newacronym{rmse}{RMSE}{Root Mean Square Error}

\newacronym{lmm}{LMM}{Linear Mixed effect Model}

\newacronym{sd}{SD}{Standard Deviation}

\newacronym{anova}{ANOVA}{analysis of variance}

\newacronym{aic}{AIC}{Akaike information criterion}



%%% Photogrammétrie
\newacronym{gcp}{GCP}{Ground Control Point}

\newacronym{gsd}{GSD}{Ground Sample Distance}

\newacronym{gcp_rmse}{GCP RMSE}{Ground Control Point Root Mean Square Error}

\newacronym{tpt}{TPT}{Total Processing Time}

\newacronym{c2m}{C2M}{Cloud-to-mesh distance}



%%% Andromède
\newacronym{tempo}{TEMPO}{Réseau de surveillance des herbiers de Posidonie}

\newacronym{recor}{RECOR}{Réseau de surveillance des récifs coralligènes}



%%% Institutions
\newacronym{inrae}{INRAE}{Institut National de Recherche pour l'Agriculture, l'alimentation et l'Environnement}

\newacronym{ird}{IRD}{Institut de Recherche pour le Développement}

\newacronym{cnrs}{CNRS}{Centre National de la Recherche Scientifique}

\newacronym{cirad}{CIRAD}{Centre de coopération Internationale en Recherche Agronomique pour le Développement}

\newacronym{umr}{UMR}{Unité Mixte de Recherche}

\newacronym{tetis}{TETIS}{Territoire, Environnement, Télédétection et Information Spatiale}

\newacronym{marbec}{MARBEC}{MARine Biodiversity, Exploitation and Conservation}

\newacronym{anrt}{ANRT}{Agence Nationale pour la Recherche Technologique}

\newacronym{aermc}{AERMC}{Agence de l'Eau Rhône-Méditerranée-Corse}

\newacronym{osu}{OSU}{Observatoire des Sciences de l'Univers}

\newacronym{oreme}{OREME}{Observatoire de REcherche Méditerranéen de l'Environnement}

%%% Début du document - page de garde
\begin{document}
%\onehalfspacing
%\begin{center}
%\thispagestyle{empty}
\pagenumbering{gobble}
\includepdf[pages={1}]{./coverpage/couverture_these_UM.pdf}

%%{\textit{Synopsis of the Thesis}}
%\vskip .7cm
%~\\~\\~\\~
%{\Large \textbf{Thesis Title}}
%% {\Large \textbf {An Overview}}
%% \vskip .2cm
%% {\Large \textbf {of}}
%% \vskip .2cm
%% {\Large \textbf {Everything}}
%\vskip 4 cm
%{\it Submitted in partial fulfillment of the requirements for the degree of}
%\vskip 0.5cm
%{\bfseries \huge Doctor of Philosophy}
%\vskip 2 cm
%~\\~\\~
%\textit{by} \\
%\vskip .5cm
%\textbf{\Large YOUR NAME}
%\vskip 6.5 cm
%\centerline{\includegraphics[height=31mm,width=69mm]{images/university}}
%\vskip 0.1cm
%\large  SCHOOL OF COMPUTING SCIENCE AND ENGINEERING \\
%\vspace{-.3cm} VIT UNIVERSITY \\
%\vspace{-.3cm} VELLORE - 632 014\\
%\textbf{Month, Year}}
%\end{center}
\renewcommand{\labelitemi}{$\bullet$}

\onehalfspacing

% avant-propos
\cleardoublepage
\pagenumbering{roman} % Numérotation des pages en chiffres romain
\phantomsection
\addcontentsline{toc}{section}{AVANT-PROPOS}
{\centerline { {\sffamily \Large AVANT-PROPOS}}}

\vspace*{1cm}
\vskip 0.5cm
\noindent

\normalsize
\noindent Cette thèse à été réalisée de septembre 2017 à juillet 2020 entre Andromède Océanologie et l'\acrshort{umr} \acrshort{tetis} (\acrshort{inrae}, \acrshort{cirad}, \acrshort{cnrs}, AgroParisTech) à Montpellier, en collaboration avec l'\acrshort{umr} \acrshort{marbec} (Université de Montpellier, \acrshort{cnrs}, \acrshort{ird}, Ifremer, Montpellier). Ce travail a été cofinancé par Andromède Océanologie et l'\gls{anrt}, et les missions de terrain ont eu lieu dans le cadre des suivis environnementaux \acrshort{tempo} et \acrshort{recor} financés par l'\gls{aermc}.

\medskip

\noindent Cette thèse a été dirigée par Julie Deter et Sandra Luque, et a bénéficié des conseils avisés de Dino Ienco sur les aspects deep learning ainsi que des membres du comité de thèse Nicolas Mouquet, Gérard Subsol, François Guilhaumon, Maria Dornelas et Christiane Weber. 

\bigskip

%%% ARTICLES
\noindent\textbf{Publications}

% 1er article méthodo
\noindent{\href{https://doi.org/10.3389/fmars.2019.00276}{\textbf{Marre, G.}, Holon, F., Luque, S., Boissey, P., Deter, J., 2019. Monitoring marine habitats with photogrammetry: a cost-effective, accurate, precise and high-resolution reconstruction method. Frontiers in Marine Science 6:276, 158–170}.}

\medskip

% 2ème article herbiers
\noindent{\textbf{Marre G.}, Deter, J., Holon, F., Boissery, P., Luque, S. Fine-scale automatic mapping of living Posidonia oceanica seagrass beds with underwater photogrammetry. Marine Ecology Progress Series - \textit{Accepté sous conditions de modifications mineures.}}

\medskip

% 3ème article 
\noindent{\textbf{Marre G.}, De Almeida Braga, C., Ienco, D., Luque, S., Holon, F., Deter, J. Deep convolutional neural networks to monitor coralligenous reefs: operationalizing biodiversity and ecological assessment. Ecological Informatics - \textit{En cours de révision.}}


\bigskip

%%% AUTRES ARTICLES
\noindent\textbf{Autres publications}

\noindent{\href{https://doi.org/10.3389/fmars.2019.00276}{Holon, F., \textbf{Marre, G.}, Parravicini, V., Mouquet, N., Bockel, T., Descamp, P., Tribot, A-S., Boissery, P., Deter, J. A predictive model based on multiple coastal anthropogenic pressures explains the degradation status of a marine ecosystem: Implications for management and conservation. Biological Conservation 222, 125-135.}}

\bigskip

%%% RAPPORTS 
\noindent\textbf{Rapports d'activité}

\noindent{Holon, F., \textbf{Marre, G.}, Delaruelle, G., Holon, F., Guilbert, A., Deter, J., 2018. Application de la photogrammétrie à la surveillance biologique: mise au point de la méthode. Rapport d'activité pour l'\acrlong{aermc}, 57 pages.}

\noindent{Holon, F., \textbf{Marre, G.}, Delaruelle, Fery, C., G., Holon, F., Guilbert, A., Deter, J., 2018. Acquisitions photogrammétriques 2018 - 2019 et développements méthodologiques. Rapport d'activité pour l'\acrlong{aermc}, 145 pages.}

\bigskip

%%% VULGARISATION
\noindent\textbf{Articles de vulgarisation scientifique}

\noindent{\href{https://medtrix.fr/cahier-de-surveillance-3/}{\textbf{Marre, G.}, Luque, S., Holon, F., Boissery, P., Deter, J., 2018. Application de la photogrammétrie à la surveillance biologique des habitats sous-marins. Cahiers de la surveillance Medtrix n°3, Mars 2018.}}

\medskip

\noindent{\href{https://www.agropolis.fr/publications/sciences-marines-et-littorales-en-occitanie-dossier-thematique-agropolis-international.php}{\textbf{Marre, G.}, Luque, S., Holon, F., Boissery, P., Deter, J., 2019. La photogrammétrie : une méthode d’observation innovante pour l’étude et la conservation du milieu marin. Les dossiers d'Agropolis International N°24: Sciences marines et littorales en Occitanie, Février 2019.}}

\bigskip

%%% PRESENTATIONS ORALES
\noindent\textbf{Communications dans des congrès}

\noindent{\textbf{Marre, G.}, Ropars, B., 2017. At the crossroads between robotics, photogrammetry and ecology for the study of marine habitats. Communication orale au congrès Ecolotech. Salon de l'Ecologie, Montpellier, 9 novembre 2017.}

\medskip

\noindent{\textbf{Marre, G.}, Luque, S., Holon, F., Boissery, P., Deter, J., 2018. Rencontre entre robotique, photogrammétrie et écologie pour l'étude et le suivi des fonds marins. Communication orale au congrès Medtrix. Montpellier, 14 mars 2018.}

\medskip

\noindent{\textbf{Marre, G.}, Luque, S., Holon, F., Boissery, P., Deter, J., 2018. Développement de la photogrammétrie pour 
l’étude et le suivi d’habitats marins. Communication orale à l'Apéro Technique de l'\acrshort{osu} \acrshort{oreme}. Montpellier, 28 mai 2018.}

\medskip

\noindent{\textbf{Marre, G.}, Luque, S., Holon, F., Boissery, P., Deter, J., 2019. Méthodes photographiques pour la caractérisation de la structure et de la biodiversité des récifs coralligènes. Communication orale aux 10 ans de l'\acrshort{osu} \acrshort{oreme}. Montpellier, 11 octobre 2019.}

\medskip

\noindent{\textbf{Marre, G.}, Luque, S., Holon, F., Boissery, P., Deter, J., 2019. Suivi de communautés coralligènes par des
réseaux de neurones convolutifs. Communication orale au congrès Ecolotech. Salon de l'Ecologie, Montpellier, 7 novembre 2019.}

\medskip

\noindent{\textbf{Marre, G.}, Luque, S., Holon, F., Boissery, P., Deter, J., 2019. Photogrammétrie sous-marine et analyses d’images pour le suivi d’habitats benthiques Méditerranéens. Communication orale au congrès Merigeo 2020. Bordeaux, 16 mars 2020.}

\bigskip

%%% ENCADREMENT
\noindent\textbf{Activités d'encadrement}

\noindent{Co-encadrement du stage de fin d'études (M2 - cursus ingénieur) de \textbf{Gaïlé Lejay} en 2018, ayant donné lieu à la rédaction d'un mémoire de fin d'études : "Utilisation de modèles 3D en écologie sous-marine, détermination d’indicateurs dérivés des modèles et analyse de la variation des paramètres selon l’état de conservation des habitats".}

\medskip

\noindent{Co-encadrement du stage de fin d'études (M2 - Computer Science) de \textbf{Cédric De Almeida Braga} en 2019, ayant donné lieu à la rédaction d'un mémoire de fin d'études : "Characterization of coralligenous assemblages : from automatic image classification to 3D species mapping".}
 % avant-propos

% remerciements
\cleardoublepage
\pagestyle{plain}

\phantomsection
\addcontentsline{toc}{section}{REMERCIEMENTS}
{\centerline { {\sffamily \Large REMERCIEMENTS}}}

\vspace*{1cm}
\vskip 0.5cm
\noindent

Direction
Collègues
Amis
Famille
AMbar
Blabla


\vskip 0.3cm
\noindent
 \qquad  \qquad \qquad \qquad \qquad \qquad \qquad \qquad \quad \textbf{Guilhem}


% Table des matières
\cleardoublepage
\tableofcontents
\clearpage

% Table des figures
\addcontentsline{toc}{section}{\bf TABLE DES FIGURES}
\listoffigures 

% Table des tableaux
\clearpage
\addcontentsline{toc}{section}{\bf LISTE DES TABLEAUX}
\listoftables 

% Termes et abbréviations
\clearpage
\addcontentsline{toc}{section}{\bf LISTE DES SIGLES ET ABRÉVIATIONS}
\printnoidxglossary[type=\acronymtype, title=\Huge\bf{Liste des sigles et abréviations}]


\mainmatter
\pagestyle{fancy}
\pagenumbering{arabic} % Numérotation des pages en chiffres arabes
\onehalfspace

%%% Introduction
\clearpage
\part{Introduction, contexte et problématique}

\fancyhead[LE,RO]{Section \thesubsection}
\fancyhead[LO,RE]{CHAPITRE \thechapter}

% Pied de page avec le récif
%\fancyfoot[C]{\includegraphics[width=10cm]{images/misc/bandeau_bas.png}\\}

%\chapter{Introduction générale} \label{Introduction générale}

\newpage

\section[Habitats]{Les habitats benthiques Méditerranéens}
\subsection{Le contexte Méditerranéen: un hot spot de biodiversité}

\subsection[Les herbiers]{Les herbiers de Posidonie}
\subsubsection{Un habitat sensible}
\subsubsection{Suivi des herbiers de Posidonie en Méditerranéen française}

Service crops are crops grown with the aim of providing non-marketed ecosystem services, i.e. differing from food, fiber and fuel production. Vineyard soils face various agronomic issues such as poor organic carbon levels, erosion, fertility losses, and numerous studies have highlighted the ability of service crops to address these issues.

\noindent\textit{Cette partie a fait l'objet d'une publication dans la revue Agriculture, Ecosystem \& Environment :}

\subsection{Les récifs coralligènes}
\subsubsection{Petits frères des récifs coralliens}
\subsubsection{Suivis des récifs coralligènes en Méditerranée française}

\section[La photogrammétrie]{La photogrammétrie sous marine : principes et contraintes}
\subsection{La photogrammétrie}
\subsubsection{Principes théoriques}
\subsubsection{Acquisition des images}
\subsubsection{Résolution et précision des modèles}

\subsection[Spécificités marines]{Particularités inhérentes au milieu marin}
\subsubsection{Illumination naturelle variable et faible}
\subsubsection{Les problèmes de la réfraction}
\subsubsection{Présence d'objets mobiles sur la scène}

\subsection[Photogrammétrie et écologie marine]{Utilisation de la photogrammétrie en écologie marine}
\subsubsection{Assemblages 2D}
\subsubsection{Modèles 3D}

\newpage

\section[Analyse d'image]{Analyses d'images par apprentissage profond}
\subsection{Historique des réseaux de neurones convolutifs}
\subsubsection{Du neurone au réseau de neurones}
\subsubsection{Analyses d'images: les convolutions}
\subsubsection{Vers des réseaux de plus en plus profonds}

\newpage

\subsection[Reconnaissance d'espèces]{Application pour la reconnaissance d'espèces}
\subsubsection{Des données généralement complexes}

\small

\noindent{\href{https://doi.org/10.1016/j.agee.2017.09.030}{Garcia, L., Celette, F., Gary, C., Ripoche, A., Valdés-Gómez, H., Metay, A., 2018. Management of service crops for the provision of ecosystem services in vineyards: A review. Agriculture, Ecosystems \& Environment 251, 158–170}}

\noindent{\textit{Version auteur :} \url{https://hal.archives-ouvertes.fr/hal-01614417v2/document}}

\normalsize

\subsubsection{Stratégies d'optimisation des performances}
\subsubsection{Reconnaissance d'espèces de corail: un cas d'étude similaire}

\paragraph{Soil physical properties and water budget}

\paragraph{Soil chemical fertility}


\textbf{Acknowledgements}
The authors are grateful to Elaine Bonnier for English language corrections, and Hélène Frey for her beautiful picture. This research benefited from research activities carried out in the FertilCrop project, in the framework of the FP7 ERA-Net programme CORE Organic Plus. 

\section[Problématique]{Problématique de recherche}


\medskip
\noindent\textbf{\textit{Les marqueurs fonctionnels des espèces végétales présentes dans les enherbements viticoles permettent-ils d'expliquer l'impact de la communauté végétale sur les principaux services de support en viticulture ? L'étude des liens entre marqueurs fonctionnels et services de support permet-elle de sélectionner et piloter les espèces végétales pour maximiser la fourniture de services ?}}
\medskip

\noindent{Cette problématique s'est traduite en plusieurs questions de recherche :}

\begin{enumerate}[leftmargin=*]
\item \textbf{\textit{Quelles relations peut-on mettre en évidence entre les marqueurs fonctionnels des cultures de services en système viticole, et les services écosystémiques de support (protection des sols, fourniture en ressources hydriques et azotées) qu'elles rendent dans ces agrosystèmes ?}}
\item \textbf{\textit{Dans quelle mesure les marqueurs fonctionnels des communautés permettent-ils d'expliquer les services écosystémiques réalisés par les cultures de services dans ces agrosystèmes ?}}
\item \textbf{\textit{Peut-on piloter les services rendus par les cultures de services par le choix des espèces planifiées et de leurs traits, et par des interventions techniques au cours de leur cycle de croissance ?}}
\end{enumerate}

Pour y répondre, nous posons deux hypothèses de travail : $i)$ l'approche fonctionnelle par les traits des espèces et marqueurs fonctionnels des communautés est générique et peut être utilisée dans les systèmes viticoles enherbés, et $ii)$ la réalisation des services peut être évaluée par des indicateurs de fonctionnement du système sol-vigne-culture de service (stabilité structurale, couverture du sol, état hydrique et azoté du sol). 

\medskip

\noindent \textit{Les hypothèses suivantes seront testées dans cette thèse :}

\begin{description}
\item[Hypothèse 1]\label{c1:h} il existe des liens statistiques entre les marqueurs fonctionnels des communautés végétales composant les enherbements, leurs fonctions et les services qu'elles rendent aux viticulteur$\cdot$rice$\cdot$s.
\item[Hypothèse 2] : les marqueurs fonctionnels des plantes sont représentatifs du fonctionnement des espèces, et permettent de comparer les espèces et les communautés qu'elles composent en termes d'impacts sur le fonctionnement de l'agrosystème.
\item[Hypothèse 3] : les opérations techniques affectant les communautés pendant leur cycle de croissance permettent de modifier le niveau de réalisation des services 

\end{description}

En conséquence, afin de répondre à ces questions de recherche et tester les hypothèses définies, les objectifs de cette thèse sont les suivants :
\begin{enumerate}
\item la description des propriétés fonctionnelles de différentes communautés composées d'espèces planifiées et d'espèces spontanées, en particulier du point de vue de leur impact sur les ressources (eau, azote) et la structure du sol (stabilité) ;
\item l'évaluation et la description au champ des fonctions des cultures de services associées aux services de stabilisation des sols, de limitation du ruissellement, de fourniture en eau, et au service d'engrais vert ;
\item l'identification de valeurs de marqueurs fonctionnels des communautés permettant de placer le système dans une zone de compromis favorables entre services et dysservices ;
\item l'identification de leviers d'action techniques permettant le pilotage des services et dysservices par des interventions sur les cultures de services.
\end{enumerate}

La partie suivante présente la démarche générale de la thèse ainsi que les expérimentations mises en places et les mesures réalisées pour répondre aux objectifs ci-dessus.


%%% 
\clearpage
\part{Résultats}

\fancyhead[LE,RO]{\rightmark}
\fancyhead[LO,RE]{CHAPITRE \thechapter}

\addtocontents{toc}{\protect\setcounter{tocdepth}{0}}


\input{chapitre1-methode}

\clearpage
\pagestyle{plain}

% Bibliographie
\part{Références bibliographiques}
%\def\bibfont{\footnotesize}
\bibliographystyle{apalike}
\bibliography{references}

\end{document}
