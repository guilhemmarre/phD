\documentclass[11pt,a4paper]{book} % This style for A4 format.

% Importation des packages, commandes
\usepackage{amsmath, amssymb, amsthm}
\usepackage{graphicx,color}
\usepackage{multirow}
%\usepackage[top=3.3cm, bottom=3.5cm, left=3.3cm, right=3.3cm]{geometry}
\usepackage{epsfig}
\usepackage{color, colortbl}
\usepackage{setspace}
\usepackage{latexsym}
\usepackage{mathrsfs}
\usepackage{graphicx}
\usepackage{tikz}
\usepackage[titletoc]{appendix}
\usepackage{alltt}
\graphicspath{ {images/} }
\usepackage{longtable}
\usepackage{tikz}
%\usepackage{acronym}
\usepackage{pgfplots}
\usepackage[hidelinks]{hyperref}
%\usepackage{tocloft}
%\setcounter{tocdepth}{1}
%\renewcommand{\cftdot}{}
\usepackage{titletoc,tocloft}
%\setlength{\cftsecindent}{2cm}
\setlength{\cftsubsecindent}{1.2cm}
\setlength{\cftsubsubsecindent}{1cm}
%\dottedcontents{section}[1.5em]{}{1.3em}{.6em}

\usepackage[sc]{mathpazo}
\usepackage{sectsty}
\usepackage{titlesec}
\usepackage{textcase}



\usepackage{natbib}
\usepackage[english,frenchb]{babel}
\usepackage[latin1,utf8]{inputenc}
\usepackage[T1]{fontenc}
\usepackage{csquotes}

\usepackage{caption}
\captionsetup[table]{labelsep=space,labelfont=bf}
\captionsetup[figure]{labelsep=space,labelfont=bf}
\setcounter{page}{1}
\newtheorem{definition}{Definition}
\newcommand{\brdefinition}{\begin{definition}}
	\newcommand{\erdefinition}{\end{definition}}

\newtheorem{corollary}{Corollary}
\newcommand{\bcorollary}{\begin{corollary}}
	\newcommand{\ecorollary}{\end{corollary}}
\newtheorem{example}{Example}
\newcommand{\bexample}{\begin{example}}
	\newcommand{\eexample}{\end{example}}
\newtheorem{remark}{Remark}
\newcommand{\bremark}{\begin{remark}}
	\newcommand{\eremark}{\end{remark}}
\newcommand{\bproof}{{\bf {Proof:}}}
\newcommand{\eproof}{}
\newcommand{\bsolution}{{\bf {Solution:}}}
\newcommand{\esolution}{}
\newtheorem{theorem}{Theorem}
\newcommand{\btheorem}{\begin{theorem}}
	\newcommand{\etheorem}{\end{theorem}}
\newtheorem{lemma}{Lemma}
\newcommand{\blemma}{\begin{lemma}}
	\newcommand{\elemma}{\end{lemma}}
\newcommand{\UD}{\ensuremath{\bigtriangleup}}
\newcommand{\IC}{\ensuremath{\mathcal{C}} }
%\renewcommand{\rm}{\normalshape}
\newcommand{\ol}{\overline}
\def\cen{\centerline}
\def\pnq{\par\noindent\quad}
\def\pn{\par\noindent}
\newcommand{\non}{\nonumber}
\newcommand{\mC}{{\mathbb C}}
\newcommand{\mN}{{\mathbb N}}
\newcommand{\mU}{{\mathbb U}}
\newcommand{\cA}{{\mathcal A}}
\newcommand{\cR}{{\mathcal R}}
\newcommand{\cS}{{\mathcal S}}
\newcommand{\cT}{{\mathcal T}}
\newcommand{\cUCV}{{\mathcal UCV}}
\newcommand{\cST}{{\mathcal ST}}
\newcommand{\cK}{{\mathcal K}}
\newcommand{\ds}{\displaystyle}
\newcommand{\brdef}{\begin{defi}}
	\newcommand{\erdef}{\end{defi}}
\newcommand{\bcor}{\begin{cor}}
	\newcommand{\ecor}{\end{cor}}
\newcommand{\bth}{\begin{thm}}
	\newcommand{\ble}{\begin{lem}}
		\newcommand{\ele}{\end{lem}}
	\newcommand{\bcha}{\end{cha}}\pagestyle{plain}
\renewcommand{\theequation}{\thechapter.\arabic{equation}}
\renewcommand{\thetheorem}{\thesection.\arabic{theorem}}
\renewcommand{\thecorollary}{\thesection.\arabic{corollary}}
\renewcommand{\thelemma}{\thesection.\arabic{lemma}}
\renewcommand{\thedefinition}{\thesection.\arabic{definition}}
\renewcommand{\theexample}{\thesection.\arabic{example}}
\renewcommand{\theremark}{\thesection.\arabic{remark}}
\renewcommand{\thechapter}{\arabic{chapter}}

\def\cen{\centerline}
\def\pnq{\par\noindent\quad}
\def\pn{\par\noindent}
\def\sevenpoint{%
	\def\rm{\sevenrm}%
	\def\it{\sevenit}%
	\def\bf{\sevenbf}%
	\rm}
%\pretoler
\renewcommand{\theequation}{\thechapter.\arabic{equation}}
\theoremstyle{definition}
\renewcommand*{\proofname}{{\rm Proof}}
%%%%%%%%%%%%%%%%%%%%%%%%%%%%%%%%%%%%%%%%%%%%%%%%%%%%%%%%%%%%%%%%%%%Chapter title format
\usepackage{sectsty}
\usepackage{titlesec}
\chapterfont{\centering }
\titleformat{\chapter}[display]
{\bf\centering}
{\chaptertitlename\ \thechapter}{16pt}{\large}
\sectionfont{\normalfont}
%\renewcommand*\contentsname{\large \centerline{TABLE OF CONTENTS}}
\renewcommand\cftchappresnum{} % prefix "Chapter " to chapter number in ToC
\cftsetindents{chapter}{0em}{3em}      % set amount of indenting
\cftsetindents{section}{0em}{3em}

\titleformat{\subsubsection}
{\normalfont\normalsize\bfseries}{\thesubsubsection}{1em}{}

% For Terms and Abbreviations 
\usepackage[acronym,section]{glossaries}
\makenoidxglossaries
%\renewcommand*\glspostdescription{\cftdotfill{\cftsecdotsep}}
%\renewcommand{\glossarysection}[2][]{{\centering\bfseries\MakeTextUppercase{#2}\par}}

%\renewcommand*{\bibpagespunct}{\addcolon\space}

\definecolor{butter1}{rgb}{0.988,0.914,0.310}
\definecolor{chocolate1}{rgb}{0.914,0.725,0.431}
\definecolor{chameleon1}{rgb}{0.541,0.886,0.204}
\definecolor{skyblue1}{rgb}{0.447,0.624,0.812}
\definecolor{plum1}{rgb}{0.678,0.498,0.659}
\definecolor{scarletred1}{rgb}{0.937,0.161,0.161}


%For Table Column Width
\usepackage{array}
\newcolumntype{L}[1]{>{\raggedright\let\newline\\\arraybackslash\hspace{0pt}}m{#1}}
\newcolumntype{C}[1]{>{\centering\let\newline\\\arraybackslash\hspace{0pt}}m{#1}}
\newcolumntype{R}[1]{>{\raggedleft\let\newline\\\arraybackslash\hspace{0pt}}m{#1}}

% sur le net
\renewcommand{\arraystretch}{1.2} 
\usepackage{makecell}%To keep spacing of text in tables
\setcellgapes{4pt}%parameter for the spacing

\usepackage{etoolbox}

\makeatletter
\patchcmd{\ttlh@hang}{\parindent\z@}{\parindent\z@\leavevmode}{}{}
\patchcmd{\ttlh@hang}{\noindent}{}{}{}
\makeatother

\usepackage{a4wide}
\usepackage{tabularx}
\usepackage{chngpage}
\usepackage{siunitx}
\usepackage{subscript}
\usepackage{rotfloat}
\usepackage{rotating}
\newcommand*{\innerTabular}[1]{{\begin{tabular}[c]{@{}c@{}}#1\end{tabular}}}
\usepackage{pdfpages}
\usepackage{booktabs}
\usepackage{calc}
\usepackage{fancybox}
\usepackage{enumitem}
\usepackage{eurosym}
\usepackage[skins]{tcolorbox}
\usepackage{lscape}
%\usepackage{lineno}
%\usepackage[nottoc,numbib]{tocbibind} %mettre la biblio dans la toc
%\newcommand{\nocontentsline}[3]{}
%\newcommand{\tocless}[2]{\bgroup\let\addcontentsline=\nocontentsline#1{#2}\egroup}

\newcommand{\myparagraph}[1]{\paragraph{#1}\mbox{}\\} %titre de paragraphe

\newcolumntype{Y}{>{\centering\arraybackslash}X} %modif de tabularx pour centrer

\definecolor{Gray}{gray}{0.9} %couleur dans les tableaux

\renewcommand{\thesection}{\arabic{section}} %numeros romains

\usepackage[Sonny]{fncychap} % joli chapitre

\usepackage{fancyhdr}
\fancyhead{}

\renewcommand{\labelitemi}{$\bullet$}

\setcounter{secnumdepth}{3}

%from nico with love
\newenvironment{changemargin}[3]{\begin{list}{}{%
\setlength{\topsep}{0pt}%
\setlength{\leftmargin}{0pt}%
\setlength{\rightmargin}{0pt}%
\setlength{\topmargin}{0pt}%
\setlength{\listparindent}{\parindent}%
\setlength{\itemindent}{\parindent}%
\setlength{\parsep}{0pt plus 1pt}%
\addtolength{\leftmargin}{#1}%
\addtolength{\rightmargin}{#2}%
\addtolength{\topmargin}{#3}%
}\item }{\end{list}}

% Importation des acronymes
% On défini le style: long et court
\setacronymstyle{long-sc-short}


%%% Mathématiques
\newacronym{pca}{PCA}{principal components analysis}

\newacronym{hac}{HAC}{hierarchical ascendant classification}

\newacronym{rmse}{RMSE}{Root Mean Square Error}

\newacronym{lmm}{LMM}{Linear Mixed effect Model}

\newacronym{sd}{SD}{Standard Deviation}

\newacronym{anova}{ANOVA}{analysis of variance}

\newacronym{aic}{AIC}{Akaike information criterion}



%%% Photogrammétrie
\newacronym{gcp}{GCP}{Ground Control Point}

\newacronym{gsd}{GSD}{Ground Sample Distance}

\newacronym{gcp_rmse}{GCP RMSE}{Ground Control Point Root Mean Square Error}

\newacronym{tpt}{TPT}{Total Processing Time}

\newacronym{c2m}{C2M}{Cloud-to-mesh distance}



%%% Andromède
\newacronym{tempo}{TEMPO}{Réseau de surveillance des herbiers de Posidonie}

\newacronym{recor}{RECOR}{Réseau de surveillance des récifs coralligènes}



%%% Institutions
\newacronym{inrae}{INRAE}{Institut National de Recherche pour l'Agriculture, l'alimentation et l'Environnement}

\newacronym{ird}{IRD}{Institut de Recherche pour le Développement}

\newacronym{cnrs}{CNRS}{Centre National de la Recherche Scientifique}

\newacronym{cirad}{CIRAD}{Centre de coopération Internationale en Recherche Agronomique pour le Développement}

\newacronym{umr}{UMR}{Unité Mixte de Recherche}

\newacronym{tetis}{TETIS}{Territoire, Environnement, Télédétection et Information Spatiale}

\newacronym{marbec}{MARBEC}{MARine Biodiversity, Exploitation and Conservation}

\newacronym{anrt}{ANRT}{Agence Nationale pour la Recherche Technologique}

\newacronym{aermc}{AERMC}{Agence de l'Eau Rhône-Méditerranée-Corse}

\newacronym{osu}{OSU}{Observatoire des Sciences de l'Univers}

\newacronym{oreme}{OREME}{Observatoire de REcherche Méditerranéen de l'Environnement}
%\glsaddall

% Booléen pour afficher ou pas les commentaires de chapitres (pour la rédaction uniquement)
\setboolean{displaytete}{false}

%%% Debut du document
\begin{document}

% Page de garde
\includepdf[scale=1, pages=-,pagecommand={\thispagestyle{empty}}]{./coverpage/couverture_these_UM.pdf}

% Style préambule pour la première partie
\pagestyle{preambule}

% Page blanche sans numérotation
\newpage
\clearpage
\thispagestyle{empty}
\hfill
\newpage

% Début numérotation chiffres romains
\pagenumbering{roman}

% Remerciements
\pagestyle{preambule}

\phantomsection
\addcontentsline{toc}{section}{REMERCIEMENTS}
{\centerline { {\sffamily \Large REMERCIEMENTS}}}

\vspace*{1cm}
\vskip 0.5cm
\noindent

Direction
Collègues
Amis
Famille
AMbar
Blabla


\vskip 0.3cm
\noindent
 \qquad  \qquad \qquad \qquad \qquad \qquad \qquad \qquad \quad \textbf{Guilhem}


% Avant propos
\newpage
\begin{spacing}{1.5}
\phantomsection
\addcontentsline{toc}{section}{AVANT-PROPOS}
{\centerline { {\sffamily \Large AVANT-PROPOS}}}

\vspace*{1cm}
\vskip 0.5cm
\noindent

\normalsize
\noindent Cette thèse à été réalisée de septembre 2017 à juillet 2020 entre Andromède Océanologie et l'\acrshort{umr} \acrshort{tetis} (\acrshort{inrae}, \acrshort{cirad}, \acrshort{cnrs}, AgroParisTech) à Montpellier, en collaboration avec l'\acrshort{umr} \acrshort{marbec} (Université de Montpellier, \acrshort{cnrs}, \acrshort{ird}, Ifremer, Montpellier). Ce travail a été cofinancé par Andromède Océanologie et l'\gls{anrt}, et les missions de terrain ont eu lieu dans le cadre des suivis environnementaux \acrshort{tempo} et \acrshort{recor} financés par l'\gls{aermc}.

\medskip

\noindent Cette thèse a été dirigée par Julie Deter et Sandra Luque, et a bénéficié des conseils avisés de Dino Ienco sur les aspects deep learning ainsi que des membres du comité de thèse Nicolas Mouquet, Gérard Subsol, François Guilhaumon, Maria Dornelas et Christiane Weber. 

\bigskip

%%% ARTICLES
\noindent\textbf{Publications}

% 1er article méthodo
\noindent{\href{https://doi.org/10.3389/fmars.2019.00276}{\textbf{Marre, G.}, Holon, F., Luque, S., Boissey, P., Deter, J., 2019. Monitoring marine habitats with photogrammetry: a cost-effective, accurate, precise and high-resolution reconstruction method. Frontiers in Marine Science 6:276, 158–170}.}

\medskip

% 2ème article herbiers
\noindent{\textbf{Marre G.}, Deter, J., Holon, F., Boissery, P., Luque, S. Fine-scale automatic mapping of living Posidonia oceanica seagrass beds with underwater photogrammetry. Marine Ecology Progress Series - \textit{Accepté sous conditions de modifications mineures.}}

\medskip

% 3ème article 
\noindent{\textbf{Marre G.}, De Almeida Braga, C., Ienco, D., Luque, S., Holon, F., Deter, J. Deep convolutional neural networks to monitor coralligenous reefs: operationalizing biodiversity and ecological assessment. Ecological Informatics - \textit{En cours de révision.}}


\bigskip

%%% AUTRES ARTICLES
\noindent\textbf{Autres publications}

\noindent{\href{https://doi.org/10.3389/fmars.2019.00276}{Holon, F., \textbf{Marre, G.}, Parravicini, V., Mouquet, N., Bockel, T., Descamp, P., Tribot, A-S., Boissery, P., Deter, J. A predictive model based on multiple coastal anthropogenic pressures explains the degradation status of a marine ecosystem: Implications for management and conservation. Biological Conservation 222, 125-135.}}

\bigskip

%%% RAPPORTS 
\noindent\textbf{Rapports d'activité}

\noindent{Holon, F., \textbf{Marre, G.}, Delaruelle, G., Holon, F., Guilbert, A., Deter, J., 2018. Application de la photogrammétrie à la surveillance biologique: mise au point de la méthode. Rapport d'activité pour l'\acrlong{aermc}, 57 pages.}

\noindent{Holon, F., \textbf{Marre, G.}, Delaruelle, Fery, C., G., Holon, F., Guilbert, A., Deter, J., 2018. Acquisitions photogrammétriques 2018 - 2019 et développements méthodologiques. Rapport d'activité pour l'\acrlong{aermc}, 145 pages.}

\bigskip

%%% VULGARISATION
\noindent\textbf{Articles de vulgarisation scientifique}

\noindent{\href{https://medtrix.fr/cahier-de-surveillance-3/}{\textbf{Marre, G.}, Luque, S., Holon, F., Boissery, P., Deter, J., 2018. Application de la photogrammétrie à la surveillance biologique des habitats sous-marins. Cahiers de la surveillance Medtrix n°3, Mars 2018.}}

\medskip

\noindent{\href{https://www.agropolis.fr/publications/sciences-marines-et-littorales-en-occitanie-dossier-thematique-agropolis-international.php}{\textbf{Marre, G.}, Luque, S., Holon, F., Boissery, P., Deter, J., 2019. La photogrammétrie : une méthode d’observation innovante pour l’étude et la conservation du milieu marin. Les dossiers d'Agropolis International N°24: Sciences marines et littorales en Occitanie, Février 2019.}}

\bigskip

%%% PRESENTATIONS ORALES
\noindent\textbf{Communications dans des congrès}

\noindent{\textbf{Marre, G.}, Ropars, B., 2017. At the crossroads between robotics, photogrammetry and ecology for the study of marine habitats. Communication orale au congrès Ecolotech. Salon de l'Ecologie, Montpellier, 9 novembre 2017.}

\medskip

\noindent{\textbf{Marre, G.}, Luque, S., Holon, F., Boissery, P., Deter, J., 2018. Rencontre entre robotique, photogrammétrie et écologie pour l'étude et le suivi des fonds marins. Communication orale au congrès Medtrix. Montpellier, 14 mars 2018.}

\medskip

\noindent{\textbf{Marre, G.}, Luque, S., Holon, F., Boissery, P., Deter, J., 2018. Développement de la photogrammétrie pour 
l’étude et le suivi d’habitats marins. Communication orale à l'Apéro Technique de l'\acrshort{osu} \acrshort{oreme}. Montpellier, 28 mai 2018.}

\medskip

\noindent{\textbf{Marre, G.}, Luque, S., Holon, F., Boissery, P., Deter, J., 2019. Méthodes photographiques pour la caractérisation de la structure et de la biodiversité des récifs coralligènes. Communication orale aux 10 ans de l'\acrshort{osu} \acrshort{oreme}. Montpellier, 11 octobre 2019.}

\medskip

\noindent{\textbf{Marre, G.}, Luque, S., Holon, F., Boissery, P., Deter, J., 2019. Suivi de communautés coralligènes par des
réseaux de neurones convolutifs. Communication orale au congrès Ecolotech. Salon de l'Ecologie, Montpellier, 7 novembre 2019.}

\medskip

\noindent{\textbf{Marre, G.}, Luque, S., Holon, F., Boissery, P., Deter, J., 2019. Photogrammétrie sous-marine et analyses d’images pour le suivi d’habitats benthiques Méditerranéens. Communication orale au congrès Merigeo 2020. Bordeaux, 16 mars 2020.}

\bigskip

%%% ENCADREMENT
\noindent\textbf{Activités d'encadrement}

\noindent{Co-encadrement du stage de fin d'études (M2 - cursus ingénieur) de \textbf{Gaïlé Lejay} en 2018, ayant donné lieu à la rédaction d'un mémoire de fin d'études : "Utilisation de modèles 3D en écologie sous-marine, détermination d’indicateurs dérivés des modèles et analyse de la variation des paramètres selon l’état de conservation des habitats".}

\medskip

\noindent{Co-encadrement du stage de fin d'études (M2 - Computer Science) de \textbf{Cédric De Almeida Braga} en 2019, ayant donné lieu à la rédaction d'un mémoire de fin d'études : "Characterization of coralligenous assemblages : from automatic image classification to 3D species mapping".}

\end{spacing}

% Table des matières
\newpage
\begingroup
\hypersetup{linkcolor=black}
\begin{spacing}{1.5}
	\tableofcontents
\end{spacing}
\endgroup
\newpage

% Liste des figures
\newpage
%For Removing Extra Space in List of Tables and Figures before every chapter
\let\origaddvspace\addvspace % on stock addvspace dans origaddvspace
\renewcommand{\addvspace}[1]{}
\begingroup
\hypersetup{linkcolor=black}
\begin{spacing}{1.5}
	\phantomsection
	\addcontentsline{toc}{section}{LISTE DES FIGURES}
	\setcounter{lofdepth}{2} \listoffigures
\end{spacing}
\endgroup
\newpage

% Liste des tableaux
\newpage
\begingroup
\hypersetup{linkcolor=black}
\begin{spacing}{1.5}
	\phantomsection
	\addcontentsline{toc}{section}{LISTE DES TABLEAUX}
	\setcounter{lotdepth}{2} \listoftables
\end{spacing}
\endgroup
\newpage

% Termes et abbréviations
% mise en page un peu à la mano car pas identique aux autres de base
\newpage
\vspace*{2cm}
\begingroup
\hypersetup{linkcolor=bluecite}
\begin{spacing}{1.5}
    \phantomsection
    \addcontentsline{toc}{section}{LISTE DES SIGLES ET ABRÉVIATIONS}
    \printnoidxglossary[type=\acronymtype, title=\Huge\bf\color{black}{LISTE DES ABRÉVIATIONS}\vspace*{0.5cm}]
\end{spacing}
\endgroup
\newpage

%\clearpage
%\pagestyle{preambule}
%\addcontentsline{toc}{section}{LISTE DES SIGLES ET ABRÉVIATIONS}
%\printnoidxglossary[type=\acronymtype, title={\centerline { {\sffamily \Large LISTE DES SIGLES ET ABRÉVIATIONS}}}\vspace*{0.5cm}]


% Restauration de la valeur originale de addvspace
\renewcommand{\addvspace}[1]{\origaddvspace{#1}}

\mainmatter % partie principale du bouquin
\pagestyle{main}
%\pagenumbering{arabic} % On commence la numérotation en chiffres arabes

\part{Introduction, contexte et problématique}
\pagestyle{intro}
%\chapter{Introduction générale} \label{Introduction générale}

\newpage

\section[Habitats]{Les habitats benthiques Méditerranéens}
\subsection{Le contexte Méditerranéen: un hot spot de biodiversité}

\subsection[Les herbiers]{Les herbiers de Posidonie}
\subsubsection{Un habitat sensible}
\subsubsection{Suivi des herbiers de Posidonie en Méditerranéen française}

Service crops are crops grown with the aim of providing non-marketed ecosystem services, i.e. differing from food, fiber and fuel production. Vineyard soils face various agronomic issues such as poor organic carbon levels, erosion, fertility losses, and numerous studies have highlighted the ability of service crops to address these issues.

\noindent\textit{Cette partie a fait l'objet d'une publication dans la revue Agriculture, Ecosystem \& Environment :}

\subsection{Les récifs coralligènes}
\subsubsection{Petits frères des récifs coralliens}
\subsubsection{Suivis des récifs coralligènes en Méditerranée française}

\section[La photogrammétrie]{La photogrammétrie sous marine : principes et contraintes}
\subsection{La photogrammétrie}
\subsubsection{Principes théoriques}
\subsubsection{Acquisition des images}
\subsubsection{Résolution et précision des modèles}

\subsection[Spécificités marines]{Particularités inhérentes au milieu marin}
\subsubsection{Illumination naturelle variable et faible}
\subsubsection{Les problèmes de la réfraction}
\subsubsection{Présence d'objets mobiles sur la scène}

\subsection[Photogrammétrie et écologie marine]{Utilisation de la photogrammétrie en écologie marine}
\subsubsection{Assemblages 2D}
\subsubsection{Modèles 3D}

\newpage

\section[Analyse d'image]{Analyses d'images par apprentissage profond}
\subsection{Historique des réseaux de neurones convolutifs}
\subsubsection{Du neurone au réseau de neurones}
\subsubsection{Analyses d'images: les convolutions}
\subsubsection{Vers des réseaux de plus en plus profonds}

\newpage

\subsection[Reconnaissance d'espèces]{Application pour la reconnaissance d'espèces}
\subsubsection{Des données généralement complexes}

\small

\noindent{\href{https://doi.org/10.1016/j.agee.2017.09.030}{Garcia, L., Celette, F., Gary, C., Ripoche, A., Valdés-Gómez, H., Metay, A., 2018. Management of service crops for the provision of ecosystem services in vineyards: A review. Agriculture, Ecosystems \& Environment 251, 158–170}}

\noindent{\textit{Version auteur :} \url{https://hal.archives-ouvertes.fr/hal-01614417v2/document}}

\normalsize

\subsubsection{Stratégies d'optimisation des performances}
\subsubsection{Reconnaissance d'espèces de corail: un cas d'étude similaire}

\paragraph{Soil physical properties and water budget}

\paragraph{Soil chemical fertility}


\textbf{Acknowledgements}
The authors are grateful to Elaine Bonnier for English language corrections, and Hélène Frey for her beautiful picture. This research benefited from research activities carried out in the FertilCrop project, in the framework of the FP7 ERA-Net programme CORE Organic Plus. 

\section[Problématique]{Problématique de recherche}


\medskip
\noindent\textbf{\textit{Les marqueurs fonctionnels des espèces végétales présentes dans les enherbements viticoles permettent-ils d'expliquer l'impact de la communauté végétale sur les principaux services de support en viticulture ? L'étude des liens entre marqueurs fonctionnels et services de support permet-elle de sélectionner et piloter les espèces végétales pour maximiser la fourniture de services ?}}
\medskip

\noindent{Cette problématique s'est traduite en plusieurs questions de recherche :}

\begin{enumerate}[leftmargin=*]
\item \textbf{\textit{Quelles relations peut-on mettre en évidence entre les marqueurs fonctionnels des cultures de services en système viticole, et les services écosystémiques de support (protection des sols, fourniture en ressources hydriques et azotées) qu'elles rendent dans ces agrosystèmes ?}}
\item \textbf{\textit{Dans quelle mesure les marqueurs fonctionnels des communautés permettent-ils d'expliquer les services écosystémiques réalisés par les cultures de services dans ces agrosystèmes ?}}
\item \textbf{\textit{Peut-on piloter les services rendus par les cultures de services par le choix des espèces planifiées et de leurs traits, et par des interventions techniques au cours de leur cycle de croissance ?}}
\end{enumerate}

Pour y répondre, nous posons deux hypothèses de travail : $i)$ l'approche fonctionnelle par les traits des espèces et marqueurs fonctionnels des communautés est générique et peut être utilisée dans les systèmes viticoles enherbés, et $ii)$ la réalisation des services peut être évaluée par des indicateurs de fonctionnement du système sol-vigne-culture de service (stabilité structurale, couverture du sol, état hydrique et azoté du sol). 

\medskip

\noindent \textit{Les hypothèses suivantes seront testées dans cette thèse :}

\begin{description}
\item[Hypothèse 1]\label{c1:h} il existe des liens statistiques entre les marqueurs fonctionnels des communautés végétales composant les enherbements, leurs fonctions et les services qu'elles rendent aux viticulteur$\cdot$rice$\cdot$s.
\item[Hypothèse 2] : les marqueurs fonctionnels des plantes sont représentatifs du fonctionnement des espèces, et permettent de comparer les espèces et les communautés qu'elles composent en termes d'impacts sur le fonctionnement de l'agrosystème.
\item[Hypothèse 3] : les opérations techniques affectant les communautés pendant leur cycle de croissance permettent de modifier le niveau de réalisation des services 

\end{description}

En conséquence, afin de répondre à ces questions de recherche et tester les hypothèses définies, les objectifs de cette thèse sont les suivants :
\begin{enumerate}
\item la description des propriétés fonctionnelles de différentes communautés composées d'espèces planifiées et d'espèces spontanées, en particulier du point de vue de leur impact sur les ressources (eau, azote) et la structure du sol (stabilité) ;
\item l'évaluation et la description au champ des fonctions des cultures de services associées aux services de stabilisation des sols, de limitation du ruissellement, de fourniture en eau, et au service d'engrais vert ;
\item l'identification de valeurs de marqueurs fonctionnels des communautés permettant de placer le système dans une zone de compromis favorables entre services et dysservices ;
\item l'identification de leviers d'action techniques permettant le pilotage des services et dysservices par des interventions sur les cultures de services.
\end{enumerate}

La partie suivante présente la démarche générale de la thèse ainsi que les expérimentations mises en places et les mesures réalisées pour répondre aux objectifs ci-dessus.


% Résultats: les articles
\part{Résultats}

\pagestyle{titre_chapitre}
\input{chapitre1-methode}

\pagestyle{titre_chapitre}
% Définition du nom du chapitre
\chapter{Fine scale monitoring of living Posidonia oceanica} \label{chapitre2-herbiers}

%%%%%%%%%%%%%%%%%%%%%%%%%%%%%
%%% Figure cover chapitre %%%
%%%%%%%%%%%%%%%%%%%%%%%%%%%%%
\pagestyle{main}
\begin{figure}[H] 
	\begin{center}
	\includegraphics[width=\linewidth]{./chapitre2/cover.jpg}
    \end{center}
\end{figure}

% Bullet points du début de chapitre
\begin{colbox}{resume}
  \vspace{-2pt}
{\color{textresume}\small
\begin{itemize}[leftmargin=0in]\itemsep3pt
\item les garrigues sont des milieux naturels méditerranéens ouverts, caractérisés par une végétation très hétérogène~;
\item cette hétérogénéité se caractérise par l'assemblage en mosaïque de quatre strates verticales à une échelle très fine~: le sol nu, l'herbe, les ligneux bas et les ligneux hauts~;
\item l'hétérogénéité de cette mosaïque varie de manière continue dans le paysage~;
\item cette hétérogénéité est associée à une très forte biodiversité floristique et faunistique~;
\item cette hétérogénéité est la résultante~:
\begin{itemize}
  \item d'une importante variabilité climatique et topographique~;
  \item d'activités humaines qui ont façonné les milieux de garrigues~;
\end{itemize}
\end{itemize}
}
\vspace{-2pt}
%\end{fullminipage}
\end{colbox}

\clearpage

\noindent\textbf{Fine scale monitoring of living Posidonia oceanica}

% Auteurs
\noindent Guilhem Marre, Florian Holon, Sandra Luque, Pierre Boissery et Julie Deter

% NB sans indentation
\noindent\textit{En cours de révision dans...}

\noindent\textbf{Abstract}


\noindent\textbf{Keywords}


\section[Intruduction]{Introduction}\label{chapitre2_1}

\newpage

\section{Materials and methods}\label{chapitre2_2}

\newpage

\section{Results}\label{chapitre2_3}

\newpage

\section{Discussion}\label{chapitre2_4}

\newpage

\section{Conclusion}\label{chapitre2_5}

\newpage

\section{Conflict of interest}\label{chapitre2_6}
The authors declare that the research was conducted in the absence of any commercial or financial relationships that could be construed as a potential conflict of interest.

\newpage

\section{Author contributions}\label{chapitre2_7}


\section{Funding}\label{chapitre2_8}
This study beneficiated from a financing of the French Water Agency (\acrlong{aermc}) (convention n° 2017-1118) and the LabCom InToSea (ANR Labcom 2, Université de Montpellier UMR 9190 MARBEC / Andromède Océanologie). Guilhem Marre received a PhD grant (2017-2020) funded by Agence Nationale pour la Recherche Technologique (ANRT) and Andromède Océanologie.

\section{Acknowledgments}\label{chapitre2_9}

%\input{./chapter5/ch5}

%\input{./chapter6/ch6}

% Discussion générale
\pagestyle{main}
\part{Discussion Générale}
\pagestyle{discussion}
\chapter*{} \label{chapitre6-discussion} % "*" pour ne pas afficher le nom de chapitre
\setcounter{section}{0} % 
\renewcommand*{\theHsection}{chY.\the\value{section}}
% See https://tex.stackexchange.com/questions/71162/reset-section-numbering-between-unnumbered-chapters
\pagestyle{discussion}

\section[Premiere partie]{Premiere partie de discussion}

\subsection{Premiere souspartie de section}

\newpage

\subsection{Deuxieme souspartie de section}

\newpage

\section[Deuxieme partie]{Deuxieme partie de discussion}

\subsection{Premiere souspartie de section}

\newpage

\subsection{Deuxieme souspartie de section}

\newpage

\pagestyle{main}
\backmatter% possiblement inutile

% Bibliographie
\part*{REFERENCES BIBLIOGRAPHIQUES}
\pagestyle{references}
\phantomsection
\bookmarksetup{startatroot}
\pagestyle{references}
%\addcontentsline{toc}{section}{RÉFÉRENCES BIBLIOGRAPHIQUES}
\bibliographystyle{apalike} % chicago
\bibliography{references}
\titleformat{\part}[display]
{\sffamily \centering \huge \titlerule[1pt]\vspace{8pt}}%format
{}%
{5pt}%
{}%
[\vspace{12pt}\titlerule]

% Annexes
\pagestyle{main}
\part*{ANNEXES}
\bookmarksetupnext{level=section}
\pagestyle{appendix}
\clearpage
\begin{appendices}
	%\appendixpage % rajouter une page vide avec "Annexes"
	%\noappendicestocpagenum
	%\addappheadtotoc
	\chapter{Article 1: Monitoring...} %\label{annexe1-article1}
    \includepdf[scale=1, pages=-,pagecommand={\thispagestyle{empty}}]{./annexes/Marre_2019_monitoring_marine_habitats_with_PG.pdf}
	%\input{part-1}
	%\input{./appendix/part-2}
	%\input{./appendix/part-3}
	%\input{./appendix/part-4}
\end{appendices}


%\bookmarksetupnext{level=section}
%\pagestyle{appendix}
%\begin{appendix}

	%\appendixpage
%	\noappendicestocpagenum
%	\appendix
	%\addappheadtotoc
	
	% 1er article
	%\chapter{Article 1: Monitoring...} \label{annexe1-article1}
	%\includepdf[scale=1, pages=-,pagecommand={\thispagestyle{empty}}]{./annexes/Marre_2019_monitoring_marine_habitats_with_PG.pdf}
	
%	\newpage
	
	% 2eme article
	%\chapter{Article 2: Fine scale...} \label{annexe1-article2}
	%\includepdf[scale=1, pages=-,pagecommand={\thispagestyle{empty}}]{./annexes/}
    
 %   \newpage
	
	% 3eme article
	%\chapter{Article 3: Deep learning...} \label{annexe1-article3}
	%\includepdf[scale=1, pages=-,pagecommand={\thispagestyle{empty}}]{./annexes/}

 %   \newpage

%	\chapter{Article de vulgarisation : La garrigue vue du ciel} \label{vulgarisation}

\newpage
\newpage

La garrigue, mosaïque de paysages, tend à s'uniformiser face à l'avancée progressive de la forêt. Mais contrairement aux idées reçues, le regain de la forêt n'est pas toujours positif. C'est le cas pour la garrigue où la fermeture des milieux est problématique. Des méthodes basées sur la reconnaissance de motifs à partir d'images satellites sont développées pour comprendre et cartographier le paysage et aider les acteurs concernés par le devenir de la garrigue, comme Jean-René, éleveur de chèvres dans le causse d'Aumelas.

Lorsque Jean-René, le petit fils de M. Seguin, se promène dans la garrigue à la recherche de ses chèvres encore égarées - décidément, c'est de famille - il y passe parfois des heures. Pas facile en effet de s'y retrouver dans ce dédale de végétation : ça pique, ça gratte et puis on n'y voit pas grand-chose avec cette mosaïque de grands chênes verts, ces touffes de chênes kermès, de genêts et autres genévriers.

La garrigue, c'est un gigantesque puzzle composé de plusieurs pièces imbriquées : des arbres, des buissons, de l'herbe. Seulement, depuis que les collègues de Jean-René ont arrêté leur activité pastorale à cause d'une augmentation de la compétition économique [1], \#crisedel'emploi, l'herbe et les buissons ne sont plus broutés, ils poussent en paix et le paysage tend à se fermer pour laisser place à une forêt dense.

\textbf{Mais alors pourquoi c'est pas cool le regain de la forêt ?}


Cette fermeture des milieux pose quelques problèmes. En premier lieu pour Jean-René lui-même qui a de plus en plus de mal à savoir où faire pâturer ses chèvres : où peuvent-elles circuler dans ce labyrinthe ? Où est la boustifaille ? Mais surtout où diable se sont-elles encore fourrées ? Et puis ça dégrade aussi la qualité du paysage : au grand dam d'Isodore De Lahaute, citadin en mal de nature qui ne verra plus que des arbres à perte de vue lors de ses sorties champêtres, au lieu de petits murets \emph{sooo} XIXième siècle et des prairies sèches parsemées d'orchidées ou d'iris sauvages.

L'idéal, ce serait d'avoir une carte pour s'orienter, ou bien un (très grand) escabeau pour y voir un peu plus clair. Jean-René a bien pensé à acheter une montgolfière, comme à l'époque de la grande guerre où des clichés étaient pris depuis des ballons pour aider les poilus à s'y retrouver dans les tranchées [2]. Seulement, c'est un peu cher (\#crisedel'emploi n°2) et puis maintenant il y a bien mieux : les images satellites. Elles ont maintenant une telle précision qu'il est presque possible de voir la calvitie naissante de Jean-René.

\textbf{Des satellites pour prendre de la hauteur}

Les motifs dessinés par les arbres et par les buissons sont maintenant bien apparents pour Jean-René muni de telles images. C'est facile à reconnaître. En effet, question motif, le cerveau est balèze : il est capable de reconnaître quantités de textures très rapidement, de manière intuitive. Paradoxalement, cette intuition pose problème : comme on n'a pas tous le même cerveau, nous n'avons pas tous les mêmes perceptions. Selon que ce soit Jean-René ou son filleul qui essaie de déterminer la structure de la garrigue, l'interprétation ne sera pas tout à fait la même, d'autant que Jean-René, un peu astigmate, ne voit plus très bien.

Heureusement pour lui, les maths associés à l'informatique parviennent à traduire la perception humaine des textures [3,4]. Si les algorithmes de reconnaissance de motifs ne détectent pas aussi bien les textures que l’œil humain (\#manisstillthebest), ils présentent l'avantage d'être automatiques et de fournir des résultats constants dans le temps.

\textbf{Analyser des motifs pour comprendre le paysage}

Une approche possible en détection automatique de texture est l'approche fréquentielle. Le principe est de repérer des structures qui se répètent dans l'espace. Par exemple, Jean-René, adepte de la pizza, a bien remarqué que dans celle qu'il affectionne tout particulièrement, la pizza reine, il y a des motifs qui se répètent. Son préféré : celui dessiné par les champignons, bien remarquable. En effet : "une bonne reine est composée de champignons de 2 cm environ, espacés d'environ 3 cm" (propos recueillis auprès de Giovanni, pizzaiolo étoilé). Une reine de 33 cm, cuisinée selon les principes évoqués par Giovanni présente donc un motif d'une fréquence de 6,5 champignons par largeur de pizza. Grâce à cette approche, on peut traduire l'aspect visuel d'une pizza en termes fréquentiels. Ainsi on aura, par largeur de pizza : 6,5 champignons, 8 lardons, 56 bouts de fromages râpes, etc.

C'est bien beau tout ça, l'approche de fréquentielle, les motifs, mais il fait comment le Jean-René pour retrouver ses chèvres avec ça ?

En fait, ces motifs peuvent directement être liés à d'autres mesures qui nous intéressent. On peut par exemple en déduire à quel point les buissons et les arbres sont connectés entre eux et en déduire le chemin emprunté par les chèvres. On peut également connaître le pourcentage d'herbe dans une partie de la garrigue, ou autrement dit la part de gueuleton potentiel pour une chèvre. Tout ça automatiquement, sur l'ensemble du territoire occupé par Jean-René.

\textbf{Une approche qui a encore de l'avenir devant elle}

Cette méthode ne sert pas qu'à Jean-René, elle est également utile aux gestionnaires des espaces naturels, comme le CEN-LR (Centre des Espaces Naturels - Languedoc-Roussillon). Ils sont friands d'outils qui leur permettent d'avoir une vision globale du territoire. Ainsi, l'analyse des motifs dans une image satellitaire permet de connaître le taux de fermeture d'une garrigue. \emph{A fortiori}, on peut suivre l'évolution de la fermeture des milieux avec plusieurs images acquises à des dates différentes. Cette approche synthétique du territoire n'en est qu'à ses débuts et pourrait permettre de comprendre le comportement de certains animaux : quel type de paysage l'Outarde canepetière préfère-t-elle pour procréer [5] ? Elle permettrait également d'étudier l'impact de mesures de gestion : faut-il brûler les broussailles pour maintenir les garrigues ouvertes ? Ou bien faut-il les girobroyer ? Quelle sont les pratiques les plus efficaces à longs termes ? Celles qui respectent le mieux l'environnement ?

La préservation des milieux ouverts méditerranéens comme la garrigue est de fait intimement liée aux interventions humaines (agriculture, pastoralisme, débroussaillage, etc.) si bien que l’avenir de la diversité biologique dans ces milieux ne peut être déconnecté de celui des activités humaines [5,6]. Encore faut-il bien comprendre lesquelles sont à favoriser et lesquelles sont à encadrer.

\hrulefill

[1] Sirami, C., Nespoulous, A., Cheylan, J.-P., Marty, P., Hvenegaard, G.T., Geniez, P., Schatz, B., Martin, J.-L., 2010. Long-term anthropogenic and ecological dynamics of a Mediterranean landscape: Impacts on multiple taxa. Landscape and Urban Planning 96, 214–223.

[2] \href{https://fr.wikipedia.org/wiki/Photographie_aérienne}{https://fr.wikipedia.org/wiki/Photographie\_aérienne}

[3] Couteron, P., Barbier, N., Gautier, D., 2006. Textural ordination based on Fourier spectral decomposition: a method to analyze and compare landscape patterns. Landscape Ecology 21, 555–567.

[4] Olivier Regniers. Méthodes d'analyse de texture pour la cartographie d'occupations du sol par télédetection très haute résolution : application à la forêt, la vigne et les parcs ostréicoles. Traitement du signal et de l'image. Université de Bordeaux, 2014. Français.

[5] Pointereau, P., Doxa, A., Coulon, F., Jiguet, F., Paracchini, M.L., 2010. Analysis of spatial and temporal variations of High Nature Value farmland and links with changes in bird populations: a study on France. Publications Office.

[6] Médail, F., Diadema, K., 2006. Biodiversité végétale méditerranéennee et anthropisation : approches macro et micro-régionales. Annales de géographie 651, 618.

	%\input{./appendix/diachronique}
	%\input{./appendix/chapitre-qgis}
	%\input{./appendix/method_ecology}

%\end{appendix}

% Résumé
\pagestyle{abstract}
\backmatter % peut-être pas indispensable
%\begin{center}
%\large\textbf{ABSTRACT}
%\end{center}
%\vspace*{1cm}
%\chapter*{\centering \underline{\large\textbf{ABSTRACT}}}
%\addcontentsline{toc}{chapter}{\bf ~~~~~~~~~~ABSTRACT}
\phantomsection
%\addcontentsline{toc}{section}{RÉSUMÉ}
\small
\newgeometry{left=2cm,right=2cm, top=1.7cm, bottom=2cm}
{\centerline {{\sffamily \Large RÉSUMÉ}}}

\noindent\textbf{Caractérisation de l’hétérogénéité spatiale de milieux naturels à partir d’imagerie optique très haute résolution spatiale : cas d’application aux milieux méditerranéens de garrigue.}

\noindent La préservation de la biodiversité est un enjeu prioritaire, identifié aussi bien au niveau national qu’au niveau européen et international. L’hétérogénéité spatiale des milieux naturels est l’une des composantes clefs pour l’étude de la biodiversité et permet de comprendre le fonctionnement des écosystèmes. Le bassin Méditerranéen est un \emph{hotspot} de biodiversité pour lequel le lien entre biodiversité et hétérogénéité spatiale des paysages s’illustre particulièrement bien. Les milieux méditerranéens s'organisent en mosaïques hétérogènes de quatre strates verticales~: le sol nu, l’herbe, les ligneux bas et les ligneux hauts. La biodiversité de ces milieux est aujourd'hui menacée par une fermeture de milieux naturels qui entraîne la disparition de certains habitats et l’homogénéisation des paysages, homogénéisation qui entraîne elle-même une augmentation des risques d’incendies.

\noindent Cette thèse se propose de développer des indices caractérisant l’hétérogénéité spatiale des milieux naturels dans un contexte méditerranéen à partir d’images de télédétection à très haute résolution. Parmi les différentes méthodes permettant de caractériser l'hétérogénéit

\noindent\textbf{Mots clefs~:}{ \it Hétérogénéité spatiale, Télédétection, Texture, Très haute résolution spatiale, Biodiversité, Paysages méditerranéens, Conservation de l’avifaune}.
\vskip 0.5cm
\noindent

{\centerline {{\sffamily \Large ABSTRACT}}}

\noindent\textbf{Characterization of the spatial heterogeneity of natural environments from very high spatial resolution optical imagery : an application case to garrigue Mediterranean habitats.}

\noindent The preservation of biodiversity is a priority issue, both at national, European and international levels. In order to provide a better understanding of ecosystem functioning, spatial heterogeneity of natural environments is becoming one of the key components for the study of biodiversity. The Mediterranean basin is a \emph{hotspot} of biodiversity for which the synergies between biodiversity and spatial heterogeneity of landscapes are particularly important. Mediterranean environments are organized into heterogeneous mosaics of four vertical strata~: bare soil, herbs, low ligneous and high ligneous. The biodiversity of these unique hotspots is now threatened by a closure of the landscape that leads to the habitat loss and landscape homogenization. The loss of heterogeneity is also leading to an increase in fire risks.

\noindent This thesis aims to develop indices characterizing the spatial heterogeneity of natural landscapes in a Mediterranean context using very high spatial resolution remote sensing images. Among the various methods dedicated to the characterization of heterogeneity, the FOTO (FOurier Based Textural Ordination) method is particularly relevant because it produces uncorrelated texture gradients in an unsupervised manner, allowing continuous variations in spatial heterogeneity to be characterized at different spatial scales. Thus, the first objective of this thesis is to test the potential of texture indices derived from the FOTO method for the characterization of spatial heterogeneity relative to four vertical strata. The second objective is to test the sensitivity of our approach to technical and environmental 

\noindent\textbf{Key words~:}{ \it Spatial heterogeneity, Remote sensing, Texture, Very high spatial resolution, biodiversity, Mediterranean landscape, Birds conservation}.
%\normalsize
%\restoregeometry
%\vskip 0.5cm
%\noindent


\end{document}


% Rappels positionnement de figures et tables
% h= here mas o menos
% t = top of the page
% b = bottom of the page
% p = Put on a special page for floats only
% ! Override internal parameters LaTeX uses for determining "good" float positions
% H = Places the float at precisely the location in the LaTeX code. Requires the float package. This is somewhat equivalent to "h!"

% Structure image multiple
%\begin{figure}[h]
%\caption{Caption for this figure with two images}
%\label{fig:image2}
%    \begin{subfigure}{0.5\textwidth}
%        \includegraphics[width=0.9\linewidth, height=5cm]{lion-logo} 
%        \caption{Caption1}
%        \label{fig:subim1}
%    \end{subfigure}
%    \begin{subfigure}{0.5\textwidth}
%        \includegraphics[width=0.9\linewidth, height=5cm]{mesh}
%        \caption{Caption 2}
%        \label{fig:subim2}
%    \end{subfigure}
%\end{figure}


% Détails tableaux
% \hline = rajouter une ligne horitontale
% {l c r} = gauche, centré, droite
% { | c | c | c |} = trois colonnes centrées avec barres verticales