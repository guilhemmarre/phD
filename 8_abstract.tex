%\begin{center}
%\large\textbf{ABSTRACT}
%\end{center}
%\vspace*{1cm}
%\chapter*{\centering \underline{\large\textbf{ABSTRACT}}}
%\addcontentsline{toc}{chapter}{\bf ~~~~~~~~~~ABSTRACT}
\phantomsection
%\addcontentsline{toc}{section}{RÉSUMÉ}
\small
\newgeometry{left=2cm,right=2cm, top=1.7cm, bottom=2cm}
{\centerline {{\sffamily \Large RÉSUMÉ}}}

\noindent\textbf{Développement de la photogrammétrie et d’analyses d’images pour l’étude et le suivi d’habitats marins.}

\noindent Dans un contexte de changement climatique et d’érosion de la biodiversité marine, la surveillance écologique des habitats marins les plus sensibles est primordiale et nécessite des méthodes opérationnelles de suivi permettant aux décideurs et gestionnaires d’établir des mesures de conservation pertinentes et d’évaluer leur efficacité. TEMPO et RECOR sont deux réseaux de surveillance centrés sur les herbiers de posidonie et les récifs coralligènes, les deux habitats les plus riches et sensibles de Méditerranée. L’objectif de cette thèse est de répondre aux besoins de la surveillance des habitats marins par le développement de méthodes d’évaluation de leur état de santé, basées sur deux techniques d’analyses d’images clés : les réseaux de neurones convolutifs et la photogrammétrie. Les résultats montrent que les réseaux de neurones convolutifs sont capables de reconnaître les principales espèces des assemblages coralligènes sur des photos sous-marines issues de RECOR, avec une précision semblable à celle d’un expert taxonomiste. Par ailleurs, nous avons montré que la photogrammétrie permettait de reproduire en 3D un habitat marin avec une grande précision, suffisante pour un suivi de la structure de l’habitat et de la distribution d’espèces à fine échelle. À partir de ces reconstructions, nous avons mis au point une méthode de cartographie automatique des herbiers de posidonie, permettant de réaliser un suivi temporel de la qualité écologique de cet habitat sensible. Enfin, nous avons caractérisé la structure 3D des récifs coralligènes à partir de leurs reconstructions photogrammétriques et étudié les liens avec la structuration des assemblages qui les composent. Ce travail de thèse a permis de développer des méthodes opérationnelles, aujourd’hui intégrées aux réseaux de surveillance TEMPO et RECOR, et ouvre la voie à de futures recherches, notamment la caractérisation de l’activité biologique des récifs coralligènes grâce au couplage entre photogrammétrie, réseaux de neurones et acoustique sous-marine.

\noindent\textbf{Mots clefs~: }{habitats marins, reconstructions 3D, reconnaissance d’images, cartographie, qualité écologique, suivis}.
\vskip 0.2cm
\noindent

{\centerline {{\sffamily \Large ABSTRACT}}}

\noindent\textbf{Underwater photogrammetry and image processing based methods for the monitoring of marine habitats.}

\noindent In a context of climate change and the erosion of marine biodiversity, ecological monitoring of the most sensitive marine habitats is of paramount importance. In particular, there is a need for operational methods that enable decision-makers and managers to establish relevant conservation measures and to evaluate their effectiveness. TEMPO and RECOR are two monitoring networks focusing on Posidonia meadows and coralligenous reefs, the two richest and most sensitive habitats in the Mediterranean. The objective of this thesis is to meet the needs of effective monitoring of marine habitats by developing methods for assessing their health, based on two key image analysis methods: convolutional neural networks and photogrammetry. The results show that convolutional neural networks are capable of recognizing the main species of coralligenous assemblages in underwater photographs from RECOR, with a precision similar to that of an expert taxonomist. Furthermore, we have shown that photogrammetry can reproduce a marine habitat in three dimensions with a high degree of accuracy, sufficient for monitoring habitat structure and species distribution at a fine scale. Based on these reconstructions, we have developed a method for automatic mapping of Posidonia meadows, enabling temporal monitoring of the ecological quality of this sensitive habitat. Finally, we characterized the three-dimensional structure of coralligenous reefs based on their photogrammetric reconstructions and studied the links with the structuring of the assemblages that make them up. This PhD work has led to the development of operational methods that are now integrated into the TEMPO and RECOR monitoring networks. Results of this work paves the way for future research, in particular concerning characterization of the biological activity of coralligenous reefs thanks to the coupling of photogrammetry, neural networks and underwater acoustics.

\noindent\textbf{Key words~: }{marine habitats, 3D reconstructions, image recognition, mapping, ecological status, monitoring}.
