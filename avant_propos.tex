\addcontentsline{toc}{section}{\bf AVANT-PROPOS}

\huge
\noindent{Avant-propos}

\bigskip

\normalsize
Cette thèse à été réalisée de septembre 2017 à juillet 2020 entre Andromède Océanologie et l'\acrshort{umr} \acrshort{tetis} (\acrshort{inrae}, \acrshort{cirad}, \acrshort{cnrs}, AgroParisTech) à Montpellier, en collaboration avec l'UMR MARBEC (Université de Montpellier, \acrshort{cnrs}, \acrshort{ird}, Ifremer, Montpellier). Ce travail a été cofinancé par Andromède Océanologie et l'\gls{anrt}, et les missions de terrain ont eu lieu dans le cadre des suivis environnementaux \acrshort{TEMPO} et \acrshort{recor} financés par l'\gls{aermc}.

\medskip

Cette thèse a été dirigée par Julie Deter et Sandra Luque, et a bénéficié des conseils avisés de Dino Ienco sur les aspects deep learning ainsi que des membres du comité de thèse Nicolas Mouquet, Gérard Subsol, François Guilhaumon, Maria Dornelas et Christiane Weber. 

%%% ARTICLES
\myparagraph{Articles publiés dans des revues à comité de lecture}

% 1er article méthodo
\noindent{\href{https://doi.org/10.3389/fmars.2019.00276}{\textbf{Marre, G.}, Holon, F., Luque, S., Boissey, P., Deter, J., 2019. Monitoring marine habitats with photogrammetry: a cost-effective, accurate, precise and high-resolution reconstruction method. Frontiers in Marine Science 6:276, 158–170}.}

\medskip

% 2ème article herbiers
\noindent{\textbf{Marre G.}, Deter, J., Holon, F., Boissery, P., Luque, S. Fine-scale automatic mapping of living Posidonia oceanica seagrass beds with underwater photogrammetry. Marine Ecology Progress Series - \textit{Accepté sous conditions de modifications mineures.}}

\medskip

% 3ème article 
\noindent{\textbf{Marre G.}, De Almeida Braga, C., Ienco, D., Luque, S., Holon, F., Deter, J. Deep convolutional neural networks to monitor coralligenous reefs: operationalizing biodiversity and ecological assessment. Ecological Informatics - \textit{En cours de révision.}}




%%% AUTRES ARTICLES
\myparagraph{Autres publications}

\noindent{\href{https://doi.org/10.3389/fmars.2019.00276}{Holon, F., \textbf{Marre, G.}, Parravicini, V., Mouquet, N., Bockel, T., Descamp, P., Tribot, A-S., Boissery, P., Deter, J. A predictive model based on multiple coastal anthropogenic pressures explains the degradation status of a marine ecosystem: Implications for management and conservation. Biological Conservation 222, 125-135}}


%%% RAPPORTS
\myparagraph{Rapports d'activité}

\noindent{Holon, F., \textbf{Marre, G.}, Delaruelle, G., Holon, F., Guilbert, A., Deter, J., 2018. Application de la photogrammétrie à la surveillance biologique: mise au point de la méthode. Rapport d'activité pour l'\acrlong{aermc}, 57 pages.}

\noindent{Holon, F., \textbf{Marre, G.}, Delaruelle, Fery, C., G., Holon, F., Guilbert, A., Deter, J., 2018. Acquisitions photogrammétriques 2018 - 2019 et développements méthodologiques. Rapport d'activité pour l'\acrlong{aermc}, 145 pages.}

%%% VULGARISATION
\myparagraph{Articles de vulgarisation scientifique}

\noindent{\href{https://medtrix.fr/cahier-de-surveillance-3/}{\textbf{Marre, G.}, Luque, S., Holon, F., Boissery, P., Deter, J., 2018. Application de la photogrammétrie à la surveillance biologique des habitats sous-marins. Cahiers de la surveillance Medtrix n°3, Mars 2018.}}

\medskip

\noindent{\href{https://www.agropolis.fr/publications/sciences-marines-et-littorales-en-occitanie-dossier-thematique-agropolis-international.php}{\textbf{Marre, G.}, Luque, S., Holon, F., Boissery, P., Deter, J., 2019. La photogrammétrie : une méthode d’observation innovante pour l’étude et la conservation du milieu marin. Les dossiers d'Agropolis International N°24: Sciences marines et littorales en Occitanie, Février 2019.}}



%%% PRESENTATIONS ORALES
\myparagraph{Communications dans des congrès}

\noindent{\textbf{Marre, G.}, Ropars, B., 2017. At the crossroads between robotics, photogrammetry and ecology for the study of marine habitats. Communication orale au congrès Ecolotech. Salon de l'Ecologie, Montpellier, 9 novembre 2017.}

\medskip

\noindent{\textbf{Marre, G.}, Luque, S., Holon, F., Boissery, P., Deter, J., 2018. Rencontre entre robotique, photogrammétrie et écologie pour l'étude et le suivi des fonds marins. Communication orale au congrès Medtrix. Montpellier, 14 mars 2018.}

\medskip

\noindent{\textbf{Marre, G.}, Luque, S., Holon, F., Boissery, P., Deter, J., 2018. Développement de la photogrammétrie pour 
l’étude et le suivi d’habitats marins. Communication orale à l'Apéro Technique de l'\acrshort{osu} \acrshort{oreme}. Montpellier, 28 mai 2018.}

\medskip

\noindent{\textbf{Marre, G.}, Luque, S., Holon, F., Boissery, P., Deter, J., 2019. Méthodes photographiques pour la caractérisation de la structure et de la biodiversité des récifs coralligènes. Communication orale aux 10 ans de l'\acrshort{osu} \acrshort{oreme}. Montpellier, 11 octobre 2019.}

\medskip

\noindent{\textbf{Marre, G.}, Luque, S., Holon, F., Boissery, P., Deter, J., 2019. Suivi de communautés coralligènes par des
réseaux de neurones convolutifs. Communication orale au congrès Ecolotech. Salon de l'Ecologie, Montpellier, 7 novembre 2019.}

\medskip

\noindent{\textbf{Marre, G.}, Luque, S., Holon, F., Boissery, P., Deter, J., 2019. Photogrammétrie sous-marine et analyses d’images pour le suivi d’habitats benthiques Méditerranéens. Communication orale au congrès Merigeo 2020. Bordeaux, 16 mars 2020.}



%%% ENCADREMENT
\myparagraph{Activités d'encadrement}

\noindent{Co-encadrement du stage de fin d'études (M2 - cursus ingénieur agronome) de \textbf{Gaïlé Lejay} en 2018, ayant donné lieu à la rédaction d'un mémoire de fin d'études : "Caractérisation fonctionnelle des enherbements viticoles : services potentiels et relations traits-services".}

\medskip

\noindent{Co-encadrement du stage de fin d'études (M2 - Computer Science) de \textbf{Cédric De Almeida Braga} en 2019, ayant donné lieu à la rédaction d'un mémoire de fin d'études : "Characterization of coralligenous assemblages : from automatic
image classification to 3D species mapping".}
