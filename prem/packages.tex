
% Packages basiques pour la mise en page globale
\usepackage[utf8]{inputenc}
\usepackage[T1]{fontenc}
\usepackage[left=2.9cm, % marge gauche
			right=2.9cm, % marge droite
			top=2cm, % marge haut 1.3-1.5 mieux mais il faut ajuster les en-têtes dans ce cas
			bottom=2.3cm, % marge bas
			includefoot,
			]{geometry}

% math stuff
\usepackage{amsmath, amssymb, amsthm} % for equation
%\usepackage{latexsym} % provide additional symbols
%\usepackage{mathrsfs} % provide stuff for math

% for graphics and figures
\usepackage{graphicx,color} % graphics and define new colors
\usepackage{tikz} % to draw graph
\usetikzlibrary{calc}
\newcommand*{\bord}{1mm} % pour les bords arrondis des images
\usepackage{adjustbox} % to place figure within, and control size and position
\usepackage{caption}
\usepackage[tikz]{bclogo}
\graphicspath{ {images/}}
\usepackage{wrapfig}
%\usepackage{epsfig} % old package for deal with figures
%\usepackage{pgfplots} % to draw plot
%\usepackage{xcolor} % more advanced management of color

% for table
\usepackage{multirow}
\usepackage{longtable}
\usepackage{booktabs}
\usepackage{makecell}
\renewcommand\cellgape{\Gape[6pt]} % interligne entre cellules d'un tableau
%\usepackage{diagbox} % to draw diag separation in table

% space and positioning management
\usepackage{float}
%\usepackage{parskip} % paragraph positioning management (small-med-bigskip ne marchent plus si actif)
\usepackage{setspace} % positioning environnement
\usepackage{soul} % for hyphenation
\usepackage{calc} % to set lenght, counter
\usepackage{fancyhdr} % to custom header position
\usepackage[bottom]{footmisc} % so that footnotes always are in in the bottom


% font and symbols management
\usepackage{gensymb} % for degree symbol (among others)
\usepackage{libertine} %use libertine font
\usepackage{textcase} %Upper or lower case commands
\usepackage{sectsty} % custom sections header
\usepackage[nobottomtitles*]{titlesec} % custom sections header / option nobottomtitles to avoid section titles at the bottom of the page
%\usepackage[explicit]{titlesec}

\usepackage{titletoc} % handle section tilte in toc (add to content etc.)
\usepackage[normalem]{ulem} % to strike out text, and highlight etc.
%\usepackage{times}

% pdf management
\usepackage{pdfpages} % to include pdf
\usepackage{pdflscape} %to provide landscape mode in pdf render

\usepackage[titletoc]{appendix}
\usepackage[titles]{tocloft} % list of figures, table ...

\usepackage{ifthen} % if then condition
\usepackage{enumitem} % to control item indent

%\usepackage{csquotes} % gestion des citations
%\usepackage{alltt} % custom verbatim environnement
%\usepackage{acronym}


\usepackage[unicode=true,
bookmarksopen={true},
pdffitwindow=true,
colorlinks=true,
linkcolor=bluecite, % couleur des liens internes au texte
citecolor=bluecite, % couleur références dans le texte
urlcolor=black, % couleur du texte avec lien URL
hyperfootnotes=true,
pdfstartview={FitH},
naturalnames=false,
pdfpagemode= UseNone,
backref=page,
breaklinks=true]{hyperref}

\usepackage{bookmark} % to manage bookmarks (levels etc.)

%\xapptocmd{\appendices}{%
%  \renewcommand{\theHchapter}{\appendixname.\thechapter}
%}{}{}

\newcommand{\myparagraph}[1]{\paragraph{#1}\mbox{}\\} %titre de paragraphe


% ---------------------------------------------------------

\newtheorem{example}{Example}
\newcommand{\bexample}{\begin{example}}
	\newcommand{\eexample}{\end{example}}
\newtheorem{remark}{Remarque}
\newcommand{\bremark}{\begin{remark}}
	\newcommand{\eremark}{\end{remark}}
\newtheorem{theorem}{Theorem}
\newcommand{\btheorem}{\begin{theorem}}
	\newcommand{\etheorem}{\end{theorem}}

%\newtheorem{encadre}{Encadre}
\newcounter{encadrecompteur}
%\setcounter{encadre}{1}


\newenvironment{encadre}{
	\refstepcounter{encadrecompteur}
}

\renewcommand{\theencadrecompteur}{\thechapter.\arabic{encadrecompteur}}
\newcommand{\labelencadre}{Encadré~\theencadrecompteur }


%%%%%%%%%%%%%%%% Définition des couleurs utilisées
\definecolor{butter1}{rgb}{0.988,0.914,0.310}
\definecolor{chocolate1}{rgb}{0.914,0.725,0.431}
\definecolor{skyblue1}{rgb}{0.447,0.624,0.812}
\definecolor{plum1}{rgb}{0.678,0.498,0.659}
\definecolor{scarletred1}{rgb}{0.937,0.161,0.161}
\definecolor{tete}{RGB}{151, 143, 154}
\definecolor{bluecite}{HTML}{0875b7}
\definecolor{chameleon1}{RGB}{120, 179, 31}
\definecolor{chameleon2}{RGB}{137, 179, 72}
\definecolor{chameleon3}{RGB}{184, 209, 145}
\definecolor{blueblob}{RGB}{153, 0, 0}%{51, 153, 255}
%\definecolor{vertsection}{RGB}{32,54,28}
\definecolor{bleusection}{RGB}{8,61,99}
%\definecolor{vertsubsection}{RGB}{64,108,57}
\definecolor{bleusubsection}{RGB}{59,125,205}
\definecolor{resume}{RGB}{202, 229, 237}
\definecolor{textresume}{RGB}{21, 34, 17}
%173, 235, 173
\definecolor{encadre}{RGB}{175, 207, 175}%166, 217, 166

%%%%%%%%%%%%%%%%% Pour la rédaction uniquement
% Couleur rouge pour les parties affichées comme à faire (TODO)
\newcommand{\TODO}[1]{(\textcolor{red}{TODO : #1})}

\newcommand{\pp}{\textperiodcentered}
\newcommand{\comment}[1]{}

% Marcro to display l'entête des paragraphe: pour la rédaction seulement, ensuite fixer à false
\newboolean{displaytete}
\setboolean{displaytete}{false}
\newcommand{\tete}[1]{%
	\ifthenelse{%
		\boolean{displaytete}}{% cond
		{\small\emph{\textcolor{tete}{(Idée : #1)}\\}}}{% true
		}%false
}
\setlength{\fboxrule}{0pt}
\setlength{\fboxsep}{10pt}



%%%%%%%%%%%%%%%% define colorbox
\newsavebox{\selvestebox}
\newenvironment{colbox}[1]
  {\newcommand\colboxcolor{#1}%
   \begin{lrbox}{\selvestebox}%
	 \begin{minipage}{0.95\textwidth-2\fboxsep\relax}}
  {\end{minipage}\end{lrbox}%
	%\begin{center}
  \colorbox{\colboxcolor}{\framebox{\usebox{\selvestebox}}}%
	}
	 %\end{center}}
% define la loupe des ancadre
\newcommand\monimage{\includegraphics[width =17pt]{./misc/loupe}}
\newcommand\mabarre{\includegraphics{barre.png}}

\renewcommand\bcStyleTitre[1]{\bfseries{#1}}

	 %\begin{minipage}{\dimexpr\columnwidth-2\fboxsep\relax}
	 
	 
	 
%%%%%%%%%%%%%%%% Style et espacements des sections

% Package de style de chapitres avec son template "Sonny"
%\usepackage[Sonny]{fncychap} % http://www.nagel-net.de/Latex/DOKU/fncychap.pdf
%\titleformat{\chapter}[hang]{\fillast}{\bfseries\MakeUppercase{\chaptertitlename\space\thechapter : #1}}{1ex minus .1ex}{\bfseries\uppercase}
%\titleformat{\chapter}[hang]{\fillast}{\bfseries{\chaptertitlename\space\thechapter :}}{1ex minus .1ex}{\bfseries}

% Alternative de titre mais qui fout le bordel
%\newpagestyle{headplain}{%
%\sethead[\thepage][][]{}{}{\thepage}}
%\titleformat{\chapter}
%[block]
%{\normalfont\huge\bfseries}
%{\thechapter}
%{0pt} %
%{\hspace{20pt}\huge}
%\titlespacing*{\chapter}{0pt}{0pt}{30pt}





%\allsectionsfont{\MakeTextUppercase}

% Noms des variables utilisées dans les bandeaux et tableaux
\renewcommand{\&}{and}
\renewcommand{\cftpartfont}{\sffamily\Large}
\renewcommand{\partname}{PARTIE}
\renewcommand{\chaptername}{Chapitre} % Affichage en en tête des pages du style "main"
% https://latex.org/forum/viewtopic.php?t=1758
\newcommand{\sectionname}{Section} % Section name jamais défini donc c'est un newcommand seulement
\renewcommand{\appendixname}{Annexe} % nom de l'annexe "Annexe A",...
%\renewcommand{\appendixtocname}{Annexes}
%\renewcommand{\appendixpagename}{Annexes}
%\renewcommand{\contentsname}{\small TABLE DES MATIÈRES} % défini plus bas
\renewcommand{\listfigurename}{\Large LISTE DES FIGURES}
\renewcommand{\listtablename}{\Large LISTE DES TABLEAUX}
\sectionfont{\normalfont}

\titleformat{\part}[display]
{\sffamily \centering \huge \titlerule[1pt]\vspace{8pt}}%format
{\partname~\thepart}%
{5pt}%
{}%
[\vspace{12pt}\titlerule]


%%%%%%%%%%%%%%%%%%%%%%%%%%% Style des titres
% Titre de section
\titleformat{\section}
{\color{bleusection}\raggedright\normalfont\fontsize{18pt}{22pt}\selectfont}{\thesection}{1em}{}
\titlespacing {\section} {0pt} {12pt} {10pt}
% {left spacing}{before spacing}{after spacing}[right]
% spacing: how to read {12pt plus 4pt minus 2pt}
%           12pt is what we would like the spacing to be
%           plus 4pt means that TeX can stretch it by at most 4pt
%           minus 2pt means that TeX can shrink it by at most 2pt
%\titlespacing*{\section} {0pt}{0pt}{2.3ex plus .2ex}

% Titre de sous-section
\titleformat{\subsection}
{\color{bleusubsection}\normalfont\fontsize{16pt}{18pt}\selectfont}{\thesubsection}{1em}{}
\titlespacing {\subsection} {0pt} {12pt} {6pt}
%\titlespacing*{\subsection}{0pt}{0pt}{1.5ex plus .2ex}

% Titre de sous-sous-section
\titleformat{\subsubsection}
{\color{bleusection}\normalfont\bfseries\fontsize{12pt}{14pt}\selectfont}{\thesubsubsection}{1em}{}
\titlespacing {\subsubsection} {0pt} {12pt} {6pt}

% definition d'un nouveau niveau pour le paragraphe
% \makeatletter
% \renewcommand{\paragraph}{\@startsection{paragraph}{4}{2ex}%
%     {-3.25ex plus -1ex minus -0.2ex}%
%     {1.5ex plus 0.2ex}%
%     {\color{bleusection}\normalfont\normalsize}}
% \makeatother
% \renewcommand\theparagraph{\alph{paragraph}.}
% \titlespacing {\paragraph} {0pt} {12pt} {6pt}


% Indentation générale
% https://www.overleaf.com/learn/latex/Paragraph_formatting
\setlength{\parskip}{0.2cm plus1mm} %  
\setlength{\parindent}{0.5cm} % indentation pour un nouveau paragraphe
\setlength{\itemsep}{3pt}
\raggedbottom % controle la manière dont est répartie le texte dans une page (aussi \flushbottom)
% https://www.giss.nasa.gov/tools/latex/ltx-299.html
\renewcommand{\baselinestretch}{1.25} % Définition de l'interligne

%%%%%%%%%%%%%%%%%%%%%%%%%%% Chapter Header STYLE
\setlength{\unitlength}{1mm} % defines the unit used in the picture environment

%The green square
%\newcommand{\blob}{\textcolor{bleusubsection}{\rule[-10pt]{4cm}{1cm}}} % [offset vertical]{largeur}{hauteur}
%\newcommand{\blobblue}{\textcolor{blueblob}{\rule[-23pt]{2.2cm}{2.2cm}}} % [offset vertical]{largeur}{hauteur}

% The upper right mark
%\newcommand{\rblob}[1]{ #1
%\begin{picture}(0,0)(5,0)
%\put(0,0){\blob}
%\end{picture}}

%\newcommand{\rblobblue}[1]{ #1
%\begin{picture}(0,0)(-6.5,0)
%\put(0,0){\blobblue}
%\end{picture}}

% The upper left mark
%\newcommand{\lblob}[1]{
%\begin{picture}(0,0)(28.5,0)
%\put(0,0){\blob}
%\end{picture}%
%#1
%}


%%%%%%%%%%%%%%%%% Haut de page
% Chapitre courant pour l'introduction
% RO
\usepackage{tcolorbox}
\newcommand\rbluebandintro{ 
\begin{picture}(0,0)(5,3) % (width,height)(x-offset,y-offset)
    \begin{tcolorbox}[width=40mm, colframe=bleusubsection, colback=bleusubsection, boxsep=2mm, arc=4mm] % arc = argument de courbure, boxsep=hauteur
    \end{tcolorbox}
    \begin{picture}(0,0)(36,-3)
        \textcolor{white}{\fontsize{14}{14}{\selectfont Introduction}}
    \end{picture} 
\end{picture}   
}

% LE
\usepackage{tcolorbox}
\newcommand\lbluebandintro{ 
\begin{picture}(0,0)(35,3) % (width,height)(x-offset,y-offset)
    \begin{tcolorbox}[width=40mm, colframe=bleusubsection, colback=bleusubsection, boxsep=2mm, arc=4mm]
    \end{tcolorbox}
    \begin{picture}(0,0)(30,-3)
        \textcolor{white}{\fontsize{14}{14}{\selectfont Introduction}}
    \end{picture}  
\end{picture}   
}


% Chapitre courant pour la Méthodologie
% RO
\usepackage{tcolorbox}
\newcommand\rbluebandmethodes{ 
\begin{picture}(0,0)(5,3) % (width,height)(x-offset,y-offset)
    \begin{tcolorbox}[width=40mm, colframe=bleusubsection, colback=bleusubsection, boxsep=2mm, arc=4mm] % arc = argument de courbure, boxsep=hauteur
    \end{tcolorbox}
    \begin{picture}(0,0)(36,-3)
        \textcolor{white}{\fontsize{14}{14}{\selectfont Méthodologie}}
    \end{picture} 
\end{picture}   
}

% LE
\usepackage{tcolorbox}
\newcommand\lbluebandmethodes{ 
\begin{picture}(0,0)(35,3) % (width,height)(x-offset,y-offset)
    \begin{tcolorbox}[width=40mm, colframe=bleusubsection, colback=bleusubsection, boxsep=2mm, arc=4mm]
    \end{tcolorbox}
    \begin{picture}(0,0)(30,-3)
        \textcolor{white}{\fontsize{14}{14}{\selectfont Méthodologie}}
    \end{picture}  
\end{picture}   
}


% Chapitre courant pour la discussion
% RO
\usepackage{tcolorbox}
\newcommand\rbluebanddiscussion{ 
\begin{picture}(0,0)(5,3) % (width,height)(x-offset,y-offset)
    \begin{tcolorbox}[width=40mm, colframe=bleusubsection, colback=bleusubsection, boxsep=2mm, arc=4mm] % arc = argument de courbure, boxsep=hauteur
    \end{tcolorbox}
    \begin{picture}(0,0)(35,-3)
        \textcolor{white}{\fontsize{14}{14}{\selectfont Discussion}}
    \end{picture} 
\end{picture}   
}

% LE
\usepackage{tcolorbox}
\newcommand\lbluebanddiscussion{ 
\begin{picture}(0,0)(35,3) % (width,height)(x-offset,y-offset)
    \begin{tcolorbox}[width=40mm, colframe=bleusubsection, colback=bleusubsection, boxsep=2mm, arc=4mm]
    \end{tcolorbox}
    \begin{picture}(0,0)(27,-3)
        \textcolor{white}{\fontsize{14}{14}{\selectfont Discussion}}
    \end{picture}  
\end{picture}   
}


% Chapitre courant pour les annexes
% RO
\usepackage{tcolorbox}
\newcommand\rbluebandappendix{ 
\begin{picture}(0,0)(5,3) % (width,height)(x-offset,y-offset)
    \begin{tcolorbox}[width=40mm, colframe=bleusubsection, colback=bleusubsection, boxsep=2mm, arc=4mm] % arc = argument de courbure, boxsep=hauteur
    \end{tcolorbox}
    \begin{picture}(0,0)(35,-3)
        \textcolor{white}{\fontsize{14}{14}{\selectfont Annexes}}
    \end{picture} 
\end{picture}   
}

% LE
\usepackage{tcolorbox}
\newcommand\lbluebandappendix{ 
\begin{picture}(0,0)(35,3) % (width,height)(x-offset,y-offset)
    \begin{tcolorbox}[width=40mm, colframe=bleusubsection, colback=bleusubsection, boxsep=2mm, arc=4mm]
    \end{tcolorbox}
    \begin{picture}(0,0)(27,-3)
        \textcolor{white}{\fontsize{14}{14}{\selectfont Annexes}}
    \end{picture}  
\end{picture}   
}


% Chapitre courant pour les autres chapitres
% RO
\usepackage{tcolorbox}
\newcommand\rblueband{ 
\begin{picture}(0,0)(5,3) % (width,height)(x-offset,y-offset)
    \begin{tcolorbox}[width=40mm, colframe=bleusubsection, colback=bleusubsection, boxsep=2mm, arc=4mm] % arc = argument de courbure, boxsep=hauteur
    \end{tcolorbox}
        \begin{picture}(0,0)(35,-3)
        \textcolor{white}{\fontsize{14}{14}\selectfont\chaptername~\thechapter}
    \end{picture} 
\end{picture}   
}

% LE
\usepackage{tcolorbox}
\newcommand\lblueband{ 
\begin{picture}(0,0)(35,3) % (width,height)(x-offset,y-offset)
    \begin{tcolorbox}[width=40mm, colframe=bleusubsection, colback=bleusubsection, boxsep=2mm, arc=4mm]
    \end{tcolorbox}
    \begin{picture}(0,0)(27,-3)
        \textcolor{white}{\fontsize{14}{14}\selectfont\chaptername~\thechapter}
    \end{picture}  
\end{picture}   
}

% Chapitre courant pour les references
% RO
\usepackage{tcolorbox}
\newcommand\rbluebandreferences{ 
\begin{picture}(0,0)(5,3) % (width,height)(x-offset,y-offset)
    \begin{tcolorbox}[width=40mm, colframe=bleusubsection, colback=bleusubsection, boxsep=2mm, arc=4mm] % arc = argument de courbure, boxsep=hauteur
    \end{tcolorbox}
    \begin{picture}(0,0)(35,-3)
        \textcolor{white}{\fontsize{14}{14}{\selectfont Références}}
    \end{picture} 
\end{picture}   
}

% LE
\usepackage{tcolorbox}
\newcommand\lbluebandreferences{ 
\begin{picture}(0,0)(35,3) % (width,height)(x-offset,y-offset)
    \begin{tcolorbox}[width=40mm, colframe=bleusubsection, colback=bleusubsection, boxsep=2mm, arc=4mm]
    \end{tcolorbox}
    \begin{picture}(0,0)(27,-3)
        \textcolor{white}{\fontsize{14}{14}{\selectfont Références}}
    \end{picture}  
\end{picture}   
}




% Section courante
% Pour définir le nom de section courante
% https://tex.stackexchange.com/questions/222370/how-can-i-customize-and-use-the-leftmark-and-rightmark-commands-with-custom-fa
\pagestyle{fancy}
\renewcommand{\sectionmark}[1]{\markright{#1}} % on ne garde que le nom de section et pas avec numéro (Introduction et pas 1.2 Introduction) => pour le running section title header
\renewcommand{\subsectionmark}[1]{}

% RE
\newcommand\rsection{ 
\begin{picture}(0,0)(5,2) % (width,height)(x-offset,y-offset)
    \textcolor{black}{\fontsize{14}{14}\selectfont\sectionname~\thesubsection}
\end{picture}   
}

% LO
\newcommand\lsection{ 
\begin{picture}(0,0)(15,2) % (width,height)(x-offset,y-offset)
    \textcolor{black}{\fontsize{14}{14}\selectfont\sectionname~\thesubsection}
\end{picture}   
}

%%%% Backup qui fonctionne
% RE
%\newcommand\rsection{ 
%\begin{picture}(0,0)(5,2) % (width,height)(x-offset,y-offset)
%    \textcolor{black}{\fontsize{14}{14}\selectfont\sectionname~\thesubsection}
%\end{picture}   
%}

% LO
%\newcommand\lsection{ 
%\begin{picture}(0,0)(15,2) % (width,height)(x-offset,y-offset)
%    \textcolor{black}{\fontsize{14}{14}\selectfont\sectionname~\thesubsection}
%\end{picture}   
%}


%%%%%%%%%%%%%%%%% Bas de page
% % Numérotation de pages
\newcommand\numberbaleine{ 
\begin{picture}(0,0)(5,-2) % (width,height)(x-offset,y-offset)
    \includegraphics[width=1.5cm]{images/misc/numerotation_pages.png}
\end{picture}
\fontsize{12}{12}\selectfont\thepage
}

% Commenté car a priori useless mais tester si jamais (le 25/02 à 19h30)
%\renewcommand{\chaptermark}[1]{%
%\markboth{\MakeUppercase{%
%\chaptername}\ \thechapter.%
%\ #1}{}}




%%%%%%%%%%%%%%%% Style des numéros de sections, figures etc.
\setcounter{secnumdepth}{3} % Profondeur de l'arborescence (nombre de sections / sous sections possibles)

\renewcommand{\theequation}{\thechapter.\arabic{equation}}
\newcommand{\thedefinition}{\thesection.\arabic{definition}}
\renewcommand{\theexample}{\thesection.\arabic{example}}
\renewcommand{\theremark}{\thesection.\arabic{remark}}
\theoremstyle{definition}

\renewcommand{\thechapter}{\arabic{chapter}} % Définition de la variable du numéro de chapitre
\renewcommand{\thesection}{\arabic{section}} % Définition de la variable du numéro de la section
%\renewcommand{\thesubsection}{\arabic{section}.\arabic{subsection}}
%\renewcommand{\thesubsubsection}{\arabic{section}.\arabic{subsection}.\arabic{subsubsection}}
%\setlength{\belowcaptionskip}{-15pt} % espace / marge après les figures => réduit aussi l'espace avant les tables...

%%%%%%%%%%%%%%%% Styles principaux utilisés pour les différentes parties
\pagestyle{fancy}
\newcommand{\myheaderlabel}{}

% Illustration du layout et les différents paramètres:
% https://tex.stackexchange.com/questions/132170/what-do-headheight-headsep-etc-do-in-the-vmargin-package

% Style titre chapitre
\fancypagestyle{titre_chapitre}{
%    \titleformat{\chapter} % \chapter
%    [block]
%    {\normalfont\huge\bfseries} % style du numéro de chapitre
%%    {\thechapter}
%    {0pt} %
%    {\hspace{20pt}\Large} % style du titre de chapitre
%    \titlespacing*{\chapter}{0pt}{0pt}{25pt}
    \titleformat{\chapter}[display]
    {\normalfont\bfseries\color{bleusection}}{}{0pt}{\Huge}
    \titlespacing*{\chapter}{0pt}{-60pt}{20pt}
    % {\normalfont\bfseries}{}{0pt}{\Huge}
	
	\fancyhf{}
	\setlength{\headheight}{20pt plus4mm minus4mm}
	\setlength{\headsep}{42pt} % hauteur entre le header et le début du texte
	\renewcommand{\headrulewidth}{0pt}
%	\fancyhead[RO]{\rbluebandintro}
%	\fancyhead[LE]{\lbluebandintro}
%	\fancyhead[RE]{\rsection}
%	\fancyhead[LO]{\lsection}
    \fancyfoot[C]{\vspace*{0.05cm}\includegraphics[width=12cm]{images/misc/bandeau_bas.png}\\} % Pied de page avec le récif
	\fancyfoot[LE]{\numberbaleine} % numérotation baleine pages paires
	\fancyfoot[RO]{\numberbaleine} % numérotation baleine pages impaires
}


% Style intro
\fancypagestyle{intro}{
	\fancyhf{}
	\setlength{\headheight}{20pt plus4mm minus4mm}
	\setlength{\headsep}{42pt} % hauteur entre le header et le début du texte
	\renewcommand{\headrulewidth}{0pt}
	\fancyhead[RO]{\rbluebandintro}
	\fancyhead[LE]{\lbluebandintro}
	\fancyheadoffset[loh,reh]{10mm} % horizontal offset header
	\fancyhead[RE]{\fontsize{14}{14}{\color{bleusubsection}\selectfont\rightmark}}
	\fancyhead[LO]{\fontsize{14}{14}{\color{bleusubsection}\selectfont\rightmark}}
    \fancyfoot[C]{\vspace*{0.05cm}\includegraphics[width=12cm]{images/misc/bandeau_bas.png}\\} % Pied de page avec le récif
	\fancyfoot[LE]{\numberbaleine} % numérotation baleine pages paires
	\fancyfoot[RO]{\numberbaleine} % numérotation baleine pages impaires
}


% Style méthodologie
\fancypagestyle{methodo}{
	\fancyhf{}
	\setlength{\headheight}{20pt plus4mm minus4mm}
	\setlength{\headsep}{42pt} % hauteur entre le header et le début du texte
	\renewcommand{\headrulewidth}{0pt}
	\fancyhead[RO]{\rbluebandmethodes}
	\fancyhead[LE]{\lbluebandmethodes}
	\fancyheadoffset[loh,reh]{10mm} % horizontal offset header
	\fancyhead[RE]{\fontsize{14}{14}{\color{bleusubsection}\selectfont\rightmark}}
	\fancyhead[LO]{\fontsize{14}{14}{\color{bleusubsection}\selectfont\rightmark}}
    \fancyfoot[C]{\vspace*{0.05cm}\includegraphics[width=12cm]{images/misc/bandeau_bas.png}\\} % Pied de page avec le récif
	\fancyfoot[LE]{\numberbaleine} % numérotation baleine pages paires
	\fancyfoot[RO]{\numberbaleine} % numérotation baleine pages impaires
}


% Style discussion
\fancypagestyle{discussion}{
	\fancyhf{}
	\setlength{\headheight}{20pt plus4mm minus4mm}
	\setlength{\headsep}{42pt} % hauteur entre le header et le début du texte
	\renewcommand{\headrulewidth}{0pt}
	\fancyhead[RO]{\rbluebanddiscussion}
	\fancyhead[LE]{\lbluebanddiscussion}
	\fancyheadoffset[loh,reh]{10mm} % horizontal offset header
	\fancyhead[RE]{\fontsize{14}{14}{\color{bleusubsection}\selectfont\rightmark}}
	\fancyhead[LO]{\fontsize{14}{14}{\color{bleusubsection}\selectfont\rightmark}}
    \fancyfoot[C]{\vspace*{0.05cm}\includegraphics[width=12cm]{images/misc/bandeau_bas.png}\\} % Pied de page avec le récif
	\fancyfoot[LE]{\numberbaleine} % numérotation baleine pages paires
	\fancyfoot[RO]{\numberbaleine} % numérotation baleine pages impaires
}


% Style main
\fancypagestyle{main}{
	\fancyhf{}
	\setlength{\headheight}{20pt plus4mm minus4mm}
	\setlength{\headsep}{42pt} % hauteur entre le header et le début du texte
	%\setlength{\lineskiplimit}{-\maxdimen} % Pour forcer l'espacement régulier des lignes
	\renewcommand{\headrulewidth}{0pt}
	\fancyhead[RO]{\rblueband}
	\fancyhead[LE]{\lblueband}
	\fancyheadoffset[loh,reh]{10mm} % horizontal offset header
	\fancyhead[RE]{\fontsize{14}{14}{\color{bleusubsection}\selectfont\rightmark}}
	\fancyhead[LO]{\fontsize{14}{14}{\color{bleusubsection}\selectfont\rightmark}}
    \fancyfoot[C]{\vspace*{0.05cm}\includegraphics[width=12cm]{images/misc/bandeau_bas.png}\\} % Pied de page avec le récif
	\fancyfoot[LE]{\numberbaleine} % numérotation baleine pages paires
	\fancyfoot[RO]{\numberbaleine} % numérotation baleine pages impaires
}

% Style appendix
\fancypagestyle{appendix}{
	\fancyhf{}
	\setlength{\headheight}{20pt plus4mm minus4mm}
	\setlength{\headsep}{42pt} % hauteur entre le header et le début du texte
	\renewcommand{\headrulewidth}{0pt}
	\fancyhead[RO]{\rbluebandappendix}
	\fancyhead[LE]{\lbluebandappendix}
	\fancyheadoffset[loh,reh]{10mm} % horizontal offset header
	\fancyhead[RE]{\fontsize{14}{14}{\color{bleusubsection}\selectfont\rightmark}}
	\fancyhead[LO]{\fontsize{14}{14}{\color{bleusubsection}\selectfont\rightmark}}
    \fancyfoot[C]{\vspace*{0.05cm}\includegraphics[width=12cm]{images/misc/bandeau_bas.png}\\} % Pied de page avec le récif
	\fancyfoot[LE]{\numberbaleine} % numérotation baleine pages paires
	\fancyfoot[RO]{\numberbaleine} % numérotation baleine pages impaires
}

% Style preambule
\fancypagestyle{preambule}{
	\fancyhf{}
	\setlength{\headheight}{20pt plus4mm minus4mm}
	\setlength{\headsep}{42pt}
	\renewcommand{\headrulewidth}{0pt}
	%\fancyhead[RO]{\rblob{\hphantom{\chaptername~\thechapter}}}
	%\fancyhead[LE]{\lblob{\hphantom{\chaptername~\thechapter}}}
	\fancyfoot[LE]{\numberbaleine}
	\fancyfoot[RO]{\numberbaleine}
	\fancyfoot[C]{\vspace*{0.05cm}\includegraphics[width=12cm]{images/misc/bandeau_bas.png}\\} % Pied de page avec le récif
}

% Style abstract
\fancypagestyle{abstract}{
	\fancyhf{}
	\setlength{\headheight}{20pt plus4mm minus4mm}
	\setlength{\headsep}{42pt}
	\renewcommand{\headrulewidth}{0pt}
}

% Style plain (texte basique)
\fancypagestyle{plain}{
    \fancyhf{}
    \setlength{\headheight}{20pt plus4mm minus4mm}
    \setlength{\headsep}{42pt}
    \renewcommand{\headrulewidth}{0pt}
    \fancyfoot[LE]{\numberbaleine}
    \fancyfoot[RO]{\numberbaleine}
    \fancyfoot[C]{\vspace*{0.05cm}\includegraphics[width=12cm]{images/misc/bandeau_bas.png}\\} % Pied de page avec le récif
}

% Style references
\fancypagestyle{references}{
	\fancyhf{}
	\setlength{\headheight}{20pt plus4mm minus4mm}
	\setlength{\headsep}{42pt}
	\renewcommand{\headrulewidth}{0pt}
	\fancyhead[RO]{\rbluebandreferences}
	\fancyhead[LE]{\lbluebandreferences}
    \fancyfoot[C]{\vspace*{0.05cm}\includegraphics[width=12cm]{images/misc/bandeau_bas.png}\\} % Pied de page avec le récif
	\fancyfoot[LE]{\numberbaleine} % numérotation baleine pages paires
	\fancyfoot[RO]{\numberbaleine} % numérotation baleine pages impaires
}


%%%%%%%%%%%%%%%% Style des captions
\renewcommand{\figurename}{Figure}
\DeclareCaptionFont{monvert}{\color{bleusection}}
\captionsetup[table]{labelsep=space,labelfont={monvert, bf}}
\captionsetup[figure]{labelsep=space,labelfont={monvert, bf}}
\setcounter{page}{1}
\newtheorem{definition}{Definition}
\newcommand{\brdefinition}{\begin{definition}}
	\newcommand{\erdefinition}{\end{definition}}



%%%%%%%%%%%%%%%%% Style des table des matières, figures etc.
\sectionfont{\normalfont}
\renewcommand*\contentsname{\Large \centerline{TABLE DES MATIÈRES}}
%\renewcommand{\cftchappresnum}{CH.\space} % Nom chapitre tel que dans le TOC
\setlength{\cftchapnumwidth}{\widthof{\textbf{CH~999~}}}

%\cftsetindents{chapter}{0em}{1em}      % set amount of indenting
%\cftsetindents{section}{0em}{1em}

%For reference
\usepackage{natbib}
%\bibliographystyle{apalike}%\usepackage{apalike}

\setlength{\bibhang}{15pt} % offset entre la première ligne de chaque référence et les autres lignes (pour mettre en relief le premier auteur)

%\setlength\bibindent{0em}
\renewcommand{\bibname}{Références bibliographiques} % Titre de la table bibliographique en en tête de chapitre (pas dans la toc)
%\renewcommand{\bibname}{\protect\centerline{References}}
%\newcommand{\setbibname}[1]{\bibname[#1]{\centering #1}}



%%%%%%%%%%%%%%%%% Abbréviations
% For Terms and Abbreviations 
\usepackage[acronym,section,nogroupskip]{glossaries}
% nogroupskip => empecher les regroupements de sigles par ordre alphabétique (AA BBB C DDD F GG...)
\makenoidxglossaries
%\usepackage[acronym,toc=false,section=section, nogroupskip]{glossaries}
%\makeglossaries
%\renewcommand*\glspostdescription{\cftdotfill{\cftsecdotsep}}
%\renewcommand{\glossarysection}[2][]{{\centering\bfseries\MakeTextUppercase{#2}\par}}


%For Table Column Width
\usepackage{array}
\newcommand{\PreserveBackslash}[1]{\let\temp=\\#1\let\\=\temp}
\newcolumntype{C}[1]{>{\PreserveBackslash\centering}m{#1}}
\newcolumntype{R}[1]{>{\PreserveBackslash\raggedleft}m{#1}}
\newcolumntype{L}[1]{>{\PreserveBackslash\raggedright}m{#1}}


\usepackage{etoolbox}

\makeatletter % Transforme le catcode de "@" de 12 à 11 (normal letters)
\patchcmd{\ttlh@hang}{\parindent\z@}{\parindent\z@\leavevmode}{}{}
\patchcmd{\ttlh@hang}{\noindent}{}{}{}
\makeatother % Remet le catcode de "@" de 11 à 12


%%%%%%%%%%%%%%%%% Styles abstract, statut,...
% Style auteur du papier
\newcommand{\authorpaper}[1]{
\begingroup
\leftskip0.2cm
\noindent\textbf{Authors:\space}#1
\vspace{0pt}
\par
\endgroup}

% Style statut de l'article
\newcommand{\status}[1]{
\begingroup
\leftskip0.2cm
\noindent\textbf{Paper status:\space}#1
\vspace{10pt}
\par
\endgroup}

% Style abstract
\newcommand{\abstract}[1]{
\begingroup
\leftskip0.2cm
\noindent\textbf{Abstract:\space}#1
\vspace{12pt}
\par
\endgroup}

% Style keywords
\newcommand{\keyword}[1]{
\begingroup
\leftskip0.2cm
\noindent\textbf{Keywords:\space}#1
\vspace{12pt}
\par
\endgroup
}

% Style résumé
\newcommand{\resume}[1]{
\begingroup
\leftskip0.2cm
\noindent\textbf{Résumé:\space}#1
\vspace{12pt}
\par
\endgroup}

% Style mots clés
\newcommand{\motclef}[1]{
\begingroup
\leftskip0.2cm
\noindent\textbf{Mot-clefs:\space}#1
\vspace{12pt}
\par
\endgroup
}
