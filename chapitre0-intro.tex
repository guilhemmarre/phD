\chapter{Introduction générale} \label{chap:c1}

\section{Fonctionnement des systèmes viticoles et enjeux}
\subsection{Vers une viticulture agroécologique}


\begin{Encadre}[!htb]
\caption{La viticulture française en quelques chiffres} \label{c1:boite1}
%\begin{adjustwidth}{-1cm}{+1cm}
%\centering\fbox{\parbox{\textwidth}{
\begin{tcolorbox}[skin=widget,
boxrule=1mm,
coltitle=black,
colframe=blue!45!white,
colback=blue!15!white,
width=(\linewidth),before=\hfill,after=\hfill,
adjusted title={La viticulture française en quelques chiffres : économie, surfaces et types de production}]
\begin{description}[leftmargin=0pt]
\item[Économie] La France, l'Italie et l'Espagne sont les trois premiers producteurs de vin (en volume) du monde, la première place étant attribuée à l'un ou l'autre selon les années \citep{franceagrimer_production_2018,franceagrimer_les_2018}. En 2016, la France a produit 44,3 millions d'hectolitres (hL), représentant environ 16\% de la production mondiale. Dans l'économie nationale française, les vins et spiritueux représentent le 1er poste excédentaire de la balance commerciale française pour les produits agroalimentaires (39\% des vins exportés en 2016), et le deuxième poste excédentaire tous produits confondus derrière le secteur aéronautique et spatial \citep{lemoyne_commerce_2018}. La France est également le deuxième consommateur mondial de vin, derrière les États-Unis \citep{franceagrimer_production_2018,franceagrimer_les_2018}.
\item[Surfaces viticoles] La viticulture représente 2,7\% de la \gls{surf} française, et représente la deuxième orientation technico-économique des exploitations françaises à l'échelle du territoire, derrière les grandes cultures (14\% des exploitations, \citet{agreste_repartition_2017,agreste_agreste_2017}). Parmi les exploitations viticoles françaises, 34\% des exploitations ont une surface inférieure à 15 ha, et 62\% des exploitations ont une surface inférieure à 30 ha \citep{agreste_pratiques_2017}.
\item[Types de productions] À la vendange 2016, 58\% des productions de raisin de cuve françaises ont reçu une \gls{aoprot}. Les appellations définissent autant de cahier des charges imposant certaines pratiques viticoles, et limitant les rendements dans la plupart des cas, afin de garantir un niveau de qualité du vin prédéfini. Sur le territoire national, 8\% des exploitations viticoles pratiquent une agriculture biologique,  biodynamie certifiée, et autres modes biologiques certifiés et non certifiés). La part d'exploitations pratiquant une viticulture biologique est variable selon les régions, elle est par exemple de 8\% en Languedoc hors \gls{po}, de 19\% dans les Bouches-Du-Rhône et seulement d'1\% en  Champagne \citep{agreste_pratiques_2017}.
\end{description}
\end{tcolorbox}
%}}
%\end{adjustwidth}
\end{Encadre}



\subsubsection{Variabilité du rendement de la vigne}

\subsubsection{Élaboration du rendement}

Le rendement de la vigne s'élabore, comme beaucoup d'espèces fruitières, sur deux cycles consécutifs : l'année $n-1$ (campagne précédente) et l'année $n$ (année de la récolte). 


\subsubsection{Effet d'un enherbement sur les bilans hydriques et azotés}

\subsection{L'entretien des sols au cœur des enjeux en viticulture }
\subsubsection{Érosion et dégradation des sols viticoles}

\subsubsection{Pression phytosanitaire dans les vignobles}

\subsubsection{Changement climatique}

\subsubsection{Enherber pour répondre aux enjeux viticoles ?}


\newpage

\section{Pilotage des cultures de services pour la fourniture de services écosystémiques en vignoble}

\noindent\textit{Cette partie a fait l'objet d'une publication dans la revue Agriculture, Ecosystem \& Environment :}

\small

\noindent{\href{https://doi.org/10.1016/j.agee.2017.09.030}{Garcia, L., Celette, F., Gary, C., Ripoche, A., Valdés-Gómez, H., Metay, A., 2018. Management of service crops for the provision of ecosystem services in vineyards: A review. Agriculture, Ecosystems \& Environment 251, 158–170}}

\noindent{\textit{Version auteur :} \url{https://hal.archives-ouvertes.fr/hal-01614417v2/document}}

\normalsize

\myparagraph{Abstract}

Service crops are crops grown with the aim of providing non-marketed ecosystem services, i.e. differing from food, fiber and fuel production. Vineyard soils face various agronomic issues such as poor organic carbon levels, erosion, fertility losses, and numerous studies have highlighted the ability of service crops to address these issues.

\paragraph{Keywords}

Cover crop, sustainable viticulture, agroecology, ecosystem services, trade-off management

\subsection{Introduction}


\subsection{Services and disservices of service crops in vineyards}

\subsubsection{Supporting and regulating services for viticulture}

\paragraph{Soil physical properties and water budget}

\paragraph{Soil chemical fertility}

\paragraph{Regulation of pests and natural enemies}

\subsubsection{Environmental and cultural services}

\paragraph{Conservation of biodiversity and wildlife}	

\begin{figure}[htb]
	\begin{center}
		\includegraphics[height=6cm,keepaspectratio]{paysage2}
			\caption[Intercropped vineyard with flowering spontaneous vegetation in south of France]{Intercropped vineyard with flowering spontaneous vegetation in south of France (\textcopyright Hélène Frey)}
		\label{c1:fig7}
       \end{center}
\end{figure}

\subsection{A framework to handle the complexity of managing service crops: services \textit{vs.} disservices, context and timing}
\subsubsection{Managing the balance between services and disservices}

\myparagraph{Managing support services: the example of soil water and nitrogen availability}

\subsection{Conclusions}

\myparagraph{Acknowledgements}
The authors are grateful to Elaine Bonnier for English language corrections, and Hélène Frey for her beautiful picture. This research benefited from research activities carried out in the FertilCrop project, in the framework of the FP7 ERA-Net programme CORE Organic Plus. 

\newpage

\section{Ecologie fonctionnelle, approche par les traits des plantes}

\subsection{Diversité et fonctions des plantes dans les écosystèmes}

\subsubsection{Définition des fonctions des plantes et des écosystèmes}

Ce modèle a fait l'objet de multiples débats scientifiques (Craine, 2007, 2005; Enquist, 2010; Grace, 1991; Grime, 2007; Tilman, 2007), mais est cependant encore mobilisé dans des études récentes (e.g. Storkey, 2006; Storkey et al., 2013). 

\subsection{Approche fonctionnelle, ou "trait-based approach"}


\subsubsection{Approche fonctionnelle basée sur les traits et marqueurs}

\subsubsection{L'approche fonctionnelle pour l'étude des services écosystémiques}

\subsubsection{Cadre conceptuel pour l'approche fonctionnelle en agroécologie}

\subsection{Problématique, questions de recherche et objectifs scientifiques}

\medskip
\noindent\textbf{\textit{Les marqueurs fonctionnels des espèces végétales présentes dans les enherbements viticoles permettent-ils d'expliquer l'impact de la communauté végétale sur les principaux services de support en viticulture ? L'étude des liens entre marqueurs fonctionnels et services de support permet-elle de sélectionner et piloter les espèces végétales pour maximiser la fourniture de services ?}}
\medskip

\noindent{Cette problématique s'est traduite en plusieurs questions de recherche :}

\begin{enumerate}[leftmargin=*]
\item \textbf{\textit{Quelles relations peut-on mettre en évidence entre les marqueurs fonctionnels des cultures de services en système viticole, et les services écosystémiques de support (protection des sols, fourniture en ressources hydriques et azotées) qu'elles rendent dans ces agrosystèmes ?}}
\item \textbf{\textit{Dans quelle mesure les marqueurs fonctionnels des communautés permettent-ils d'expliquer les services écosystémiques réalisés par les cultures de services dans ces agrosystèmes ?}}
\item \textbf{\textit{Peut-on piloter les services rendus par les cultures de services par le choix des espèces planifiées et de leurs traits, et par des interventions techniques au cours de leur cycle de croissance ?}}
\end{enumerate}

Pour y répondre, nous posons deux hypothèses de travail : $i)$ l'approche fonctionnelle par les traits des espèces et marqueurs fonctionnels des communautés est générique et peut être utilisée dans les systèmes viticoles enherbés, et $ii)$ la réalisation des services peut être évaluée par des indicateurs de fonctionnement du système sol-vigne-culture de service (stabilité structurale, couverture du sol, état hydrique et azoté du sol). 

\medskip

\noindent \textit{Les hypothèses suivantes seront testées dans cette thèse :}

\begin{description}
\item[Hypothèse 1]\label{c1:h} il existe des liens statistiques entre les marqueurs fonctionnels des communautés végétales composant les enherbements, leurs fonctions et les services qu'elles rendent aux viticulteur$\cdot$rice$\cdot$s.
\item[Hypothèse 2] : les marqueurs fonctionnels des plantes sont représentatifs du fonctionnement des espèces, et permettent de comparer les espèces et les communautés qu'elles composent en termes d'impacts sur le fonctionnement de l'agrosystème.
\item[Hypothèse 3] : les opérations techniques affectant les communautés pendant leur cycle de croissance permettent de modifier le niveau de réalisation des services 

\end{description}

En conséquence, afin de répondre à ces questions de recherche et tester les hypothèses définies, les objectifs de cette thèse sont les suivants (Fig. \ref{c2:fig1}) :
\begin{enumerate}
\item la description des propriétés fonctionnelles de différentes communautés composées d'espèces planifiées et d'espèces spontanées, en particulier du point de vue de leur impact sur les ressources (eau, azote) et la structure du sol (stabilité) ;
\item l'évaluation et la description au champ des fonctions des cultures de services associées aux services de stabilisation des sols, de limitation du ruissellement, de fourniture en eau, et au service d'engrais vert ;
\item l'identification de valeurs de marqueurs fonctionnels des communautés permettant de placer le système dans une zone de compromis favorables entre services et dysservices ;
\item l'identification de leviers d'action techniques permettant le pilotage des services et dysservices par des interventions sur les cultures de services.
\end{enumerate}

La partie suivante présente la démarche générale de la thèse ainsi que les expérimentations mises en places et les mesures réalisées pour répondre aux objectifs ci-dessus.
