%\chapter{Introduction générale} \label{Introduction générale}


\section{Les habitats benthiques Méditerranéens}
\subsection{Le contexte Méditerranéen: un hot spot de biodiversité}

\subsection{Les herbiers de Posidonie}
\subsubsection{Un habitat sensible}
\subsubsection{Suivi des herbiers de Posidonie en Méditerranéen française}

Service crops are crops grown with the aim of providing non-marketed ecosystem services, i.e. differing from food, fiber and fuel production. Vineyard soils face various agronomic issues such as poor organic carbon levels, erosion, fertility losses, and numerous studies have highlighted the ability of service crops to address these issues.

\noindent\textit{Cette partie a fait l'objet d'une publication dans la revue Agriculture, Ecosystem \& Environment :}

\subsection{Les récifs coralligènes}
\subsubsection{Petits frères des récifs coralliens}
\subsubsection{Suivis des récifs coralligènes en Méditerranée française}

\section{La photogrammétrie sous marine : principes et contraintes}
\subsubsection{La photogrammétrie}
\subsubsection{Principes théoriques}
\subsubsection{Acquisition des images}
\subsubsection{Résolution et précision des modèles}

\subsection{Particularités inhérentes au milieu marin}
\subsubsection{Illumination naturelle variable et faible}
\subsubsection{Les problèmes de la réfraction}
\subsubsection{Présence d'objets mobiles sur la scène}

\subsection{Utilisation de la photogrammétrie en écologie marine}
\subsubsection{Assemblages 2D}
\subsubsection{Modèles 3D}

\newpage

\section{Analyses d'images par apprentissage profond}
\subsection{Historique des réseaux de neurones convolutifs}
\subsubsection{Du neurone au réseau de neurones}
\subsubsection{Analyses d'images: les convolutions}
\subsubsection{Vers des réseaux de plus en plus profonds}

\newpage

\subsection{Application pour la reconnaissance d'espèces}
\subsubsection{Des données généralement complexes}

\small

\noindent{\href{https://doi.org/10.1016/j.agee.2017.09.030}{Garcia, L., Celette, F., Gary, C., Ripoche, A., Valdés-Gómez, H., Metay, A., 2018. Management of service crops for the provision of ecosystem services in vineyards: A review. Agriculture, Ecosystems \& Environment 251, 158–170}}

\noindent{\textit{Version auteur :} \url{https://hal.archives-ouvertes.fr/hal-01614417v2/document}}

\normalsize

\subsubsection{Stratégies d'optimisation des performances}
\subsubsection{Reconnaissance d'espèces de corail: un cas d'étude similaire}

\paragraph{Soil physical properties and water budget}

\paragraph{Soil chemical fertility}


\textbf{Acknowledgements}
The authors are grateful to Elaine Bonnier for English language corrections, and Hélène Frey for her beautiful picture. This research benefited from research activities carried out in the FertilCrop project, in the framework of the FP7 ERA-Net programme CORE Organic Plus. 

\section{Problématique de recherche}


\medskip
\noindent\textbf{\textit{Les marqueurs fonctionnels des espèces végétales présentes dans les enherbements viticoles permettent-ils d'expliquer l'impact de la communauté végétale sur les principaux services de support en viticulture ? L'étude des liens entre marqueurs fonctionnels et services de support permet-elle de sélectionner et piloter les espèces végétales pour maximiser la fourniture de services ?}}
\medskip

\noindent{Cette problématique s'est traduite en plusieurs questions de recherche :}

\begin{enumerate}[leftmargin=*]
\item \textbf{\textit{Quelles relations peut-on mettre en évidence entre les marqueurs fonctionnels des cultures de services en système viticole, et les services écosystémiques de support (protection des sols, fourniture en ressources hydriques et azotées) qu'elles rendent dans ces agrosystèmes ?}}
\item \textbf{\textit{Dans quelle mesure les marqueurs fonctionnels des communautés permettent-ils d'expliquer les services écosystémiques réalisés par les cultures de services dans ces agrosystèmes ?}}
\item \textbf{\textit{Peut-on piloter les services rendus par les cultures de services par le choix des espèces planifiées et de leurs traits, et par des interventions techniques au cours de leur cycle de croissance ?}}
\end{enumerate}

Pour y répondre, nous posons deux hypothèses de travail : $i)$ l'approche fonctionnelle par les traits des espèces et marqueurs fonctionnels des communautés est générique et peut être utilisée dans les systèmes viticoles enherbés, et $ii)$ la réalisation des services peut être évaluée par des indicateurs de fonctionnement du système sol-vigne-culture de service (stabilité structurale, couverture du sol, état hydrique et azoté du sol). 

\medskip

\noindent \textit{Les hypothèses suivantes seront testées dans cette thèse :}

\begin{description}
\item[Hypothèse 1]\label{c1:h} il existe des liens statistiques entre les marqueurs fonctionnels des communautés végétales composant les enherbements, leurs fonctions et les services qu'elles rendent aux viticulteur$\cdot$rice$\cdot$s.
\item[Hypothèse 2] : les marqueurs fonctionnels des plantes sont représentatifs du fonctionnement des espèces, et permettent de comparer les espèces et les communautés qu'elles composent en termes d'impacts sur le fonctionnement de l'agrosystème.
\item[Hypothèse 3] : les opérations techniques affectant les communautés pendant leur cycle de croissance permettent de modifier le niveau de réalisation des services 

\end{description}

En conséquence, afin de répondre à ces questions de recherche et tester les hypothèses définies, les objectifs de cette thèse sont les suivants :
\begin{enumerate}
\item la description des propriétés fonctionnelles de différentes communautés composées d'espèces planifiées et d'espèces spontanées, en particulier du point de vue de leur impact sur les ressources (eau, azote) et la structure du sol (stabilité) ;
\item l'évaluation et la description au champ des fonctions des cultures de services associées aux services de stabilisation des sols, de limitation du ruissellement, de fourniture en eau, et au service d'engrais vert ;
\item l'identification de valeurs de marqueurs fonctionnels des communautés permettant de placer le système dans une zone de compromis favorables entre services et dysservices ;
\item l'identification de leviers d'action techniques permettant le pilotage des services et dysservices par des interventions sur les cultures de services.
\end{enumerate}

La partie suivante présente la démarche générale de la thèse ainsi que les expérimentations mises en places et les mesures réalisées pour répondre aux objectifs ci-dessus.
