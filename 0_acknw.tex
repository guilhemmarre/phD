
\phantomsection
\addcontentsline{toc}{section}{REMERCIEMENTS}
{\centerline { {\sffamily \Large REMERCIEMENTS}}}

\vspace*{1cm}
\vskip 0.5cm
\noindent

J’ai toujours trouvé quelque peu ridicule la comparaison du rendu d’un manuscrit de thèse avec un accouchement, mais bien que la biologie ne me permette pas de faire l’expérience des deux, force est de constater qu’à seulement quelques heures de l'échéance, mes sentiments sont plutôt confus… Consécration d’un travail achevé ou frustration de ne pas avoir pu aller plus loin ? Joie de la libération et excitation de l’avenir ou nostalgie d’une période qui se termine et questionnements existentiels ? Quoiqu’il en soit, la thèse est décidément une expérience qui ne laisse pas indemne… Mais comme souvent, ce qui compte le plus ce sont les gens avec qui on partage ces moments importants…

« Je souhaite porter un toast ! »… mais par qui commencer ?

Tout d’abord, je tiens à remercier Florian, Pierre et Laurent, pour m’avoir accordé leur entière confiance dans la réalisation de ce travail de recherche. Sans votre soutien, je n’aurais probablement jamais fait de thèse, moi qui ne me destinais pas spécialement à la recherche et encore moins à l’épreuve du travail doctoral… J’ai commencé ma carrière andromedienne à remonter des bennes à sédiment, j’en ressors 4 ans plus tard avec une thèse, comme quoi tout est possible !

Bien sûr, je ne saurais que trop remercier mes chères et tendres directrices Julie et Sandra. Vous avez toujours été là pour moi, toujours de bons conseils, d’une efficacité redoutable et d’un optimisme inébranlable ! Tous les doctorants n’ont pas la chance d’avoir pareil encadrement, j’en suis bien conscient, et si ces trois années se sont si bien passées c’est en grande partie grâce à vous. 

Un grand merci également à vous tous qui avez enrichi ma réflexion scientifique et apporté un regard extérieur à mon travail. Je pense bien sûr aux membres de mon comité de thèse – Christiane Weber, Nicolas Mouquet, Gérard Subsol, François Guilhaumon et Maria Dornelas – pour votre bienveillance, votre disponibilité et vos précieux conseils, mais aussi à ceux qui m’ont accordé de leur temps pour parler deep learning – Dino Ienco, Jérôme Pasquet et Rémi Cresson – sans vous on ne serait pas allé aussi loin et aussi vite avec Cédric. En parlant du loup, un grand merci à Cédric De Almeida Braga qui a fait un super stage de M2. Ces quelques mois avec toi ont été un vrai plaisir, je te souhaite de te régaler dans ta thèse (à ton tour !) et dans ta vie rennaise.

Je souhaite bien évidemment remercier l’Agence de l’eau Rhône-Méditerranée-Corse pour son soutien financier et logistique dans ce projet de thèse et les réseaux de surveillance, sans quoi je n’aurais pas eu accès à autant de matière première pour mes recherches. Un grand merci à Pierre Boissery pour les échanges toujours riches d’enseignements, ses récits d'aventures et son regard pertinent concernant la gestion des habitats marins côtiers.

Comment écrire ses remerciements sans parler de ceux avec qui j’ai partagé mon quotidien au bureau ou en mer ? Un grand merci à tous les valeureux andromediens – Gwen, Marie, Anto, Sté, Agathe, Thomas, Seb, Célia, Caro et Sylvie – pour la bonne humeur au bureau et sur le terrain, vous êtes une deuxième famille! Et bien sûr merci aussi à tous les indep’ qu'on voit souvent – Thomas, Yaya, Nono, Mollon, Thibault, Justine, Mathieu… – pour avoir partagé avec vous tous ces moments de galère, de rires, ces belles plongées, ces couchers de soleil, ces mers démontées… J’ai énormément appris à vos côtés, techniquement et humainement, merci à vous !

Comme je suis chanceux, j’ai deux bureaux, donc deux fois plus de collègues ! Un grand merci à Florian et Marc pour ces deux ans et demi de cohabitation à la Maison de la Télédétection. C’était un plaisir de partager le bureau avec vous ! Il m’a semblé bien vide à votre départ… Merci Marc pour ton template LaTeX qui m'a permis de gagner beaucoup de temps et de m'y mettre sur le tard...  Mention spéciale aussi à mon cher Arthur pour être régulièrement monté nous voir pour discuter sciences et refaire le monde autour d’un café. Et bien sûr merci à tous les doctorants – Marc, Jacques, Christian, Eric, Dav, Hugo, Milo, Sarah, Fred… – pour l’ambiance agréable de travail et les pauses déjeuner au CIRAD. Enfin, merci à Isa pour l’accueil toujours aussi chaleureux dès l’arrivée à la MTD et à Baron pour les discussion socio-politiques !

Même s’il aura été une période compliquée pour tout le monde, y compris pour moi, je dois bien tirer mon chapeau au covid19 pour m’avoir permis d’avancer largement sur le travail de rédaction… Quoi de mieux qu’un confinement de deux mois pour finir de rédiger sa thèse ? Le timing semblait presque parfait… Merci !

Et bien sûr un grand merci à tous les copains qui sont une part plus qu’importante de ma vie. Merci à vous pour ces grosses soirées à l’Espantade, ces sessions kite, ces danses swing enflammées, ces festivals de fanfare, ces belles randos… merci pour cet oxygène que vous m’apportez au quotidien ! Un grand merci à toi ma Chonchon d’être toujours là pour moi, pour le meilleur comme pour le pire, depuis maintenant 12 ans ! Merci à mes fidèles colocataires de l’Espantade – Mimine, Alexia, Franz et Renan, mais aussi Thibaut, Claire, Mélo, Agathe… – de contribuer à ce bon vivre si spécial à notre belle maison. Merci aussi à toi papa pour ton aide toujours précieuse en bricolage, et merci à vous maman et Fannette d’être là dès que je vous sollicite pour un petit pépin de santé. Merci à vous trois d’être là, tout simplement. 

Ámbar, j’ai eu tant de plaisir à t’écouter chanter, sans imaginer une seule seconde qu’un jour nos chemins se croiseraient… Merci pour cette belle année passée à tes côtés, pour m'avoir supporté dans les deux sens du terme, tu es une personne magnifique et sensible. Je te souhaite de trouver le dosage entre musique et philosophie qui te rendra pleinement heureuse. Je suis fier de toi, et j'espère te revoir vite sur scène et en dehors.

Enfin, merci à ceux qui ont bien voulu relire ma thèse pour y déceler les dernières coquilles –  Renan, Gwen et Sylvie – et un grand merci aux deux valeureux rapporteurs qui ont accepté d’évaluer ce travail de thèse – Joaquim Garrabou et Thomas Corpetti – ainsi qu’aux autres membres du jury qui évalueront la soutenance – Stéphanie Manel, Sylvie Gobert et Sébastien Villéger. Merci à Laurent Ballesta pour les magnifiques photos qui permettent de donner un peu de couleur à ce manuscrit.

En vous souhaitant bonne lecture…


%Word cloud joli: https://wordart.com/create ou http://r-graph-gallery.com/
%With text mining to keep only keywords:
%http://www.sthda.com/english/wiki/text-mining-and-word-cloud-fundamentals-in-r-5-simple-steps-you-should-know#step-3-text-mining

\vskip 0.3cm
\noindent
 \qquad  \qquad \qquad \qquad \qquad \qquad \qquad \qquad \quad \textbf{Guilhem}
